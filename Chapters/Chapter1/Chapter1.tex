\part{Classical transport}\label{part:1}
\chapter{Transport equations}\label{sec:transportequations}
Imagine an infinitely large billiard table.
A particle---represented by the white billiard ball---moves frictionlessly along straight 
lines until undergoing elastic, instantaneous collisions with background 
obstacles---the black billiard balls---that do not move due to their infinite weight and neglectable velocity.
The situation is depicted in
Figure \ref{fig:particlebilliardtable}.

\begin{figure}[h!]
	\centering
	\includegraphics[width=1.0\linewidth]{Chapters/Chapter1/figs/billiardtable/particlebilliardtable}
	\caption{The white ball undergoes elastic collisions on the billiard table while the black balls' positions are fixed.}
	\label{fig:particlebilliardtable}
\end{figure}

This in many ways simplified microscopic description of particle interactions will serve as the 
starting point for this thesis. Despite its simplicity, it allows the derivation of governing 
mesoscopic and macroscopic equations, as well as the construction of one of the most 
prominent numerical methods in transport theory---the Monte Carlo method. Furthermore, 
changing the microscopic picture will motivate non-classical transport, discussed in the 
second part of this thesis.

A magnitude of real-world phenomena and topics can be formulated in the language of kinetic 
theory. These include, but are not limited to, the theory of gases; radiation therapy for cancer 
treatment; modeling of nuclear reactors; or illumination in movies and computer games.

Transport equations try to achieve the following: Instead of describing a 
system by the behavior of every single molecule, every single particle, or every single light-ray, 
they seek a description in terms of statistical quantities that evolve in time. This is both 
necessary and often desirable. Necessary, since it already becomes computationally 
impossible 
to evolve the trajectory of each of the roughly $6.022\cdot 10^{23}$ particles in one mole.
Desirable, since it is generally of no interest to know each of these trajectories individually.
Thus, transport equations are convenient tools that distill complex physical systems to 
manageable equations.

Throughout the scope of this thesis we are going to consider uncharged particle 
transport; 
that is, particles do not interact with another via long range interactions---like electrons 
do---but with the background medium or through direct collisions with another.
Given these assumptions, particles move through phase space according to Newton's laws of motion
\begin{align}
	\begin{cases}
		\dot{\bx}(t) & = \bv(t),                          \\
		\dot{\bv}(t) & = \frac{1}{m} {\bf{F}}(t,\bx(t),\bv(t)), \\
	\end{cases}
	\label{eq:NewtonsLaw}
\end{align}
with $\bx(t),\bv(t) \in \RThree$, $t\in\Rgeqz$, $m\in \Rgz$, ${\bf{F}}: \Rgeqz \times \RThree \times \RThree \to \RThree$ sufficiently smooth, and subject to initial conditions $\bx(0) = \bx_0$ and $\bv(0) = \bv_0$. If we again assume that particles move with a much smaller mass and faster than background obstacles, and if we also neglect particle-particle interactions, it is easy to track particle trajectories through the obstacle field as already seen in  Figure \ref{fig:particlebilliardtable}.

If the number of particles becomes large, tracking every single trajectory is  infeasible and, more importantly, not necessary since the distribution of particles in phase space is usually sufficient information. Transitioning from the microscopic viewpoint to the mesoscopic viewpoint is rigorously formalized in the Boltzmann-Grad limit  \cite{grad1958principles} and will be explained in a following section. From it follows the definition of the particle number $N:\Rgeqz \times \RThree \times \RThree \to \Rgeqz$. Given an incremental volume $d\bx$ and a normalized incremental angle  $d\bv$, we define
\begin{equation}
	\begin{aligned}
		N(t,\bx,\bv)\,d\bx \,d\bv \defas & \text{ ``the expected number of particles in a phase space element } d\bx\, d\bv \\
						 & \text{ \, about }\bx \text{ and } \bv\text{ at time }t\text{.}"
						 \label{eq:DefinitionNWithWords}
	\end{aligned}
\end{equation}
With this fundamental variable, we can define the angular flux
$\psi:\Rgeqz \times \RThree \times \RThree \to \Rgeqz$ by $\psi(t,\bx,\bv)=v N(t,\bx,\bv)$, where $v$ is the particle's speed and $\bv$ the particle's direction of flight. According to Prinja and Larsen \cite{prinja2010general}, we can interpret  this quantity as 
\begin{equation}
	\begin{aligned}
		\psi(t,\bx,\bv) \,d\bv \, dS\defas & \text{ ``the absolute rate at which particles in } d\bv \text{ about } \bv \\
		&\text{ \, travel through the surface increment } dS \text{ orthogonal to } \bv \\
		& \text { \, at time } t."
		\label{eq:DefinitionPsiWithWords}
	\end{aligned}
\end{equation}
%Here, $||\cdot||_{2}$ denotes the standard $L^2$ norm.
Knowing the angular flux, we can derive moments by integrating $\psi$ against $\bv^i$ over the space of possible velocities. For example, the zeroth-order moment is the scalar flux
\begin{equation}
	\begin{aligned}
		\phi(t,\bx) & \defas
		% \text{ ``the rate of particles at time }t\text{ and position }\bx" \\
			    %& = 
			    \int_{\RThree}\psi(t,\bx,\bv) \, d\bv,
			    \label{eq:DefinitionPhiWithWords}
	\end{aligned}
\end{equation}
commonly referred to as the (particle) \textit{density}.
We are now going to discuss the governing equations that describe how the angular flux evolves in time, given an initial distribution of particles and varying assumptions about the underlying physical context.

Parts of the following sections are inspired by the classical books of Chandrasekhar \cite{chandrasekhar2013radiative} and Lewis and Miller \cite{lewis1984computational}; the lecture notes of Frank \cite{kinetictheory201819}; and a collection of lecture notes, edited by Bellomo \cite{bellomo1995lecture}.

\section{Microscopic viewpoint}
Before  discussing transport problems from the mesoscopic perspective, the microscopic viewpoint deserves further consideration.
Let $\mathcal{C}_r \defas \{{\bf{c}}_i=(x_i,y_i)\}_{i=1,...,N}$ be the set of $N$ circles, centered at positions $(x_i,y_i)$, each with radius $r$.  For any prescribed initial position $\bx_0=(x_0,y_0)$ and unit velocity $\bv_0 =(v_x,v_y)$,  the trajectory of a particle with radius $r$ is unambiguous.
Let $t_n$ denote the time---and therefore the distance, because  the particle travels with unit 
speed---until undergoing the $n$-th collision. Furthermore, let $i(n)$ denote the index of the 
obstacle that causes collision $n$. With $\bx(t_n^-)$ and $\bv(t_n^-)$ as the pre-collision 
position and velocity, the post-collision position and velocity are given by
\begin{subequations}
\begin{align}
	\bx(t_n^+) & = \bx(t_n^-) ,                                     \\
	\bv(t_n^+) & = {\bf{T}}({\bf{c}}_{i(n)},\bx(t_n^-),\bv(t_n^-)).
\end{align}
\end{subequations}
\begin{figure}[h!]
	\centering
	\includegraphics[width=0.99\linewidth]{Chapters/Chapter1/figs/collisiontube/cylinder.pdf}
	\caption{The collision tube for a particle that underwent a collision at $(x_n,y_n)$ moving to the right (post-collision). After a distance $\delta t_n$, the next collision takes place with the obstacle at $(x_i,y_i)$.
	}
	\label{fig:cylinder}
\end{figure}


The operator $\mathbf{T}$ results from geometric considerations and simply translates incoming into outgoing velocities in compliance with Newtonian mechanics.
Consequently, $\delta t_n \defas t_n - t_{n-1}$ is the time (or equivalently distance) between collisions. The distribution of the $\delta t_n$ follows from a geometric argument, sketched in Figure \ref{fig:cylinder}: For a particle to travel distance $\delta t_n$ between consecutive collisions, there  can not be an obstacle with position ${\bf{c}}_i$ that overlaps with the tube of width $2r$ and length $\delta t_n$ (shown in Figure \ref{fig:cylinder} with dashed lines).
For a reference domain of area $A$ with $N$ obstacles, the likelihood of \textit{not} finding the center of an obstacle in a tube of length $\delta t$ and width $4r$ (\ie we shrink the obstacles to points and double the radius of the particle) is given by
\begin{align}
	Q_{r,A,N}(\delta t) = \left(1-\frac{4 r \delta t}{A}\right)^N.
\end{align}
%where we neglect potential overlapping of obstacles due to shrinking them to their centers.
Define $4r\cdot N / A = \sigma_t$ as the total cross section---the dimensionless number that represents the likelihood to scatter---to get
\begin{align}
	Q_{r,A,N}(\delta t) = \left(1-\frac{ \sigma_t\, \delta t}{N}\right)^N.
\end{align}
For a fixed domain size $A$ we can let $N\rightarrow \infty$ and $r\rightarrow 0$ such that $\sigma_t$ is constant to get
\begin{align}
	Q_{\sigma_t}(\delta t) = e^{-\sigma_t \,\delta t}.
\end{align}
With $Q_{\sigma_t}(\delta t)$ as the probability of particles \textit{not} colliding before traveling a distance $\delta_t$, $1-Q_{\sigma_t}(\delta_t)$ is the probability of particles traveling \textit{at most} a distance $\delta_t$.
Differentiating $1-Q_{\sigma_t}(\delta_t)$, we define 
 %Then $1-Q_{\sigma_t}(\delta_t)$ is the cumulative distribution function (CDF) or 
 the path length distribution $p_{\sigma_t} (\delta t)$  as
\begin{equation}
	\begin{split}
		p_{\sigma_t} (\delta t) \defas & \text{ ``the probability that a particle will travel a distance }
		\delta t \\ &\text{ \, between consecutive collisions in a random obstacle field with }\\
			    &\text{ \, total cross section } \sigma_t" \\
		=        & \sigma_t e^{-\sigma_t \delta t}.
		\label{eq:pdfprediction}
	\end{split}
\end{equation}
Another physical interpretation of the total cross section $\sigma_t$ is given by its inverse 
relation to the mean free path (MFP), \ie the mean distance that particles can travel in the 
obstacle field of Figure \ref{fig:particlebilliardtable}. This can be confirmed numerically. 
The 
software  \texttt{PLE.jl}\footnote{\texttt{PLE.jl}, a path length estimator that I implemented in the Julia programming language, is available under the MIT license at  \href{https://github.com/camminady/PLE.jl}{https://github.com/camminady/PLE.jl}.} allows, among other things, to efficiently trace particles through an 
ensemble of obstacles, interacting with the obstacles on the basis of Newtonian mechanics.
Consider therefore a random ensemble of obstacles with centers in $[0,1]\times[0,1]$. 
 A 
sample simulation of the first ten collisions of a single particle in a periodic domain is 
visualized in Figure  \ref{fig:obstaclegame}. When we increase the number of obstacles and 
simultaneously shrink their radii, the theoretical results of the Boltzmann-Grad limit can be 
reproduced numerically. Figure \ref{fig:histogram} summarizes the trajectory of a particle 
over 100000 collisions in a field of 10000 obstacles with $\sigma_t=4$. Together with the data 
(green) we show the best fit with ansatz $\lambda \exp{(-\lambda s)}$. As predicted by 
\eqref{eq:pdfprediction}, the best fit $\lambda$ is reasonably close to $\sigma_t$.




 
\begin{figure}
	\centering
	\includegraphics[width=0.6\linewidth]{Chapters/Chapter1/figs/pathlength/Bounce}
	\caption{Trajectory of a particle with $r=0$ in an obstacle field. The domain is 
	periodic and obstacles were allowed to overlap. The picture shows $180$ obstacles with 
	$\sigma_t=8$.  The white line overlaps with the obstacles 
	only as a result of plotting with a non-zero linewidth.}
	\label{fig:obstaclegame}
\end{figure}
\begin{figure}
	\centering
	\includegraphics[width=1.0\linewidth]{Chapters/Chapter1/figs/pathlength/stepkithistogram}
	\caption{A distribution for the distances between consecutive collisions with $\sigma_t=4$, 
	$10000$ obstacles, and $100000$ traced trajectories. Additionally to the histogram, we have 
	a best fit with ansatz $\lambda \exp{(-\lambda s)}$. The best $\lambda$ is close to the 
	expected value $\sigma_t=4$. Statistical noise becomes visible for long distances. }
	\label{fig:histogram}
\end{figure}





\section{Liouville equation}
We may ask ourselves why we would wish to describe a fully deterministic system---\ie a 
system following Newton's laws as described in \eqref{eq:NewtonsLaw}---in a probabilistic 
way---\ie using the definition from \eqref{eq:DefinitionPhiWithWords}. Clearly, the evolution of 
the distribution $\psi(t,\bx,\bv)$ should follow the same deterministic dynamics as with the 
microscopic viewpoint; otherwise the evolution of $\psi(t,\bx,\bv)$ in time would not relate to 
the evolution of the particle billiard at all.
% Instead, \enquote{[r]andomness is introduced only 
%in the initial data} \cite[p.~14]{spohn2012large}.
%\todo{This paragraph needs to be rewritten}
However, a compelling argument for the statistical description of the otherwise deterministic processes
is its inherent uncertainty. Not only does quantum physics imply uncertainty in the initial data, but the sheer impracticality of knowing every particle's position and velocity prompts an argument for a probabilistic approach.

The partial differential equation (PDE) that describes the evolution of $\psi(t,\bx,\bv)$ in time is called Liouville's equation. Consider a system of particles described by \eqref{eq:NewtonsLaw}. Let us define $\psi$ by the empirical measure as
\begin{equation}
	\psi(t,\bx,\bv) \defas \delta(\bx - \bx(t)) \delta(\bv-\bv(t)),
\end{equation}
with $\delta$ as the Dirac delta. If we now multiply with a test function $u:\mathbb{R}^3\times \mathbb{R}^3\rightarrow \mathbb{R}$, integrate over $\bx\in\RThree$ and $\bv\in\RThree$, and differentiate with respect to $t$, we obtain

\begin{equation}
	\begin{split}
		\frac{d}{dt} \int \int \psi(t,\bx,&\bv) u(\bx,\bv)\, d\bx \ d\bv
		\\
		=& \frac{d}{dt} u(\bx(t),\bv(t))
		\\
		=&\left\langle \frac{d\bx(t)}{d t},\nabla_{\bx(t)} u(\bx(t),\bv(t))\right\rangle + \left\langle \frac{d\bv(t)}{dt}, \nabla_{\bv(t)} u(\bx(t),\bv(t))\right \rangle
		\\
		=&
		\left\langle \bv(t),\nabla_{\bx(t)} u(\bx(t),\bv(t))\right\rangle + \left\langle \frac{1}{m}{\bf{F}}(t,\bx(t),\bv(t)), \nabla_{\bv(t)} u(\bx(t),\bv(t))\right \rangle
		\\
		=&
		\int \int \left(\left\langle \bv,\nabla_{\bx} u(\bx,\bv)\right\rangle + \left\langle \frac{1}{m}{\bf{F}}(t,\bx,\bv), \nabla_{\bv} u(\bx,\bv)\right \rangle \right) \\
		 &\, \cdot  \delta(\bx - \bx(t)) \delta(\bv-\bv(t)) \, d\bx\,  d\bv
		 \\
		=&
		\int \int \psi(t,\bx,\bv)\left(\left\langle \bv,\nabla_\bx u(\bx,\bv)\right\rangle + \left\langle \frac{1}{m}F(t,\bx,\bv), \nabla_\bv u(\bx,\bv)\right \rangle \right) \, d\bx \, d\bv.
		\label{eq:EmpiricalMeasureByParts}
	\end{split}
\end{equation}
Here, $\langle f, g \rangle$ denotes the scalar product $\int \int f(\bx,\bv)g(\bx,\bv) \, d\bx \, d\bv$.
Using the first and last term of \eqref{eq:EmpiricalMeasureByParts}, integrating by parts, and making use of the fact that the test function $u$ was chosen arbitrarily, yields Liouville's equation
\begin{align}
	\partial_t \psi(t,\bx,\bv) + \bv \cdot \nabla_\bx \psi(t,\bx,\bv) + \frac{1}{m}{\bf{F}}(t,\bx,\bv) \cdot \nabla_\bv \psi(t,\bx,\bv) = 0.
	\label{eq:LiouvillesEquation}
\end{align}
Liouville's equation describes the evolution of an initial distribution $\psi_0(\bx,\bv) \defas  \psi(0,\bx,\bv)$ in phase space under the presence of a non-contact force  ${\bf{F}}(t,\bx,\bv)$. Under certain assumptions, we can derive conservation of probability along characteristics \cite{babovsky1998boltzmann}.




\section{Boltzmann-Grad limit}
The idea of the Boltzmann-Grad limit that we have mentioned before was rigorously derived by Grad \cite{grad1958principles} and formalizes the transition from the microscopic particle billiard to the mesoscopic transport equation.


Clearly the microscopic viewpoint is reversible: At any given time $t$ we can stop the billiard flow, flip the velocity vector, let the process run for another time $t$, and return to the original arrangement of particles and obstacles.
Thus, when the positions of all obstacles $\mathcal{C}_r \defas \{{\bf{c}}_i=(x_i,y_i)\}_{i=1,...,N}$, as well as the initial distribution of particles  $\psi_0(\bx_0,\bv_0)$ are known, the time evolution of particles $\psi(t,\bx,\bv \,|\, \mathcal{C}_r)$ is known, too. The expression $\psi(t,\bx,\bv \,|\, \mathcal{C}_r)$ simply involves tracing back the particles' trajectories through the obstacle field to their origin.

The Boltzmann-Grad limit is then the ensemble average $\langle \psi(t,\bx,\bv \,|\, \mathcal{C}_r) \rangle_{\mathcal{C}_r}$ with $N\rightarrow \infty$ and $r\rightarrow 0$ such that $N\cdot r =\text{constant}$. It involves geometric arguments about the trajectories of particles and statistical arguments about the arrangement of obstacles. 

Several assumptions are necessary to arrive at the ensemble-averaged distribution: (i) Obstacles are placed independently. (ii) Situations where particles bounce forth and back between two (or more) obstacles infinitely often are omitted. (iii) Collisions involve exactly two participants, one particle and one obstacle.

Assumption (i) allows for obstacles to overlap. However, the likelihood for this to happen decreases when the obstacles' radii shrink and the contribution of overlapping obstacles will subsequently be dropped in the Boltzmann-Grad limit since it is of {higher order} (in $r$).

Interestingly, the resulting expression for the ensemble average is provably non-reversible, causing several controversies in the scientific community \cite{cercignani1998ludwig}. The non-reversibility follows from the non-decrease\footnote{We will actually see a non-increase in \textit{mathematical} entropy instead.} in entropy, discussed in Section \ref{sec:entropy}.







\section{Linear transport equation}
Modeling the evolution of the distribution function for the particle billiard shown in Figure \ref{fig:particlebilliardtable} can be done---in a first simplified setting---via the linear transport equation. We ignore the presence of an external force $\bf{F}$ and instead assume that particles only interact with the background obstacles. These interactions can have two possible outcomes: (i) Particles change their direction due to scattering, and (ii) particles are being absorbed. The probability of these two possible scenarios is determined by the cross sections $\sigma_s$ and $\sigma_a$, respectively. Together with $\sigma_t = \sigma_a+\sigma_s$, the probability to scatter in the case of an event is given by $\sigma_s / \sigma_t$ and the probability of being absorbed is given by $\sigma_a / \sigma_s$.
In a heterogeneous material, the cross sections will have an additional spatial dependency. Given certain materials, there might also be a dependency on the angle, \ie the likelihood to undergo an event depends on a particle's direction of travel.%direction with which particles travel.

Using all the information above and assuming particles with unit speed and velocity $\bOmega$, we obtain
%\footnote{Note that we will always use normalized spherical integration, \ie $\int_{\mathbb{S}^2} 1\,  d\bOmega' = 1$.} 
the linear transport equation as
\begin{equation}
	\begin{split}
		\partial_t \psi(t,\bx,\bOmega) +& \bOmega \cdot \nabla_\bx \psi(t,\bx,\bOmega) +\sigma_a \psi(t,\bx,\bOmega) \\
						&= {\sigma_s}\int_{\mathbb{S}^2}s(\bOmega \cdot \bOmega')\left(\psi(t,\bx,\bOmega') - \psi(t,\bx,\bOmega)\right)\, d\bOmega'+ q(t,\bx,\bOmega).
						\label{eq:lineartransport}
	\end{split}
\end{equation}
Let us dissect the equation above  into its main parts. The left-hand side of the equation is an 
advection-plus-absorption part: Particles move along straight lines while being absorbed at a 
certain rate. The right-hand side of the equation describes the gain and loss due to 
in- and out-scattering, as well as the contribution due to a source 
$q(t,\bx,\bOmega)$. The scattering kernel $s(\bOmega\cdot \bOmega')$ describes the 
probability that a particle with direction $\bOmega$ changes direction to $\bOmega'$ (or vice 
versa) in case of scattering. A common convention is
\begin{align}
	\bOmega =
	\begin{pmatrix}
		\Omega_x \\
		\Omega_y \\
		\Omega_z
	\end{pmatrix}
	=
	\begin{pmatrix}
		\sqrt{1-\mu^2} \cos(\phi) \\
		\sqrt{1-\mu^2} \sin(\phi) \\
		\mu
	\end{pmatrix},
\end{align}
with the \textit{azimuthal angle} $\phi$, the \textit{polar angle }$\theta$, and $\mu = \cos(\theta)$.
%Note that we write $\int_{4\pi} \cdot \, d\bOmega'$ to denote the 
%\textit{normalized} integration over the unit sphere.
%\footnote{This is a redefinition of the spherical integration, \ie $\int_{4\pi} \cdot \, d\bOmega' =1$ but $\int_{\mathbb{S}^2} \cdot \, d\bOmega' = 4\pi$.} over the unit sphere, \ie $\int_{4\pi} 1\, d\bOmega'=1$.
With a non-negative scattering kernel that integrates to unity, we can split the scattering operator
\begin{equation}
	\begin{split}
		S(\psi)(t,\bx,\bOmega) \defas
		\int_{\mathbb{S}^2}s(\bOmega \cdot \bOmega')\left(\psi(t,\bx,\bOmega') - 
		\psi(t,\bx,\bOmega)\right)\, d\bOmega'
	\end{split}
	\label{eq:scatteringoperator}
\end{equation}
into in-scattering
\begin{equation}
	\begin{split}
		S^+(\psi)(t,\bx,\bOmega) \defas
		\int_{\mathbb{S}^2}s(\bOmega \cdot \bOmega')\psi(t,\bx,\bOmega') \, d\bOmega'
	\end{split}
\end{equation}
and out-scattering
\begin{equation}
	\begin{split}
		S^-(\psi)(t,\bx,\bOmega)
		\defas
		\int_{\mathbb{S}^2}s(\bOmega \cdot \bOmega') d\bOmega' \psi(t,\bx,\bOmega) = \psi(t,\bx,\bOmega).
	\end{split}
\end{equation}
The simplest choice for the scattering kernel models isotropic scattering via $\displaystyle s(\bOmega \cdot 
\bOmega') = 1 \, /\, 4\pi$, \ie particles scatter into all directions equally likely. Another
choice is the Henyey-Greenstein kernel \cite{henyey1941diffuse}, satisfying the 
relation
\begin{equation}
	\begin{split}
		s_g(\bOmega' \cdot \bOmega) =\frac{1-g^2}{4\pi\left(1-2g \,\bOmega' \cdot 
		\bOmega+g^2\right)^{3/2}} = \sum_{n=0}^\infty  g^n \sum_{m=-n}^nY_n^m(\bOmega) \, 
		\overline{Y_n^m(\bOmega')},
	\end{split}
	\label{eq:HGkernelexpansion}
\end{equation}
where $Y_n^m$ are the spherical harmonic basis functions \cite{muller2006spherical} and the over-bar is used for complex conjugation.
The Henyey-Greenstein kernel is not the correct physical model for scattering but is commonly used for two reasons: (i) The parameter $g \in [-1,1]$ allows to tune the scattering from purely backward ($g=-1$) to isotropic scattering ($g=0$) and to purely forward ($g=1$), and (ii) it has the convenient aforementioned expansion in terms of spherical harmonics. The scattering kernel is visualized in Figure \ref{fig:henyeygreensteinpolar}
and the relation to the spherical harmonics is discussed in Section \ref{appendix:sphericalharmonics}.
\begin{figure}
	\centering
	\includegraphics[width=1\linewidth]{Chapters/Chapter1/figs/HGkernel/henyeygreensteinpolar.pdf}
	\caption{The Henyey-Greenstein scattering kernel $\log_{10} s_g(\cos(\theta))$ for 
	different values of $g$ 
	with $\cos\,\theta = \bOmega \cdot \bOmega'$. The radial component is the 
	$\log_{10}$-probability of scattering into the corresponding angle (relative to the 
	pre-scattering direction). For $g=0$ (orange), particles scatter isotropically. For 
	$g=1-\varepsilon$ with $0<\varepsilon\ll 1$, particles mostly scatter into the pre-scattering 
	direction (green).}
	\label{fig:henyeygreensteinpolar}
\end{figure}
%Properties of the linear transport equation are 
%Let us now investigate certain properties of the linear transport equation.

%\begin{lemma}[Positivity of the linear transport equation]
%	For given initial conditions and for an unbounded domain the solution remains positive. if 
%	$q$>0.
%\end{lemma}
%\begin{proof}

%\end{proof}
%\todo{finish this}
%\begin{lemma}[Conservation of mass if $\sigma_a=q=0$]

%\end{lemma}
%\begin{proof}

%\end{proof}

%existence and uniqueness

\subsection{Properties of the collision kernel and operator}
\label{subsec:propertiescollisionoperator}
Let us discuss the scattering operator from \eqref{eq:scatteringoperator}, given by
\begin{equation*}
	\begin{split}
		S(\psi)(t,\bx,\bOmega) \defas
		\int_{\mathbb{S}^2}s(\bOmega \cdot \bOmega')\left(\psi(t,\bx,\bOmega') - 
		\psi(t,\bx,\bOmega)\right)\, d\bOmega'
	\end{split}
\end{equation*}
and its scattering kernel $s(\bOmega\cdot \bOmega')$ more thoroughly.
For every $(t,\bx)\in \Rgeqz \times \RThree$, the operator $S$ defined by 
\eqref{eq:scatteringoperator} is self-adjoint from $L^2(\mathbb{S}^2)$ to
$L^2(\mathbb{S}^2)$. Moreover, we assume the scattering kernel to be bounded in the 
sense (cf. \cite{coulombel2005diffusion}) 
\begin{align}
	\sigma_s \int_{\mathbb{S}^2}\int_{\mathbb{S}^2} s(\bOmega\cdot\bOmega')^2 \, d\bOmega \, 
	d\bOmega' \leq K < \infty.
\end{align}
An additional assumptions is $s(\bOmega\cdot \bOmega')\geq 0$. In general, the scattering 
kernel
can depend on $\bOmega$ and $\bOmega'$ individually---as opposed to  $\bOmega\cdot 
\bOmega'$. By writing $s(\bOmega\cdot \bOmega')$, scattering from $\bOmega$ into 
$\bOmega'$ is as likely as scattering from $\bOmega'$ into $\bOmega$. 
Replacing the dependency on $\bOmega$ and $\bOmega'$ by the dependency on $\bOmega\cdot \bOmega'$---the cosine of the \textit{scattering angle}---is a significant mathematical simplification.
Lastly, since particles 
have to scatter into \textit{some} direction, 
\begin{align}
	\int_{\mathbb{S}^2} s(\bOmega\cdot \bOmega') \, d\bOmega' = 1
\end{align}
is required for any $\bOmega\in \mathbb{S}^2$.
\begin{lemma}[Range and kernel of $S$]

	Under the above assumptions, the range and kernel of the scattering operator $S$ are given 
	by
	\begin{itemize}
		\item $\mathcal{R}(S) =  \{\psi(t,\bx,\bOmega), \text{ such that } \int_{\mathbb{S}^2} 
			\psi(t,\bx,\bOmega) \, d\bOmega = 0\}$, and
		\item $\mathcal{K}(S) =  \{\psi(t,\bx,\bOmega), \text{ where } \psi(t,\bx,\bOmega)\text{  is 
			constant in } \bOmega\}$,
	\end{itemize}
	respectively. See, \eg the lecture notes by Frank \cite{kinetictheory201819} for a proof of both properties. 
\end{lemma}
Intuitively, this 
implies that:
\begin{itemize}
	\item The scattering operator is mass-preserving. Locally (at $t$ and $\bx$), particles are 
	only re-distributed (in $\bOmega$). Overall, no particles are gained or lost. 
	\item Unless $\psi$ is isotropic (independent of $\bOmega$), scattering will cause the 
	pre-collision state to be different from the post-collision state and, regardless of the 
	scattering kernel $s$, the equilibrium state of the scattering operator $S$ is  isotropic. This 
	does not imply the absence of scattering events; it merely implies a balance of in- 
	and out-scattering.
\end{itemize}


\subsection{Entropy for the linear transport equation}
\label{sec:entropy}
Let $\eta:\mathbb{R} \to \mathbb{R}$ be a differentiable, convex function, \ie $\forall x_1,x_2 \in \mathbb{R}$ and $\forall t \in [0,1]$, we have $\eta(t \, x_1 + (1-t) \, x_2) \leq t\, \eta(x_1) + (1-t) \, \eta(x_2)$. Since $\eta$ is convex, its derivative is monotonically non-decreasing.
Furthermore, recall that the scattering component of the transport equation \eqref{eq:lineartransport} was given by 
\begin{equation*}
	\begin{split}
		S(\psi)(t,\bx,\bOmega) \defas
		\int_{\mathbb{S}^2}s(\bOmega \cdot \bOmega')\left(\psi(t,\bx,\bOmega') - 
		\psi(t,\bx,\bOmega)\right)\, d\bOmega'.
	\end{split}
\end{equation*}
It follows that 
\begin{align}
	\int_{\mathbb{S}^2} \eta'\left(\psi(t,\bx,\bOmega)\right) S(\psi)(t,\bx,\bOmega) \, d\bOmega \leq 0,
\end{align}
because
	\begin{subequations}
		\begin{align}
	&\int_{\mathbb{S}^2} \eta'\left(\psi(t,\bx,\bOmega)\right) S(\psi)(t,\bx,\bOmega) \, d\bOmega \\
=&	\int_{\mathbb{S}^2} \int_{\mathbb{S}^2}\eta'\left(\psi(t,\bx,\bOmega)\right) s(\bOmega \cdot \bOmega')\left(\psi(t,\bx,\bOmega') - 
\psi(t,\bx,\bOmega)\right)\, d\bOmega'\, d\bOmega  \\
=&	\int_{\mathbb{S}^2}\int_{\mathbb{S}^2} \eta'\left(\psi(t,\bx,\bOmega')\right) s(\bOmega' \cdot \bOmega)\left(\psi(t,\bx,\bOmega) - 
\psi(t,\bx,\bOmega')\right)\, d\bOmega\, d\bOmega' \\
= & \frac{1}{2}\int_{\mathbb{S}^2}\int_{\mathbb{S}^2}
\underbrace{\left(\eta'\left(\psi(t,\bx,\bOmega)\right)- \eta'\left(\psi(t,\bx,\bOmega')\right)\right)}_{\defas c_1} \, 
\underbrace{s(\bOmega \cdot \bOmega') }_{\geq 0} \, 
\underbrace{\left(\psi(t,\bx,\bOmega') - 
\psi(t,\bx,\bOmega)\right)}_{\defas c_2}\, d\bOmega\, d\bOmega' \\
\leq & 0,
\end{align}
\end{subequations}
since the signs of $c_1$ and $c_2$ are opposite due to the convexity of $\eta$.
Additionally, if $\psi(t,\bx,\bOmega)\equiv 0$ for $||\bx||_2\rightarrow \infty$, the mathematical entropy 
\begin{align}
H(\psi)(t) \defas \int_{\mathbb{R}^3}\int_{\mathbb{S}^2} \eta\left(\psi(t,\bx,\bOmega)\right) \, d\bOmega \, d\bx	
\end{align}
is non-increasing.
The proof follows from the convexity of $\eta$ and the symmetry of the scattering kernel.

With a non-increasing entropy the process cannot be reversible. Although, on the microscopic level, particles return to their initial position when we continue the billiard game for an additional time $t$ with reversed velocities, this cannot be the case here. Flipping the velocities at time $t$ does not change the fact that the entropy will continue to decrease. Therefore, the original state of the system can never be recovered.

An explanation for this discrepancy is provided by Spohn \cite{spohn2012large}. Due to the assumptions that we make for the Boltzmann-Grad limit, certain states (situations where obstacles overlap or particles bounce forth and back indefinitely) are omitted from the description with the linear transport equation. As a result, transport described by the linear transport equation is not exactly equivalent to that described via the microscopic perspective.


\subsection{Initial and boundary conditions}
In general, (partial) differential equations describe \textit{how} a system evolves due to physical laws. Boundary and initial conditions describe \textit{which} system we are considering exactly. %define the system that is to be considered.
Thus, the linear transport equation \eqref{eq:lineartransport} needs to be equipped with suitable boundary and initial conditions.
\begin{figure}
	\centering
	\includegraphics[width=0.99\linewidth]{Chapters/Chapter1/figs/boundaryconditions/domain_with_boundary}
	\caption{Spatial domain $V\subsetneq\RThree$ with boundary $\partial V$ and outward 
	facing normal  $\bf{n}$.}
	\label{fig:domainwithboundary}
\end{figure}


For a bounded domain $V\subsetneq \RThree$ with boundary $\partial V$ and outward facing normal ${\bf{n}}$ as depicted in Figure \ref{fig:domainwithboundary}, we set the incoming flux
\begin{align}
	\psi(t,\bx,\bOmega) = \psi^{\text{bc}}(t,\bx,\bOmega)\quad\text{ for all t}\in \Rgeqz,\, \bx\in\partial V,\, \bOmega \in \mathbb{S}^2\text{ such that } \bOmega \cdot {\bf{n}} <0.
\end{align}
However, different ways to treat the boundary exist. We can as well have periodic boundaries 
or (specular) reflective boundaries, \ie particles that hit the boundary are being (isotropically) 
re-emitted into the domain.

For the initial condition, we have to prescribe the angular flux at $t=0$, \ie
\begin{align}
	\psi(0,\bx,\bOmega) = \psi_0(\bx,\bOmega)\quad \text{ for }\bx \in V, \bOmega \in \mathbb{S}^2.
\end{align}

Furthermore, we require the source $q(t,\bx,\bOmega)$ to be non-negative for all times $t \in \Rgeqz$.
Given these conditions, the uniqueness of the angular flux follows. %\todo{does it?}

%\begin{lemma}[Uniqueness for the linear transport equation]
%	see https://people.bath.ac.uk/man54/SAMBa/ITTs/ITT2/AMEC/PrinjaLarsen.pdf
%	only stationary?
%\end{lemma}
%\begin{proof}

%\end{proof}
%\todo{finish this}





\subsection{Stationary transport}
Frequently, the stationary state is of more interest than the actual evolution in time towards that state.
In the absence of time-dependent internal sources or boundary conditions, we can reduce \eqref{eq:lineartransport} to its time-independent counterpart
\begin{equation}
	\begin{split}
		& \bOmega \cdot \nabla_\bx \psi(\bx,\bOmega) +\sigma_a \psi(\bx,\bOmega) \\
		&= {\sigma_s}\int_{\mathbb{S}^2}s(\bOmega \cdot \bOmega')\left(\psi(\bx,\bOmega') - \psi(\bx,\bOmega)\right)\, d\bOmega'+ q(\bx,\bOmega).
		\label{eq:lineartransport_timeindependent}
	\end{split}
\end{equation}
Especially when applied to nuclear engineering related problems, the resolution of the temporal scale is often not necessary. Equipped with a positive source $q(\bx,\bOmega)$ and positive boundary conditions $\psi^{\text{bc}}(\bx,\bOmega)$, there exists a non-negative solution to \eqref{eq:lineartransport_timeindependent}. In Section \ref{section:neutrontransport} we will show that this steady state solution exists only for critical systems (and not for super- or subcritical ones) when introducing more physics into the equations.
\subsection{Slab geometry}
A common simplification of the spatially two- or three-dimensional transport equation is slab geometry. Consider Figure \ref{fig:slab}, depicting the situation for the two-dimensional case. When exclusively interested in the transport behavior along a single axis (the $x$-axis in this case), we can project onto that axis and reduce the dimension to one. Consequently, the solution is the same for every horizontal cut through the slab. The transport equation for slab geometry is given by
\begin{equation}
	\begin{split}
		\partial_t \psi(t,x,\mu) +& \mu \,\partial_x\psi(t,x,\mu) +\sigma_a \psi(t,x,\mu) \\
					  &= {\sigma_s}\int_{-1}^1s(\mu,\mu')\left(\psi(t,x,\mu') - \psi(t,x,\mu)\right)\, d\mu'+ q(t,x,\mu),
					  \label{eq:lineartransport_slab}
	\end{split}
\end{equation}
with $\mu = \bOmega \cdot (1,0,0)^T$ .%\todo{is this correct?}.
\begin{figure}
	\centering
	\includegraphics[width=1\linewidth]{Chapters/Chapter1/figs/slab/slab.pdf}
	\caption{Slab geometry for two-dimensional transport. Transport is projected onto the $x$-axis.}
	\label{fig:slab}
\end{figure}



%	
%	\section{Boltzmann equation}
%	The transport equations so far were linear as they describe the interaction of particles with the background medium or with forces. The Boltzmann equation, however, is non-linear as it describes the interaction \textit{between} particles. In the absence of an external force field and sources, we write the Boltzmann equation as
%	\begin{equation}
%		\begin{split}
%			\partial_t \psi(t,\bx,\bv) + \bv \cdot \nabla_\bx \psi(t,\bx,\bv)  = 
%			J(\psi,\psi) = J^+(\psi,\psi)-J^-(\psi,\psi).
%			\label{eq:boltzmanneq}
%		\end{split}
%	\end{equation}
%	The non-linearity enters the equation via the gain and loss terms, denoted by $J^+$ and $J^-$. They are given by
%	\begin{align}
%		J^+(\psi,\psi)(t,\bx,\bv) = \int_{\RThree\times \mathbb{S}^2_+} B(\mathbf{n},\mathbf{q})
%		\psi(t,\bx,\bv')\psi(t,\bx,\bw')\, d\mathbf{n}\, d\bw
%	\end{align}
%	for the gain term, and
%	\begin{align}
%		J^-(\psi,\psi)(t,\bx,\bv) =  \int_{\RThree\times \mathbb{S}^2_+} B(\mathbf{n},\mathbf{q})
%		\psi(t,\bx,\bv)\psi(t,\bx,\bw)\, d\mathbf{n}\, d\bw
%	\end{align}
%	for the loss term, respectively. The pre-collision velocities $\bv$ and $\bw$ are related to the post-collision velocities $\bv'$ and $\bw'$. More precisely,
%	\begin{align}
%		\begin{pmatrix}
%			\bv' \\ \bw'
%		\end{pmatrix}
%		=
%		T_\mathbf{n}
%		\begin{pmatrix}
%			\bv \\ \bw
%		\end{pmatrix}
%		=
%		\begin{pmatrix}
%			\bv-\mathbf{n}\,\langle \mathbf{n}, \bv-\bw\rangle \\
%			\bw+\mathbf{n}\,\langle \mathbf{n}, \bv-\bw\rangle
%		\end{pmatrix},
%	\end{align}
%	where $\langle \cdot ,\cdot \rangle$ denotes the standard inner product. With $\mathbf{q}=\bw-\bv$, we integrate $\mathbf{n}$ over $\mathbb{S}^2_+$, defined by
%	\begin{align}
%		\mathbb{S}^2_+ \defas \{\mathbf{n}\in \mathbb{S}^2 \text{ such that } \langle \mathbf{n},\mathbf{q}\rangle\geq 0\}.
%	\end{align}
%	The non-negative kernel $B$ models the way  particles interact with each other. For example, hard sphere collisions can be modeled with $B(\mathbf{n},\mathbf{q}) = |\mathbf{q}|$.
%	
%	The Boltzmann equation in this form requires particle interactions to be both binary and instantaneous.

%\subsection{Entropy for the Boltzmann equation}
%\todo{missing}


\section{Neutron transport}
\label{section:neutrontransport}
After the Boltzmann equation had established its track record as a useful tool in the study of the theory of gases, neutron transport became a highly researched topic with the development of nuclear reactors and the Manhattan project in the 1940s. Different from the scenarios considered so far, the phase space is augmented by an additional energy dependency. Consequently, particles---or the respective angular flux---are no longer solely defined by their position in space $\bx$ and their direction of flight $\bOmega$ at a given time $t$, but also by their energy $E$, defined by $E=\frac{1}{2} m v^2$. Here, $m$ is the particle's mass and $v$ its speed.
Additionally, scattering and absorption are not the only possibilities when colliding: A fission event might be initiated by the interaction of the nuclei.

The (isotropic) energy-dependent cross sections then satisfy
\begin{align}
	\sigma_a(E) + \sigma_s(E) + \sigma_f(E) = \sigma_t(E),
\end{align}
where $\sigma_f(E)$ is the fission cross section. The scattering kernel is interpreted as
\begin{equation}
	\begin{split}
			s(\bOmega\cdot \bOmega', E\rightarrow E')\,  d\bOmega' \, dE' \defas
		& \text{  ``the probability that a neutron which} \\
		&\text{ \, scatters and has pre-collision direction } \bOmega\\
		&\text{ \, and energy } E \text{, has a post-collisions direction}\\
		&\text{ \, in }d\bOmega' \text{ about } \bOmega' \text{ and energy in } dE' \text{ about } E'\text{.}"
	\end{split}
\end{equation}
%A further simplification allows to express $	s(\bOmega\cdot \bOmega', E\rightarrow E')$ in terms of two separate distributions, \ie
%\begin{align}
%	s(\bOmega\cdot \bOmega', E\rightarrow E') = s(\bOmega\cdot \bOmega') \cdot s( E\rightarrow E').
%\end{align}
In the case of a fission event, the nucleus splits into two nuclei. On average, this produces $\nu(E)$ new neutrons that can be divided into \textit{prompt neutrons} and \textit{delayed neutrons}, occurring with probability $1-\beta(E)$ and $\beta(E)$, respectively \cite{prinja2010general}. Here, $\beta(E)$ denotes the \textit{delayed neutron fraction} and is usually small ($\approx 0.001$).
When we assume (i) fission neutrons to be born with uniform direction
and (ii) energy distributed according to their prompt fission spectrum $\chi_p(E)$,
we can write the respective gain term (gain of $\psi(t,\bx,\bOmega,E)$) as
\begin{align}
	\chi_p(E) \int_0^\infty \int_{\mathbb{S}^2} (1-\beta(E'))\sigma_f(E')\nu(E')\psi(t,\bx,\bOmega',E')
	\, d\bOmega'\, dE'.
\end{align}
%The contribution of delayed neutrons can be included into the balance law by an isotropic source
%\begin{align}
%	\frac{1}{4\pi}\sum_{j=1}^6 \chi_j(E)\,  \lambda_j \, C_j(t,\bx).
%\end{align}
%Unstable nuclei were grouped into six precursor groups. These precursor densities encode the emittance of group-j nuclei \cite{prinja2010general}.
In the absence of delayed neutrons ($\beta$ being small), the energy-dependent transport equation that includes fission reads
\begin{equation}
	\begin{split}
		\frac{1}{v}\partial_t &\psi(t,\bx,\bOmega,E) + \bOmega \cdot \nabla_\bx \psi(t,\bx,\bOmega,E) +\sigma_a(E) \psi(t,\bx,\bOmega,E) \\
		=&\int_{\mathbb{S}^2}\int_0^\infty  {\sigma_s}(E')s(\bOmega \cdot \bOmega',E'\rightarrow E)\left(\psi(t,\bx,\bOmega',E') - \psi(t,\bx,\bOmega,E)\right)\, dE' \, d\bOmega'\\
		 &+
		 \chi_p(E)  \int_{\mathbb{S}^2}\int_0^\infty (1-\beta(E'))\sigma_f(E')\nu(E')\psi(t,\bx,\bOmega',E')
		\, dE' \,  d\bOmega'\\&+
		q(t,\bx,\bOmega,E),
		\label{eq:lineartransport_withenergy_withfission}
	\end{split}
\end{equation}
with the neutron speed $v=\sqrt{2E/m}$. Dropping the time derivative in \eqref{eq:lineartransport_withenergy_withfission} results in the time-independent, energy-dependent neutron transport equation
\begin{equation}
	\begin{split}
		\bOmega \, \cdot& \nabla_\bx \psi(\bx,\bOmega,E) +\sigma_a(E) \psi(\bx,\bOmega,E) \\
		=&\int_{\mathbb{S}^2}\int_0^\infty  {\sigma_s}(E')s(\bOmega \cdot \bOmega',E'\rightarrow E)\left(\psi(\bx,\bOmega',E') - \psi(t,\bx,\bOmega,E)\right)\, dE' \, d\bOmega'\\
		 &+
		 \chi_p(E)  \int_{\mathbb{S}^2}\int_0^\infty (1-\beta(E'))\sigma_f(E')\nu(E')\psi(\bx,\bOmega',E')
		\, dE' \,  d\bOmega'\\&+
		q(\bx,\bOmega,E).
		\label{eq:lineartransport_withenergy_withfission_steadystate}
	\end{split}
\end{equation}

\subsection{Criticality calculations}
In the absence of sources and with homogeneous boundary conditions the fundamental behavior of the time-independent neutron transport equation is determined by its \textit{criticality}. Here, criticality refers to the largest eigenvalue $k$ of
\begin{equation}
	\begin{split}
		\bOmega \cdot& \nabla_\bx \psi(\bx,\bOmega,E) +\sigma_a(E) \psi(\bx,\bOmega,E) \\
		=&\int_{\mathbb{S}^2}\int_0^\infty  {\sigma_s}(E')s(\bOmega \cdot \bOmega',E'\rightarrow E)\left(\psi(\bx,\bOmega',E') - \psi(\bx,\bOmega,E)\right)\, dE' \, d\bOmega'\\
		 &+
		 \frac{1}{k}\, \chi_p(E)  \int_{\mathbb{S}^2}\int_0^\infty \sigma_f(E')\nu(E')\psi(\bx,\bOmega',E')
		 \, dE' \,  d\bOmega',
		 \label{eq:lineartransport_withenergy_withfission_steadystate_keigenvalue}
	\end{split}
\end{equation}
with vacuum boundary conditions $\psi(\bx,\bOmega,E) = 0$ for $\bx \in \partial V$, $\bOmega \cdot \bn(\bx)  < 0$, and all energies $E$.
We assume that it is possible to control the expected number of fission neutrons that are generated and hence substituted $\nu(E')$ by $\nu(E')/k$ in the time-independent energy-dependent neutron transport equation. Clearly, $\psi\equiv 0 $ is a trivial solution to \eqref{eq:lineartransport_withenergy_withfission_steadystate_keigenvalue}.
However, for non-trivial solutions the system can be in one of the following three states, depending on the largest eigenvalue $k$:
\begin{enumerate}
	\item The system is subcritical for $k<1$. Fission does not generate enough particles and absorption is dominant.
	\item The system is critical for $k=1$. Fission generates particles at the exact same rate that particles are being absorbed or are leaking out of the system. A non-trivial solution to \eqref{eq:lineartransport_withenergy_withfission_steadystate_keigenvalue} exists.
	\item The system is supercritical for $k>1$. Neutrons are being generated at a higher rate than being lost.
\end{enumerate}
This so-called $k$ eigenvalue problem is of significant importance in the field of steady-state reactor physics. There,  the goal is to operate a reactor at a critical state, controlling the rate of generated fission neutrons such that the reactor neither becomes supercritical (too many neutrons being generated), nor subcritical (too few neutrons being generated).


\section{Asymptotic limits}
Let us, once again, consider the velocity-dependent, linear transport equation, expressed in 
terms of 
the 
total  cross section $\sigma_t$  and the probability that a colliding particle will scatter $c$, given by
\begin{equation*}
	\begin{split}
		\frac{1}{v}\, \partial_t \psi(t,\bx,\bOmega) +& \,\bOmega \cdot \nabla_\bx \psi(t,\bx,\bOmega) 
		+\sigma_t 
		\psi(t,\bx,\bOmega) \\
							   &= {c\, \sigma_t}\int_{\mathbb{S}^2}s(\bOmega \cdot \bOmega')\psi(t,\bx,\bOmega') \, 
							   d\bOmega'+ q(t,\bx,\bOmega).
							   \label{eq:lineartransport_again}
	\end{split}
\end{equation*}
To obtain a dimensionless equation, we set
\begin{subequations}
	\label{eq:dimensionless}%
	\begin{eqnarray}
		t &= \bar{t} \, T, \\
		\bx &= \bar{{\bx}}\, L,\\
		v &= \bar{v} \, V, \\
		\sigma_t &= \bar{\sigma}_t \, \Sigma_t,
	\end{eqnarray}

\end{subequations}
where variables with a superscript bar denote dimensionless variables and capitalized variables 
denote the reference time, length, velocity, and cross section, respectively. We further assume 
the scattering kernel to already be dimensionless. 
Substituting \eqref{eq:dimensionless} in the linear transport equation and omitting the source 
yields
\begin{equation}
	\begin{split}
		\frac{1}{\bar{v}V}\partial_{\bar{t} \, T} &{\psi}(\bar{t} \, T,\bar{{\bx}}\, L,\bOmega) + 
		\,\bOmega\cdot 
		\nabla_{\bar{{\bx}}\, L} 
		{\psi}(\bar{t} \, T,\bar{{\bx}}\, L,\bOmega) 
		+ \bar{\sigma}_t \, \Sigma_t 
		{\psi}(\bar{t} \, T,\bar{{\bx}}\, L,\bOmega) \\
							  &= c\, { \bar{\sigma}_t \, \Sigma_t}\int_{\mathbb{S}^2}s(\bOmega \cdot 
							  \bOmega' ){\psi}(\bar{t} \, 
							  T,\bar{{\bx}}\, L,\bOmega) \, d\bOmega.
							  \label{eq:lineartransport_again_dimless}
	\end{split}
\end{equation}
We define\footnote{It is the velocity $v$, not the 
	direction $\bOmega$ that has to be scaled since $\bOmega$ is, by definition, a unit-vector.} 
	$\psi(\bar{t} \, T,\bar{{\bx}}\, L,\bOmega)
= \bar{\psi} (\bar{t} ,\bar{{\bx}},\bOmega)\Psi$ 
and 
directly omit the superscript bar 
everywhere to 
get
\begin{equation}
	\begin{split}
		\frac{\Psi}{T\,V}\,\frac{1}{\bar{v}} \,\partial_t \psi(t,\bx,\bOmega) +& \,\frac{\Psi }{L}\bOmega 
		\cdot 
		\nabla_\bx 
		\psi(t,\bx,\bOmega) 
		+ \Psi\, \Sigma_t \,\sigma_t 
		\psi(t,\bx,\bOmega) \\
										       &= c\,\Psi \, \Sigma_t \, {\sigma_t}\int_{\mathbb{S}^2}s(\bOmega \cdot 
										       \bOmega')
										       \psi(t,\bx,\bOmega') \, d\bOmega',
										       \label{eq:lineartransport_again_dimless2a}
	\end{split}
\end{equation}
or equivalently\begin{equation}
	\begin{split}
		\frac{L}{V\, T} \,\frac{1}{{v}}\,\partial_t \psi(t,\bx,\bOmega) +& \,\bOmega \cdot 
		\nabla_\bx 
		\psi(t,\bx,\bOmega) 
		+ \Sigma_t\, L \,\sigma_t 
		\psi(t,\bx,\bOmega) \\
										 &= c \, \Sigma_t \,L\, {\sigma_t}\int_{\mathbb{S}^2}s(\bOmega \cdot 
										 \bOmega')\psi(t,\bx,\bOmega') \, d\bOmega'.
										 \label{eq:lineartransport_again_dimless2}
	\end{split}
\end{equation}
Equation \eqref{eq:lineartransport_again_dimless2} can be rewritten in terms of the Strouhal 
number (St $\defas L/(V\,  T)$) and the Knudsen number (Kn $\defas 1/(\Sigma_t \, 
L)$). We additionally set $v=1$ since the velocity can be controlled by $V$, and $v$ does not 
occur in the equation without being multiplied by $V$. The transport 
equation in dimensionless form then reads 
\begin{equation}
	\begin{split}
		\text{St} \,\partial_t \psi(t,\bx,\bOmega) +& \,\bOmega \cdot 
		\nabla_\bx 
		\psi(t,\bx,\bOmega) 
		+ \frac{1}{{\text{Kn}}} \sigma_t \,
		\psi(t,\bx,\bOmega) \\
							    &= \frac{c}{{\text{Kn}}}  {\sigma_t}\int_{\mathbb{S}^2}s(\bOmega \cdot 
							    \bOmega')\psi(t,\bx,\bOmega') \, d\bOmega'.
							    \label{eq:lineartransport_again_dimless_final}
	\end{split}
\end{equation}
The inverse Knudsen number $\Sigma_t \, 
L$ is a measure for 
thickness of the system in units of mean free paths. 
The Strouhal number is
 interpreted as the ratio between the time needed to cross the domain by
the characteristic velocity and the characteristic time scale \cite{kinetictheory201819}.

Under different assumptions on the order of the Knudsen and Strouhal number, we are now 
able to obtain different asymptotic limits to the standard transport equation.
Due to its importance in theoretical considerations, as well as its implications for numerical 
methods, we will discuss the diffusive scaling in more 
detail. 
Diffusive scaling uses the following assumptions: 
\begin{enumerate}
	\item The mean free path is much smaller than the reference length ($\text{Kn} = 
		\varepsilon$).
	\item Interactions happen on a much smaller time-scale than the reference time 
		($\text{St}=\varepsilon$).
	\item The likelihood to scatter is much, much larger than the likelihood to be absorbed in 
		case of a collision
		($c=1-\varepsilon^2$).
\end{enumerate}
First we rewrite \eqref{eq:lineartransport_again_dimless_final} by moving out-scattering back 
on 
the right-hand side.
\begin{equation}
	\begin{split}
		\text{St} \,\partial_t \psi(t,\bx,\bOmega) +& \,\bOmega \cdot 
		\nabla_\bx 
		\psi(t,\bx,\bOmega) 
		+ \frac{1-c}{{\text{Kn}}} \sigma_t \,
		\psi(t,\bx,\bOmega) \\
							    &= \frac{c}{{\text{Kn}}}  {\sigma_t}\int_{\mathbb{S}^2}s(\bOmega \cdot 
							    \bOmega')\left(\psi(t,\bx,\bOmega') -\psi(t,\bx,\bOmega)\right)\, d\bOmega'.
							    \label{eq:lineartransport_again_dimless_final_sigmas_sigmaa}
	\end{split}
\end{equation}
We now substitute $\text{Kn}=\varepsilon$ and $\text{St}=\varepsilon$. Next, $(1-c)\sigma_t$ and $c 
\sigma_t$ can be expressed in terms of $\varepsilon$ and dimensionless quantities as 
$\varepsilon^2\sigma_a$ and $\sigma_s$, respectively. 
A final multiplication with $\varepsilon$ results in 

\begin{equation}
	\begin{split}
		\varepsilon^2\,\partial_t \psi_{(\varepsilon)}(t,\bx,\bOmega) +& \,\varepsilon \,\bOmega \cdot 
		\nabla_\bx 
		\psi_{(\varepsilon)}(t,\bx,\bOmega) 
		+ \varepsilon^2\sigma_a \,
		\psi_{(\varepsilon)}(t,\bx,\bOmega) \\
							       &=  {\sigma_s}\int_{\mathbb{S}^2}s(\bOmega \cdot 
							       \bOmega')\left(\psi_{(\varepsilon)}(t,\bx,\bOmega') -\psi_{(\varepsilon)}(t,\bx,\bOmega)\right)\, d\bOmega'.
							       \label{eq:lineartransport_again_dimless_final_sigmas_sigmaa_eps}
	\end{split}
\end{equation}
The diffusion limit lets $\varepsilon\rightarrow 0$. To compute $\psi_{(\varepsilon)}$ for 
$\varepsilon\rightarrow 0$, we assume the existence of a Hilbert expansion for 
$\psi_{\varepsilon}$, that is
\begin{align}
	\psi_{(\varepsilon)}(t,\bx,\bOmega) = \sum_{n=0}^\infty \varepsilon^n \, \psi_i(t,\bx,\bOmega).
	\label{eq:hilbertexpansion}
\end{align}
The $\psi_i$ in the Hilbert expansion are independent of $\varepsilon$. We substitute 
\eqref{eq:hilbertexpansion} into 
\eqref{eq:lineartransport_again_dimless_final_sigmas_sigmaa_eps} and collect matching 
orders of $\varepsilon$ (up to second order) to get 
\begin{subequations}
	\label{eq:matchingepsilons}%
	\begin{eqnarray}
		\mathcal{O}\left(\varepsilon^0\right):& 
		0&=\sigma_s S(\psi_0), 	\label{eq:orders0}
		\\
		\mathcal{O}\left(\varepsilon^1\right):& 
		\bOmega \cdot 
		\nabla_\bx 
		\psi_0 &=
		\sigma_s	S(\psi_1), 	\label{eq:orders1}
		\\
		\mathcal{O}\left(\varepsilon^2\right):& 
		\partial_t \psi_0+\bOmega \cdot 
		\nabla_\bx 
		\psi_1 + \sigma_a \psi_0&=
		\sigma_s	S(\psi_2), 	\label{eq:orders2}
	\end{eqnarray}
\end{subequations}
where we omit the dependency on $t$, $\bx$, and $\bOmega$ in the $\psi_i$ and the collision 
operator $S(\psi_i)\, $.
Section \ref{subsec:propertiescollisionoperator} tells us that the kernel of $S$ consists of 
functions that are  independent of $\bOmega$; this implies $\psi_0(t,\bx,\bOmega) = \psi_0(t,\bx)$.

Since $S$ is linear, we can rewrite \eqref{eq:orders1} as 
\begin{align}
	\psi_1(t,\bx,\bOmega) =  -\frac{1}{\sigma_s} \sum_{i=1}^3 e_i(\bOmega) \partial_{x_i} \psi_0(t,\bx),
\end{align}
for some $e_i \in\mathcal{R}(S)$ with $S(e_i) = - \bOmega_i$ for $i=1,2,3$ \cite{kinetictheory201819}.
Inserting this into \eqref{eq:orders2}, we get
\begin{align}
	\partial_t \psi_0(t,\bx)-\bOmega \cdot 
	\nabla_\bx 
	 \frac{1}{\sigma_s} \sum_{i=1}^3 e_i(\bOmega) \partial_{x_i} \psi_0(t,\bx) + \sigma_a \psi_0(t,\bx)&=
	\sigma_s	S(\psi_2). \label{eq:asdasdasd}
\end{align}
For the left-hand side to be in the range of $S$, $\psi_0$ has to be chosen such that 
\begin{align}
	\int_{\mathcal{S}^2} 1 \cdot \left(\partial_t \psi_0(t,\bx)-\bOmega \cdot 
	\nabla_\bx 
	\frac{1}{\sigma_s} \sum_{i=1}^3 e_i(\bOmega) \partial_{x_i} \psi_0(t,\bx) + \sigma_a \psi_0(t,\bx) \right)\, d\bOmega&=0.
\end{align}
Changing the order of differentiation and integration in the second term, we obtain 
\begin{align}
	\partial_t \psi_0(t,\bx)- 
	\nabla_\bx \, \frac{1}{\sigma_s} \, A \,\nabla_\bx \psi_0(t,\bx) + \sigma_a \psi_0(t,\bx)  = 0,
	\label{eq:diff1}
\end{align}
with a positive definite matrix $A \in \mathbb{R}^{3\times 3}$ that has entries $A_{i,j} = \int_{\mathbb{S}^2} \bOmega_i \, e_j(\bOmega) \, d\bOmega$. 

We assumed constant cross sections independent of $\bx$, but \eqref{eq:diff1} 
can analogously be written  for spatially varying cross sections.


\section{Hyperbolic conservation laws}
Conservation laws are building blocks for the mathematical description of physical reality. 
They describe the evolution of one or several conserved quantities through space-time; 
common conserved quantities are mass, momentum, or energy. We adapt the notation of the 
classical book by LeVeque \cite{leveque1992numerical}.
A scalar hyperbolic conservation law has the form
\begin{align}
\partial_t {u}(t,x) +\partial_x f\left(u(t,x)\right) = 0.
\end{align}
The conserved quantity is ${u}: \Rgeqz \times V \rightarrow 
\mathbb{R}$. Time is denoted by $t \in \Rgeqz$ and space by $x \in 
V \subseteq
\mathbb{R}$. 
For a spatially infinite domain, conservation means that $\int_{-\infty}^\infty  u(t,x) \, dx$ is 
constant in time. The flux function of 
the system is given by $f: \mathbb{R} \rightarrow \mathbb{R}$ and can be 
interpreted as the flux of a conserved quantity across a surface. For $n$ conserved quantities 
$\bm{u}(t,x) = (u_1(t,x),\ldots,u_n(t,x))^T$ in one space dimension with fluxes 
$\bm{f}(\bm{u}(t,x)) = 
(f_1(\bm{u}(t,x)),\ldots,f_n(\bm{u}(t,x)))^T$,
\begin{align}
\partial_t \bm{u}(t,x) + \partial_x \bm{f}(\bm{u}(t,x)) = 0
\label{eq:systemhyperbolicconservationlaws}
\end{align}
is the corresponding system of $n$ conservation laws. The system is said to be 
hyperbolic if 
\begin{align}
 A(\bm{u}(t,x)) \defas \bm{f}'(\bm{u}(t,x))
\end{align}
is diagonalizable with $n$ real eigenvalues for all $\bm{u}(t,x)$. If the $n$ real eigenvalues 
are also distinct, the system is \textit{strictly }hyperbolic.
The quasilinear form of \eqref{eq:systemhyperbolicconservationlaws}  is 
\begin{align}
	\partial_t \bm{u}(t,x) + A(\bm{u}(t,x)) \partial_x\bm{u}(t,x) = 0.
\end{align}
For problems with two spatial dimension and $\bm{x}=(x,y)\in\mathbb{R}^2$, 
we write
\begin{align}
\partial_t \bm{u}(t,\bx) + \partial_x \bm{f}(\bm{u}(t,\bx)) + \partial_y \bm{g}(\bm{u}(t,\bx)) = 0,
\label{eq:systemhyperbolicconservationlaws_2d}
\end{align}
with $\bm{f}(\bm{u}(t,\bx)) = 
(f_1(\bm{u}(t,\bx)),\ldots,f_n(\bm{u}(t,\bx)))^T$ and $\bm{g}(\bm{u}(t,\bx))$ defined 
analogously. 
\FloatBarrier 
\begin{exmp}[Linear advection equation]
The simplest hyperbolic conservation law is  the one-dimensional linear advection equation 
with constant speed $a$,
\begin{align}
	\partial_t \rho(t,x) + a \, \partial_x \rho(t,x)=0,
	\label{eq:linearadvection}
\end{align}
equipped with initial condition $\rho(0,x) = \rho_0(x)$ for $t \in \Rgeqz$ and $x\in\R$.
The solution of \eqref{eq:linearadvection} is given by
$\rho(t,x) = \rho_0(x-at)$ (that is, the initial mass is just shifted under consideration of $a$), shown in Figure \ref{fig:linear}.
\begin{figure}[]
	\centering
	\includegraphics[width=1.0\linewidth]{Chapters/Chapter1/figs/probspeeds/kitlinear.pdf}	
	\caption{Linear advection equation with $\rho_0(x) = \exp(-x^2)$.}
	\label{fig:linear}
\end{figure}



\end{exmp}

\FloatBarrier
\begin{exmp}[Inviscid Burgers's equation]
	Let us now consider the inviscid Burgers's equation as a toy model for 
	non-linear hyperbolic conservation laws. The equation is given by
	\begin{align}
		\partial_t u(t,x) + u\, \partial_xu(t,x) = 0,
	\end{align}
		subject to initial data $u_0(x)= u(0,x)$.
	Figure \ref{fig:burgers} illustrates the evolution of the initial condition
	\begin{align}
	u_0(x) = \begin{cases}
	2,&x<0, \\
	2-x/2, &0\leq x< 2,\\
	1,& x\geq 2 ,
	\end{cases}
	\end{align}
	drawn in green. Since  the propagation speeds of the solution differ, a jump in the 
	solution---called a shock---forms in finite time. 
	Shocks are a ubiquitous phenomena in physical applications; exemplified by the formation 
	of traffic jams in traffic flow simulations or pressure jumps in the flow field around airfoils.
 
	\begin{figure}[h!]
		\centering
		\includegraphics[width=1.0\linewidth]{Chapters/Chapter1/figs/probspeeds/kitburgers.pdf}
		\caption{Burgers's equation with initial data $u_0(x)$ shown in green. 
		}
		\label{fig:burgers}
	\end{figure}
\end{exmp}

Since the analytical solution to non-linear hyperbolic conservation laws is known for 
exceptionally few---and usually exceptionally simple---test cases, numerical methods need to be 
considered. Significant contributions in the field of numerical methods for these types of 
equations date back to the 1950s and 1960s with work by Sergei Konstantinovich Godunov, 
Peter Lax, Kurt Otto Friedrichs, and Burton Wendroff.\footnote{This is by no means a 
	complete list of all contributions. The names were 
	selected since there exist famous numerical schemes named after all of these researchers.}
%\todo{Check date range and names}
%\todo{cite:  ref for burgers and analytical solution, see wiki}


	


\begin{exmp}[Radiative transfer with absent scattering in two space dimensions]
	
	\begin{figure}[t!]
		\centering
		\includegraphics[width=1.0\linewidth]{Chapters/Chapter1/figs/probspeed2d/1}
		\caption{Evolution of an initial distribution $\psi_0(\bx,\bOmega)$ through space along a 
			fixed  direction $\bOmega$ 
			with constant $\sigma_a$. Each \textit{bump} represents the solution at a different time step.}
		\label{fig:2dadvection}
	\end{figure}
	
Recall that the linear transport equations without scattering or sources can be written as
\begin{equation*}
\begin{split}
\partial_t \psi(t,\bx,\bOmega) +& \bOmega \cdot \nabla_\bx \psi(t,\bx,\bOmega) +\sigma_a 
\psi(t,\bx,\bOmega) =0.
\end{split}
\end{equation*}
With scattering being absent, the equation can easily be solved for any fixed direction 
$\bOmega$. Choosing  $\psi(0,\bx,\bOmega) =  \psi_0(\bx,\bOmega)$, the solution becomes
\begin{equation*}
\begin{split}
\psi(t,\bx,\bOmega)  =   \psi_0(\bx-t\bOmega,\bOmega) e^{-\sigma_a t}.
\end{split}
\end{equation*}
Thus, the initial condition is propagated along direction $\bOmega$ with absorption taking 
place at the same time. When considering scattering, the solution cannot be derived 
analytically, instead numerical algorithms need to be used. Note that without 
scattering, the linear transport equation resembles the structure of the simple linear 
advection equation. This is illustrated in Figure \ref{fig:2dadvection}, where we can see the 
solution $\psi(t,\bx,\bOmega)$ to the linear transport equation for different times $t$ with a 
fixed direction $\bOmega$. As initial condition, we chose $\psi_0(\bx,\bOmega) = 
\exp(-||\bx||_2^2)$, the fixed direction is $\bOmega = (1/\sqrt{2},1/\sqrt{2})^T$, and $\sigma_a 
= 
0.1$.


	
	
\end{exmp}




%\section{Applications}
%Equipped with the necessary mathematical and physical background, we will postpone the 
%discussion of how to solve the aforementioned equations for a moment. Instead, let us first 
%focus on relevant applications.
%\todo{Find something that I can actually write about without sounding like an idiot}
%\subsection{Nuclear engineering}
%ubiquitous in nuclear engineering 
%pervasive %
%here dive into diffusion too




%\subsection{Simulation of plasma}
%here talk about coupling
%renewables


%\subsection{Computer graphics}
%here talk about monte carlo





