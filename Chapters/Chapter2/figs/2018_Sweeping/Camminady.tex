\documentclass[12pt]{article}
\usepackage{rotating}
\usepackage[export]{adjustbox}
\usepackage{bm}
\usepackage{empheq}
\usepackage[most]{tcolorbox}
\usepackage{caption}
\usepackage{graphicx}
\usepackage{subfig}
\usepackage{wasysym}
\usepackage[margin=1in]{geometry}
\usepackage{amsmath,amsthm,amssymb}
\usepackage{hyperref}
\usepackage{algorithm2e}
\usepackage{algorithmic}
\usepackage{float}
\usepackage{multimedia}
\usepackage{tikz}
\usepackage{atbegshi,picture}



\newenvironment{theorem}[2][Theorem]{
  \begin{trivlist}
  \item[\hskip \labelsep {\bfseries #1}\hskip \labelsep {\bfseries
  #2.}]}{
\end{trivlist}}
\newenvironment{lemma}[2][Lemma]{
  \begin{trivlist}
  \item[\hskip \labelsep {\bfseries #1}\hskip \labelsep {\bfseries
  #2.}]}{
\end{trivlist}}
\newenvironment{exercise}[2][Exercise]{
  \begin{trivlist}
  \item[\hskip \labelsep {\bfseries #1}\hskip \labelsep {\bfseries
  #2.}]}{
\end{trivlist}}
\newenvironment{reflection}[2][Reflection]{
  \begin{trivlist}
  \item[\hskip \labelsep {\bfseries #1}\hskip \labelsep {\bfseries
  #2.}]}{
\end{trivlist}}
\newenvironment{proposition}[2][Proposition]{
  \begin{trivlist}
  \item[\hskip \labelsep {\bfseries #1}\hskip \labelsep {\bfseries
  #2.}]}{
\end{trivlist}}
\newenvironment{corollary}[2][Corollary]{
  \begin{trivlist}
  \item[\hskip \labelsep {\bfseries #1}\hskip \labelsep {\bfseries
  #2.}]}{
\end{trivlist}}



\title{A short proof that sweeping is always
  possible for a spatial discretization with regular triangles and no hanging
nodes}
\author{Thomas Camminady\thanks{Corresponding author:
  \texttt{camminady@kit.edu}},
  Martin Frank\\
Karlsruhe Institute of Technology}
\begin{document}


\maketitle

\abstract{Sweeping is a commonly used procedure to explicitly solve the
  discrete ordinates equation, which itself is a common approximation of the
  neutron transport equation. To sweep through the computational domain, an
  ordering of the spatial cells is required that obeys the flow of information.
  We show that this ordering can always be found, assuming a discretization of
  the spatial domain with regular triangles with no hanging nodes.
}



\section{Introduction}
Iterative methods for the numerical solution of transport processes often
make use of sweeping to invert the streaming operator \cite{adams2002fast}.
Sweeping requires to march through all spatial cells in a way that obeys the
flow of information, prescribed by a direction $\Omega$.
By that, a dependency between cells is induced which implies a dependency
graph $G$.
Here, two nodes $i'$ and $i$ share a directed edge, pointing from
$i'$ to $i$, if spatial cell $C_i$ does depend on spatial cell $C_{i'}$.
To be more precise: $C_i$ depends on $C_{i'}$, if and only if they share
an edge $e_{i',i}$ and the
inward pointing normal $n_i$ (inward pointing with respect to cell $C_i$)
onto edge $e_{i',i}$ satisfies $\langle n_i,\Omega\rangle >0$. Fig.
\ref{fig:smallmeshlabels} shows a triangulation with labeled cells. For two
different directions $\Omega$, Fig. \ref{fig:directegraph1} and
\ref{fig:directegraph2} show the induced dependency for $\Omega$ downward
pointing and $\Omega$ upward pointing, respectively.
\begin{figure}
  \centering
  \begin{minipage}{0.3\textwidth}
    \includegraphics[width=0.9\linewidth]{smallmesh_labels}
    \caption{Triangulation of the computational domain with labels.}
    \label{fig:smallmeshlabels}
  \end{minipage}
  \hspace{0.5cm}
  \begin{minipage}{0.3\textwidth}
    \includegraphics[width=0.9\linewidth]{directegraph1}
    \caption{Induced dependency graph for \\$\Omega=(0,-1)^T$.}
    \label{fig:directegraph1}
  \end{minipage}
  \hspace{0.5cm}
  \begin{minipage}{0.3\textwidth}
    \includegraphics[width=0.9\linewidth]{directegraph2}
    \caption{Induced dependency graph for \\$\Omega=(0,+1)^T$.}
    \label{fig:directegraph2}
  \end{minipage}
\end{figure}
The inward and
outward pointing normal vectors are sketched for a cell $C_i$ in Fig.
\ref{fig:cellci}.

Every directed acyclic graph has a unique topological sorting
\cite{skiena2008section}. This
allows to march through the cells in a proper way, i.e. visiting cell $C_i$
after visiting all cells that it depends on.
We will show that the
dependency graph $G$ is indeed acyclic for a triangulation of regular
triangles and no hanging nodes and sweeping
is therefore always possible.

While a lot of research has focused onto the development of fast sweeping
strategies, especially for parallel architectures
\cite{bailey2008analysis,baker1998sn,koch1992parallel,koch1992solution}, the
authors are not aware of a rigorous proof that sweeping is always possible
under
certain
conditions.

\begin{figure}[h!]%
  \centering
  \includegraphics[width=0.3\linewidth]{vectors}
  \caption{Incoming and outgoing normal vector for cell $C_i$ and the
  corresponding angle $\alpha_i$.}
  \label{fig:cellci}
\end{figure}

% Fig. \ref{fig:discretizationAndOrdering} shows two resulting dependency
%graphs
% for the same triangulation, each for a different direction $\Omega$.

\newpage

\section{Main Result}
\begin{theorem}{1}
  Consider a domain that is discretized by a set of cells
  $\{C_i\}_{i=1,\dots,I}$ where each cell $C_i$ is a triangle and we
  do not allow
  for hanging nodes, shown in Fig. \ref{fig:mesh}.
  Furthermore, we have a
  fixed direction $\Omega$ that
  prescribes the flow of
  information. Under these conditions, sweeping is possible.
\end{theorem}
\begin{figure}
  \centering
  \begin{minipage}{0.3\textwidth}
    \includegraphics[width=1\linewidth,left]{1.pdf}
    \caption{Triangulation of an arbitrary domain.}
    \label{fig:mesh}
  \end{minipage}
  \hspace{3cm}
  \begin{minipage}{0.3\textwidth}
    \includegraphics[width=1\linewidth,right]{2c.pdf}
    \caption{A cycle within the given triangulation.}
    \label{fig:cycle}
  \end{minipage}
\end{figure}

\begin{proof}
  We know that sweeping is possible, if and only if the induced dependency
  graph
  $G$ is acyclic.
  Now assume the dependency graph is not acyclic and provoke a
  contradiction.

  If the dependency graph $G$ is not acyclic, there exists a sequence of
  cells \\
  $\{C_{i_1},C_{i_2}, \dots,C_{i_M},C_{i_1}\}$, such that two consecutive
  cells
  share an edge
  $e_{{i_j},{i_{j+1}}}$ and $\langle n_{i_j},\Omega_k\rangle >0, \forall
  j=1,\dots,M$. Without loss of generality, let $i_j=j$ and assume that we
  pass
  through the cells in a counterclockwise manner as shown in Fig
  \ref{fig:cycle}.

  Now label the angles between $n_i$ and $n_{i+1}$ by
  $\alpha_{i}$ as shown in Fig.\ref{fig:cellci}. A counterclockwise turn
  corresponds to $\alpha_{i}>0$ and
  a
  clockwise turn to $\alpha_{i}<0$.
  We know that $-\pi<\alpha_{i}<\pi$ as we
  consider regular triangles. Since we perform a full counterclockwise turn,
  $\sum_{i=1}^M \alpha_i=2\pi$.
  Denote the angle between $n_1$ and $\Omega^\perp$ by $\theta$ as sketched
  in
  Fig. \ref{fig:vectors2}, with $0<\theta<\pi$ and
  $\Omega^\perp$ the vector normal to
  $\Omega$.
  Let $R_{\alpha_i}$ be the rotation matrix that encodes rotating with
  magnitude $\alpha_i$ around the $z$-axis.
  Then
  \begin{align*}
    n_{i+1} =R_{\alpha_{i}}n_i = \prod_{k=1}^{i} R_{\alpha_{k}}
    n_1= R_{\sum_{k=1}^{i} \alpha_k} n_1.
  \end{align*}
  If we turn $n_1$ (counterclockwise) by more than $\theta$ but less than
  $\theta+\pi$, then $\langle n_1,\Omega\rangle<0$.
  However, there exist $i^*$ such that
  $\sum_{k=1}^{i^*-1} \alpha_k \leq \theta$, but $\theta<\sum_{k=1}^{{i^*}}
  \alpha_k <
  \theta+\pi<2\pi$, as $-\pi<\alpha_{i^*}<\pi$.
  Then $n_{i^*+1}$ is $n_1$ turned (counterclockwise) by more than $\theta$,
  but
  less
  than $\theta+\pi$.
  Therefore $\langle n_{i^*+1}, \Omega\rangle <0$ which contradicts the
  assumption
  and finishes
  the proof.
\end{proof}

\begin{figure}[]
  \centering
  \includegraphics[width=0.3\linewidth]{vectors2}
  \caption{The normal $n_1$, with $\Omega$ and $\Omega^\perp$, as well as the
  angle $\theta$ between $n_1$ and $\Omega^\perp$.}
  \label{fig:vectors2}
\end{figure}

\begin{figure}
  \centering
  \includegraphics[width=0.7\linewidth]{parallelogram}
  \caption{Quadrilaterals can be arranged in a circle such that no
    topological
  ordering can be obtained for $\Omega=(0,0,-1)^T$. }
  \label{fig:parallelogram}
\end{figure}

\section{The three dimensional case}
The analogue setup for the three dimensional case would be a triangulation with
tetrahedra, again with no hanging nodes.
% Here, the previously described chain
%of arguments does no longer hold, since we have rotations around different
%axes
%(always the edge that three consecutive tetrahedra share). Rewriting the
%product
%of rotation matrices as the rotation matrix of the sum of the angles is
%therefore
%no longer possible.
In general, allowing arbitrary convex quadrilaterals for the triangulation of a
computational domain does not imply the absence of circular dependencies. This
is sketched in Fig. \ref{fig:parallelogram}. Quadrilaterals can be arranged in
a plane, perpendicular to the $z$-axis such that they form a circular
dependency for a direction of flow along the $z$-axis.

For tetrahedra however, the authors assume the analogue result to hold as for
the two dimensional case. In the absence of hanging nodes, sweeping should be
possible. Turning the previous proof into a three dimensional version appears
difficult, as rotation matrices no longer rotate around the same axis.
Therefore different techniques have to be investigated to proof this claim.
\bibliographystyle{plain}
\bibliography{Camminady}

\end{document}
