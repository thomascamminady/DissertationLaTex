\begin{figure}[t]
  % S4
  \begin{subfigure}{0.24\textwidth}
    \includegraphics[width=1.0\linewidth]{Chapters/Chapter3/fig/compressing/PLOTS_1/r0S4cuts.pdf}
  \end{subfigure}
  \hfill
  \begin{subfigure}{0.24\textwidth}
    \includegraphics[width=1.0\linewidth]{Chapters/Chapter3/fig/compressing/PLOTS_1/r4S4cuts.pdf}
  \end{subfigure}
  \hfill
  \begin{subfigure}{0.24\textwidth}
    \includegraphics[width=1.0\linewidth]{Chapters/Chapter3/fig/compressing/PLOTS_1/r8S4cuts.pdf}
  \end{subfigure}
  \hfill
  \begin{subfigure}{0.24\textwidth}
    \includegraphics[width=1.0\linewidth]{Chapters/Chapter3/fig/compressing/PLOTS_1/r10S4cuts.pdf}
  \end{subfigure}

  % S6
  \begin{subfigure}{0.24\textwidth}
    \includegraphics[width=1.0\linewidth]{Chapters/Chapter3/fig/compressing/PLOTS_1/r0S6cuts.pdf}
  \end{subfigure}
  \hfill
  \begin{subfigure}{0.24\textwidth}
    \includegraphics[width=1.0\linewidth]{Chapters/Chapter3/fig/compressing/PLOTS_1/r4S6cuts.pdf}
  \end{subfigure}
  \hfill
  \begin{subfigure}{0.24\textwidth}
    \includegraphics[width=1.0\linewidth]{Chapters/Chapter3/fig/compressing/PLOTS_1/r8S6cuts.pdf}
  \end{subfigure}
  \hfill
  \begin{subfigure}{0.24\textwidth}
    \includegraphics[width=1.0\linewidth]{Chapters/Chapter3/fig/compressing/PLOTS_1/r10S6cuts.pdf}
  \end{subfigure}

  % S8
  \begin{subfigure}{0.24\textwidth}
    \includegraphics[width=1.0\linewidth]{Chapters/Chapter3/fig/compressing/PLOTS_1/r0S8cuts.pdf}
  \end{subfigure}
  \hfill
  \begin{subfigure}{0.24\textwidth}
    \includegraphics[width=1.0\linewidth]{Chapters/Chapter3/fig/compressing/PLOTS_1/r4S8cuts.pdf}
  \end{subfigure}
  \hfill
  \begin{subfigure}{0.24\textwidth}
    \includegraphics[width=1.0\linewidth]{Chapters/Chapter3/fig/compressing/PLOTS_1/r8S8cuts.pdf}
  \end{subfigure}
  \hfill
  \begin{subfigure}{0.24\textwidth}
    \includegraphics[width=1.0\linewidth]{Chapters/Chapter3/fig/compressing/PLOTS_1/r10S8cuts.pdf}
  \end{subfigure}

  % S10
  \begin{subfigure}{0.24\textwidth}
    \includegraphics[width=1.0\linewidth]{Chapters/Chapter3/fig/compressing/PLOTS_1/r0S10cuts.pdf}
  \end{subfigure}
  \hfill
  \begin{subfigure}{0.24\textwidth}
    \includegraphics[width=1.0\linewidth]{Chapters/Chapter3/fig/compressing/PLOTS_1/r4S10cuts.pdf}
  \end{subfigure}
  \hfill
  \begin{subfigure}{0.24\textwidth}
    \includegraphics[width=1.0\linewidth]{Chapters/Chapter3/fig/compressing/PLOTS_1/r8S10cuts.pdf}
  \end{subfigure}
  \hfill
  \begin{subfigure}{0.24\textwidth}
    \includegraphics[width=1.0\linewidth]{Chapters/Chapter3/fig/compressing/PLOTS_1/r10S10cuts.pdf}
  \end{subfigure}

  \caption{Scalar flux for the line-source test case  at $t=1 $
    second using the octahedron (lerp) quadrature.
    The solution is visualized along three cuts: Horizontally (blue),
    vertically (purple), and diagonally (orange). The reference
    solution is given in black. The spatial domain is $[-1.5 \text{
    cm}, 1.5\text{ cm}]^2$ with $n_x=n_y = 100$. Rows vary in $n_q$ and
  columns vary in $\delta$, respectively.}
  \label{fig:rsn:ls:cuts}
\end{figure}
