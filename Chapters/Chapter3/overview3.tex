\begin{figure}[t]
	% S4
	\begin{subfigure}{0.24\textwidth}
		\includegraphics[width=1.0\linewidth]{Chapters/Chapter3/fig/compressing/PLOTS_2/r0S4imagesclarge.pdf}
	\end{subfigure}
	\hfill
	\begin{subfigure}{0.24\textwidth}
		\includegraphics[width=1.0\linewidth]{Chapters/Chapter3/fig/compressing/PLOTS_2/r4S4imagesclarge.pdf}
	\end{subfigure}
	\hfill
	\begin{subfigure}{0.24\textwidth}
		\includegraphics[width=1.0\linewidth]{Chapters/Chapter3/fig/compressing/PLOTS_2/r8S4imagesclarge.pdf}
	\end{subfigure}
	\hfill
	\begin{subfigure}{0.24\textwidth}
		\includegraphics[width=1.0\linewidth]{Chapters/Chapter3/fig/compressing/PLOTS_2/r10S4imagesclarge.pdf}
	\end{subfigure}
	
	% S6
	\begin{subfigure}{0.24\textwidth}
		\includegraphics[width=1.0\linewidth]{Chapters/Chapter3/fig/compressing/PLOTS_2/r0S6imagesclarge.pdf}
	\end{subfigure}
	\hfill
	\begin{subfigure}{0.24\textwidth}
		\includegraphics[width=1.0\linewidth]{Chapters/Chapter3/fig/compressing/PLOTS_2/r4S6imagesclarge.pdf}
	\end{subfigure}
	\hfill
	\begin{subfigure}{0.24\textwidth}
		\includegraphics[width=1.0\linewidth]{Chapters/Chapter3/fig/compressing/PLOTS_2/r8S6imagesclarge.pdf}
	\end{subfigure}
	\hfill
	\begin{subfigure}{0.24\textwidth}
		\includegraphics[width=1.0\linewidth]{Chapters/Chapter3/fig/compressing/PLOTS_2/r10S6imagesclarge.pdf}
	\end{subfigure}
	
	% S8
	\begin{subfigure}{0.24\textwidth}
		\includegraphics[width=1.0\linewidth]{Chapters/Chapter3/fig/compressing/PLOTS_2/r0S8imagesclarge.pdf}
	\end{subfigure}
	\hfill
	\begin{subfigure}{0.24\textwidth}
		\includegraphics[width=1.0\linewidth]{Chapters/Chapter3/fig/compressing/PLOTS_2/r4S8imagesclarge.pdf}
	\end{subfigure}
	\hfill
	\begin{subfigure}{0.24\textwidth}
		\includegraphics[width=1.0\linewidth]{Chapters/Chapter3/fig/compressing/PLOTS_2/r8S8imagesclarge.pdf}
	\end{subfigure}
	\hfill
	\begin{subfigure}{0.24\textwidth}
		\includegraphics[width=1.0\linewidth]{Chapters/Chapter3/fig/compressing/PLOTS_2/r10S8imagesclarge.pdf}
	\end{subfigure}	
	
	% S10
	\begin{subfigure}{0.24\textwidth}
		\includegraphics[width=1.0\linewidth]{Chapters/Chapter3/fig/compressing/PLOTS_2/r0S10imagesclarge.pdf}
	\end{subfigure}
	\hfill
	\begin{subfigure}{0.24\textwidth}
		\includegraphics[width=1.0\linewidth]{Chapters/Chapter3/fig/compressing/PLOTS_2/r4S10imagesclarge.pdf}
	\end{subfigure}
	\hfill
	\begin{subfigure}{0.24\textwidth}
		\includegraphics[width=1.0\linewidth]{Chapters/Chapter3/fig/compressing/PLOTS_2/r8S10imagesclarge.pdf}
	\end{subfigure}
	\hfill
	\begin{subfigure}{0.24\textwidth}
		\includegraphics[width=1.0\linewidth]{Chapters/Chapter3/fig/compressing/PLOTS_2/r10S10imagesclarge.pdf}
	\end{subfigure}	

	\caption{Scalar flux ($\log_{10}$) for the lattice test case  at $t=3.2 $ seconds using the octahedron (lerp) quadrature.  The spatial domain is $[0 \text{ cm}, 7\text{ cm}]^2$ with $n_x=n_y = 280$. Rows vary in $n_q$ and columns vary in $\delta$, respectively. Isolines are drawn to highlight the values $-5$ (white), $-4$ (gray), and $-3$ (black).}
	\label{fig:rsn:cb}
\end{figure}