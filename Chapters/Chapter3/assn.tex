\FloatBarrier
\section{as-S$_N$ method}
\label{sec:asSNmethod}
Section \ref{sec:asSNmethod} is, at parts, based upon the results in
Frank et al. \cite{frank2020ray}. Findings therein have subsequently
contributed to the thesis of Kusch  \cite{10.5445/IR/1000121168},
coauthor of  Frank et al. \cite{frank2020ray}.

Section \ref{sec:understanding} demonstrates that ray effects are
essentially due to the aggregation of the scalar flux, \ie the
inability to compute the spherical integral reasonably well.
Quintessentially, the rS$_N$ method adds angular diffusion,
facilitating to integrate the angular flux numerically.
Recomputing the sweeping order anew after each time step might
ultimately be an unbearable burden when it comes to the integration
of the rS${}_N$ method into existing S$_N$ codes; rendering the
rS$_N$ difficult---if not impossible---to sell. Although this problem
can seemingly be avoided when rotating forth and back, this
modification of rS${}_N$ requires further research.

The artificial scattering S$_N$ (as-S$_N$) method overcomes this
hurdle while simultaneously mitigating ray effects equally well. At
its core, the as-S$_N$ method adds artificial scattering to the
right-hand side of the transport equation in such a way that the
following requirements are satisfied:
(i) In the limit of infinite ordinates, the added term should vanish.
(ii) Artificial scattering should be independent of the number of
spatial cells and time steps. (iii) Despite adding a nonphysical term
to the right-hand side, the solution quality should increase due to
the mitigated ray effects. (iv) The ordinates remain unchanged
throughout the simulation. Lastly, (v) the effect of artificial
scattering should be investigatable both numerically and theoretically.

Starting with the linear transport equation
\begin{equation}
  \begin{split}
    \partial_t \psi(t,\bx,\bOmega) +& \bOmega \cdot \nabla_\bx
    \psi(t,\bx,\bOmega) +\sigma_t
    \psi(t,\bx,\bOmega) \\
    =& {\sigma_s}\int_{\mathbb{S}^2}s(\bOmega \cdot
    \bOmega')\psi(t,\bx,\bOmega') \, d\bOmega'
    + q(t,\bx,\bOmega),
  \end{split}
\end{equation}
we add artificial scattering to the right-hand side to obtain
\begin{equation}
  \begin{split}
    \partial_t \psi(t,\bx,\bOmega) +& \bOmega \cdot \nabla_\bx
    \psi(t,\bx,\bOmega) +\sigma_t
    \psi(t,\bx,\bOmega) \\
    =&{\sigma_s}\int_{\mathbb{S}^2}s(\bOmega \cdot
    \bOmega')\psi(t,\bx,\bOmega') \, d\bOmega'+ q(t,\bx,\bOmega)
    \\
    &+{\sigma_{as}}\int_{\mathbb{S}^2}s_\varepsilon(\bOmega \cdot
    \bOmega')\left(\psi(t,\bx,\bOmega') -\psi(t,\bx,\bOmega)
    \right)\, d\bOmega'.
  \end{split}
  \label{eq:assn}
\end{equation}
Choosing $s_\varepsilon: [-1,1] \to \mathbb{R}$ as
\begin{align}
  s_\varepsilon(\mu) = \frac{2}{\sqrt{\pi}
  \varepsilon\text{Erf}\left(\frac{2}{\varepsilon}\right)}
  \exp\left(-\frac{(1-\mu)^2}{\varepsilon^2}\right)
\end{align}
complies with the five requirements that we imposed.
The error function satisfies $\text{Erf}(x) \rightarrow 1$ as
$x\rightarrow \infty$.
In general, $s_\varepsilon$ can be any Dirac-like sequence, \ie
\begin{align}
  \int_{-1}^1s_\varepsilon(\mu) \, d\mu = 1 \quad \text{ and } \quad
  \int_{-1}^1s_\varepsilon(\mu) f(\mu)\, d\mu \rightarrow f(0)
\end{align}
for any sufficiently smooth function $f$ and $\varepsilon\rightarrow 0$.
We paraphrase some of the remarks that were initially made in Frank
et al. \cite{frank2020ray}:
\begin{itemize}
  \item In- and out-scattering cancel if $\varepsilon \rightarrow 0$
    and artificial scattering therefore vanishes. In compliance with
    (i), we therefore set $\varepsilon = \beta / n_q$ and let the
    user choose $\beta$.
  \item Particles are neither gained nor lost because the scattering
    kernel $s_\varepsilon$ is mass preserving.
  \item Similar to the P$_{N-1}$-equivalent S$_N$ method
    \cite{lathrop1971remedies,doi:10.1080/00411450.2012.672360}, we
    add a fictions term to the standard transport equation. However,
    the kernel in the as-S$_N$ method is highly forward-peaked.
  \item The as-S$_N$ method can be solved efficiently, both via
    implicit or explicit schemes.
\end{itemize}
More information on the implicit implementation of the as-S$_N$
method can be found in the original paper   \cite{frank2020ray} and
the thesis by Kusch \cite{10.5445/IR/1000121168}.
Worth mentioning, however, is the fact that the physical scattering
term and the artificial scattering term are treated differently. When
physical scattering is mostly isotropic, an expansion in spherical
harmonics is used. Contrary to that, artificial scattering is highly
forward-peaked and thus well-suited for sweeping methods. If this
distinction is omitted, implementing the as-S$_N$ method into
existing S$_N$ codes is simply realized by changing the scattering kernel.

\subsection{Angular discretization}

\label{sec:implassn}
Our goal is to include artificial scattering in the S$_N$ equations
in \eqref{eq:SN}. By simply approximating the artificial scattering
term in \eqref{eq:assn} with the chosen quadrature rule, we obtain
the \aSN equations
\begin{equation}
  \begin{split}
    \label{eq:asSNeq}
    \partial_t \psi_q(t,\bx) + \bOmega_q &\cdot \nabla_\bx
    \psi_q(t,\bx) + \sigma_t(\bx) \psi_q(t,\bx) +
    \sigma_{\text{as}}(\bx)\psi_q(t,\bx)
    \\=&\sigma_s(\bx) \sum_{p=1}^{n_q} w_p \cdot c_q \cdot
    s(\bOmega_q\cdot\bOmega_p) \psi_p(t,\bx)
    \\
    &+\sigma_{\text{as}}(\bx)\sum_{p=1}^{n_q}w_p \cdot
    c_q^{(\varepsilon)}\cdot s_\varepsilon(\bOmega_q\cdot
    \bOmega_p)\psi_p(t,\bx) \\
    &+  q_q(t,\bx)
    .
  \end{split}
\end{equation}
Here, $c_q := 1 / \sum_p w_p \cdot s(\bOmega_q\cdot \bOmega_p)$ and
$c_q^{(\varepsilon)} := 1 / \sum_p w_p \cdot
s_\varepsilon(\bOmega_q\cdot \bOmega_p)$ are normalization factors.
While, on the continuous level, these factors are the same for every
direction, we obtain a dependency on the chosen ordinates due to the
non-uniform discretization in angle. These normalization factors are
needed to obtain a simple expression for the out-scattering terms.
Moving these terms to the left-hand side of \eqref{eq:asSNeq}
stabilizes the source iteration that is used in the implicit method.
%We will often move the artificial out-scattering to the right-hand
% side via defining $\sigma_{as,q}(\bx) =
% \sigma_{\text{as}}(\bx)\sum_{p} w_p \cdot c_q\cdot
% s_\varepsilon(\bOmega_q\cdot\bOmega_p)$.

It remains to pick an adequate quadrature set. When applying
artificial scattering, the solution smears out along the directions
in the quadrature set. To ensure an evenly spread artificial
scattering effect, a quadrature with a highly uniform ordinate
distribution should be chosen.
Here, we decide to use the icosahedron (slerp) quadrature due to the
nearly uniform distribution of quadrature points. Other quadratures
were explored too, but the best results were achieved with the
icosahedron quadrature, used in all following computations.

\subsection{ Modified equation and asymptotic analysis}
According to Pomraning \cite{pomraning1992fokker}, the Fokker-Planck
operator can be a legitimate description of highly peaked scattering.
This is true if (i) the scattering kernel $s_\varepsilon(\mu)$ is a
Dirac sequence, and (ii) the transport coefficients
$p_{\varepsilon,i}:= \int_{-1}^{+1} (1-\mu)^i s_\varepsilon(\mu)\,
d\mu$ are of order $\mathcal{O}(\varepsilon^i)$.
The resulting modified equation then reads
\begin{equation} \label{eq:fokkerplancklimit}
  \begin{split}
    \partial_t \psi(t,\bx,\bOmega) &+ \bOmega\cdot \nabla_\bx
    \psi(t,\bx,\bOmega) + (\sigma_a+ \sigma_s ) \psi(t,\bx,\bOmega) \\
    &= \sigma_s \cdot \phi(t,\bx) +\pi\cdot p_{\varepsilon,1}\cdot
    \sigma_{as}\cdot  \Delta_\bOmega\, \psi(t,\bx,\bOmega)+
    \mathcal{O}\left(\varepsilon^2\right),
  \end{split}
\end{equation}
where $\Delta_\bOmega$ is the Laplace operator in spherical coordinates.
We have already shown (i).
To verify (ii), let $y=(1-\mu) / \varepsilon$. Then
\begin{subequations}
  \begin{align}
    p_{\varepsilon,i} &= \int_{2/\varepsilon}^0 (\varepsilon \, y)^i
    \frac{2}{\sqrt{\pi}\, \varepsilon
    \,\text{Erf}\left(\frac{2}{\varepsilon}\right)}
    e^{-y^2} (-\varepsilon) \, dy \\
    &=
    \frac{2}{\sqrt{\pi}\, \varepsilon
    \,\text{Erf}\left(\frac{2}{\varepsilon}\right)}\, \varepsilon^i
    \, \int_0^{2/\varepsilon}y^i e^{-y^2}\, dy \\
    &= \frac{2}{\sqrt{\pi}\, \varepsilon
    \,\text{Erf}\left(\frac{2}{\varepsilon}\right)}\, \varepsilon^i
    \left[\Gamma\left(\frac{1+i}{2}\right)-\Gamma\left(\frac{1+i}{2},\frac{4}{\varepsilon^2}\right)\right]
    \\
    &= \mathcal{O}(\varepsilon^i),
  \end{align}
\end{subequations}
where $\Gamma(\cdot)$ and $\Gamma(\cdot,\cdot)$ denote the gamma
function and the upper incomplete gamma function, respectively.
Since (ii) implies that $p_{\varepsilon,1} =
\mathcal{O}(\varepsilon)$, the operator vanishes if we let $\varepsilon\to 0$.
We set $\varepsilon=\beta /n_q$ in the discrete case so that the
angular diffusion vanishes if the number of ordinates $n_q$ tends to infinity.
This analysis shows that the product $\sigma_{\text{as}}\cdot \beta$
controls the strength of the added angular diffusion.

The asymptotic analysis that  follows is a one-to-one translation of
the original analysis \cite{pomraning1992fokker} and
substitutes the new scattering kernel. We omit the source term and
shorten notation to get
\begin{equation}
  \begin{split}
    \underbrace{\partial_t \psi(t,\bx,\bOmega) + \bOmega\cdot \nabla
    \psi}_{=:L\psi} + \sigma_a \psi
    =& \sigma_s \int_{\mathbb{S}^2}
    \frac{1}{4\pi}\left(\psi(\bOmega')-\psi(\bOmega)\right)\, d\bOmega' \\
    &+\sigma_{as} \int_{\mathbb{S}^2} s_{\varepsilon}(\bOmega \cdot
    \bOmega')\left(\psi(\bOmega')-\psi(\bOmega)\right)\, d\bOmega',
  \end{split}
\end{equation}
which we further rewrite as
\begin{equation}
  \begin{split}
    L\psi + (\sigma_a+ \sigma_s ) \psi
    = \sigma_s \phi +\sigma_{as} \int_{\mathbb{S}^2}
    s_{\varepsilon}(\bOmega \cdot
    \bOmega')\left(\psi(\bOmega')-\psi(\bOmega)\right)\, d\bOmega' .
  \end{split}
\end{equation}
Next, define $c_\varepsilon =\int_{\mathbb{S}^2}
s_{\varepsilon}(\bOmega \cdot \bOmega') \, d\bOmega'$ to get
\begin{equation}
  \begin{split}
    L\psi + (\sigma_a+ \sigma_s +\sigma_{as} c_\varepsilon) \psi
    = \sigma_s \phi +\sigma_{as} \int_{\mathbb{S}^2}
    s_{\varepsilon}(\bOmega \cdot \bOmega')\psi(\bOmega')\, d\bOmega' .
  \end{split}
\end{equation}
Writing the scattering kernel in terms of Legendre polynomials with
coefficients \newline  $s_{\varepsilon,n} = 2\pi \int_{-1}^1
s_{\varepsilon}(\mu)P_n(\mu) d\mu$ , we obtain
\begin{equation}
  \begin{split}
    \underbrace{L\psi + (\sigma_a+ \sigma_s +\sigma_{as}
    c_\varepsilon) \psi }_{\text{=:LHS}}
    = \sigma_s \phi +\sigma_{as} \int_{\mathbb{S}^2}
    \sum_{n=0}^\infty \frac{2n+1}{4\pi} s_{\varepsilon,n} P_n(\bOmega
    \cdot \bOmega')\psi(\bOmega')\, d\bOmega' .
  \end{split}
\end{equation}
We can express $P_n(\bOmega \cdot \bOmega')$ and $\psi(\bOmega)$ in
terms of spherical harmonics as
\begin{equation}
  \begin{split}
    P_n(\bOmega \cdot \bOmega') = \sum_{l=-n}^n a_{n,l}
    Y_{n,l}(\bOmega)\overline{Y_{n,l}(\bOmega')}
  \end{split}
\end{equation}
and
\begin{equation}
  \begin{split}
    \psi(\bOmega ) = \sum_{n=0}^\infty \sum_{m=-n}^n
    \frac{2n+1}{4\pi} a_{n,m} \psi_{n,m} Y_{n,m}(\bOmega).
  \end{split}
\end{equation}
Combining the last three equations yields
\begin{equation}
  \begin{split}
    \text{LHS} = \sigma_s \phi +\sigma_{as} \int_{\mathbb{S}^2}
    \sum_{n=0}^\infty \frac{2n+1}{4\pi} s_{\varepsilon,n}
    \sum_{l=-n}^n a_{n,l}
    Y_{n,l}(\bOmega)\overline{Y_{n,l}(\bOmega')} \sum_{m=-n}^n
    \frac{2n+1}{4\pi} a_{n,m} \psi_{n,m} Y_{n,m}(\bOmega')\, d\bOmega'.
  \end{split}
\end{equation}
Using orthogonality
$
\int_{\mathbb{S}^2} Y_{n,m}(\bOmega) \overline{Y_{k,l}} (\bOmega) =
\frac{4\pi}{2n+1}\frac{1}{a_{n,m}} \delta_{nk,ml}
$
then yields
\begin{subequations}
  \begin{align}
    \text{LHS} &= \sigma_s \phi +\sigma_{as} \int_{\mathbb{S}^2}
    \sum_{n=0}^\infty \frac{2n+1}{2} s_{\varepsilon,n}  \sum_{l=-n}^n
    a_{n,l} Y_{n,l}(\bOmega)\overline{Y_{n,l}(\bOmega')}
    \sum_{m=-n}^n \frac{2n+1}{4\pi} a_{n,m} \psi_{n,m}
    Y_{n,m}(\bOmega')\, d\bOmega' \\
    &=
    \sigma_s \phi +\sigma_{as}   \sum_{n=0}^\infty \frac{2n+1}{4\pi}
    s_{\varepsilon,n}  \sum_{l=-n}^n a_{n,l} Y_{n,l}(\bOmega)
    \sum_{m=-n}^n \frac{2n+1}{4\pi} a_{n,m} \psi_{n,m}
    \int_{\mathbb{S}^2}
    Y_{n,m}(\bOmega')\overline{Y_{n,l}(\bOmega')}\, d\bOmega' \\
    &=
    \sigma_s \phi +\sigma_{as}   \sum_{n=0}^\infty \frac{2n+1}{4\pi}
    s_{\varepsilon,n}  \sum_{l=-n}^n a_{n,l} Y_{n,l}(\bOmega)
    \sum_{m=-n}^n \frac{2n+1}{4\pi} a_{n,m} \psi_{n,m}
    \frac{4\pi}{2n+1}\frac{1}{a_{n,m}} \delta_{nn,ml} \\
    &=
    \sigma_s \phi +\sigma_{as}   \sum_{n=0}^\infty \frac{2n+1}{4\pi}
    s_{\varepsilon,n} \sum_{m=-n}^n   a_{n,m} Y_{n,m}(\bOmega)
    \frac{2n+1}{4\pi} a_{n,m} \psi_{n,m}
    \frac{4\pi}{2n+1}\frac{1}{a_{n,m}}  \\
    &=
    \sigma_s \phi +\sigma_{as}   \sum_{n=0}^\infty \frac{2n+1}{4\pi}
    s_{\varepsilon,n} \sum_{m=-n}^n   a_{n,m} \psi_{n,m} Y_{n,m}(\bOmega)\\
    &=
    \sigma_s \phi +\sigma_{as}   \sum_{n=0}^\infty\sum_{m=-n}^n
    \frac{2n+1}{4\pi}  a_{n,m} Y_{n,m}(\bOmega) \psi_{n,m} {2\pi}
    \int_{-1}^1 s_{\varepsilon}(\mu)P_n(\mu) d\mu\\
    &=
    \sigma_s \phi +\sigma_{as}   \sum_{n=0}^\infty  \frac{2n+1}{2}
    \int_{-1}^1 s_{\varepsilon}(\mu)P_n(\mu) d\mu\sum_{m=-n}^n
    a_{n,m} Y_{n,m}(\bOmega) \psi_{n,m}
    .
  \end{align}
\end{subequations}
Summarizing the above derivation, we obtain
\begin{equation}
  \begin{split}
    L\psi + (\sigma_a+ \sigma_s +\sigma_{as} c_\varepsilon) \psi =
    \sigma_s \phi +\sigma_{as}   \sum_{n=0}^\infty  \frac{2n+1}{2}
    \int_{-1}^1 s_{\varepsilon}(\mu)P_n(\mu) d\mu\sum_{m=-n}^n
    a_{n,m}\, \psi_{n,m} \,  Y_{n,m}(\bOmega) .
  \end{split}
\end{equation}
Next, we define $\hat{s}_\varepsilon(\mu) = s_\varepsilon(\mu) /
c_{\varepsilon}$, substitute it into the above equation, and drop the hat to get
\begin{equation}
  \begin{split}
    L\psi + (\sigma_a+ \sigma_s +\sigma_{as} ) \psi = \sigma_s \phi
    +\sigma_{as}   \sum_{n=0}^\infty  \frac{2n+1}{2} \int_{-1}^1
    s_{\varepsilon}(\mu)P_n(\mu) d\mu\sum_{m=-n}^n     a_{n,m}\,
    \psi_{n,m} \,  Y_{n,m}(\bOmega) .
  \end{split}
\end{equation}
Let now $\sigma_{as} = \hat \sigma_{as} / \Delta$ with $\hat
\sigma_{as} = \mathcal{O}(1)$. The fast variable $y$ is given by
$y=(1-\mu)/\delta$  and $\varepsilon=\delta$. Here, $\delta, \Delta \ll 1$.
Then
$
s_\varepsilon(\mu) = \hat s_\varepsilon(y)
$
and we obtain

\begin{subequations}
  \begin{align}
    \text{LHS} &= \sigma_s \phi +\frac{\hat \sigma_{as}}{\Delta}
    \sum_{n=0}^\infty  \frac{2n+1}{2} \underbrace{\int_{-1}^1 \hat
      s_{\varepsilon}\left(\frac{1-\mu}{\delta}\right)P_n(\mu)\,
    d\mu}_{\textstyle \underbrace{= \delta \int_{0}^{2/\delta} \hat
      s_{\varepsilon}(y)P_n(1-\delta y)\, dy}_{\textstyle =\delta
        \int_0^{2/\delta}\hat s_\varepsilon(y)\left(P_n(1) -\delta
    yP'_n(1) + \mathcal{O}(\delta)^2\right) \, dy}}\sum_{m=-n}^n
    a_{n,m}\, \psi_{n,m} \,  Y_{n,m}(\bOmega) \\
    &=\sigma_s \phi +\hat \sigma_{as}\frac{\delta }{\Delta}
    \sum_{n=0}^\infty  \frac{2n+1}{2} \int_0^{2/\delta}\hat
    s_\varepsilon(y)\left(P_n(1) -\delta yP'_n(1) +
    \mathcal{O}(\delta)^2\right) \, dy\underbrace{\sum_{m=-n}^n
    a_{n,m}\, \psi_{n,m} \,  Y_{n,m}(\bOmega)}_{C} . \label{eq:taylor}
  \end{align}
\end{subequations}
Together with $P_n(1)=1$ and $P_n'(1) = n(n+1)/2$
and changing back according to $y=(1-\mu)/\delta$ results in
\begin{equation}
  \begin{split}
    \text{LHS}  =&\sigma_s \phi+\hat \sigma_{as}\frac{1 }{\Delta}
    \sum_{n=0}^\infty  \frac{2n+1}{2} \int_{-1}^{1}\hat
    s_\varepsilon\left(\frac{1-\mu}{\delta}\right) \, d\mu\cdot C \\
    &-\hat \sigma_{as}\frac{n(n+1)}{2}\frac{1 }{\Delta}
    \sum_{n=0}^\infty  \frac{2n+1}{2} \int_{-1}^{1}\hat
    s_\varepsilon\left(\frac{1-\mu}{\delta}\right)(1-\mu) \, d\mu\cdot C
    \\
    &+\mathcal{O}\left(\frac{\delta^2}{\Delta}\right).
  \end{split}
\end{equation}
Reversing the fast-to-slow transformation yields
\begin{equation}
  \begin{split}
    \text{LHS}
    &-\mathcal{O}\left(\frac{\delta^2}{\Delta}\right)\\ &=\sigma_s \phi+
    \sigma_{as}  \sum_{n=0}^\infty  \frac{2n+1}{2}
    \overbrace{\int_{-1}^{1} s_\varepsilon\left(\mu\right) \,
    d\mu}^{=1/(2\pi)}\cdot C
    - \sigma_{as}\frac{n(n+1)}{2}\sum_{n=0}^\infty  \frac{2n+1}{2}
    \int_{-1}^{1} s_\varepsilon\left(\mu\right)(1-\mu) \, d\mu\cdot C.
  \end{split}
\end{equation}
Here,
\begin{subequations}
  \begin{align}
    \sigma_{as} &= \mathcal{O}\left(\frac{1}{\Delta}\right), \\
    \sigma_{as}\int_{-1}^{1} s_\varepsilon\left(\mu\right)(1-\mu) \,
    d\mu &= \mathcal{O}\left(\frac{\delta}{\Delta}\right).
  \end{align}
\end{subequations}
The first line is obvious. The second line follows from the fact that
$(1-\mu) = \mathcal{O}(\delta)$, resulting in
\begin{equation}
  \begin{split}
    L\psi& + (\sigma_a+ \sigma_s +\sigma_{as} ) \psi \\
    &= \sigma_s \phi +\sigma_{as}   \sum_{n=0}^\infty \sum_{m=-n}^n
    \frac{2n+1}{4\pi}   a_{n,m}\, \psi_{n,m} \,  Y_{n,m}(\bOmega)
    \cdot \left[1  - n(n+1)\pi\int_{-1}^{1}
    s_\varepsilon\left(\mu\right)(1-\mu) \, d\mu \right] +
    \mathcal{O}\left(\frac{\delta^2}{\Delta}\right).
  \end{split}
\end{equation}
For the next step, we follow the  computations (52) and (53) in
Pomraning's work  \cite{pomraning1992fokker} to obtain %\todo{include
% this} This yields
\begin{equation}
  \begin{split}
    L\psi& + (\sigma_a+ \sigma_s ) \psi \\
    &= \sigma_s \phi +\underbrace{\sigma_{as}
      \pi\int_{-1}^{1}s_\varepsilon\left(\mu\right)(1-\mu) \,
    d\mu}_{\textstyle = \mathcal{O}\left(\sigma_{as}\cdot
    \varepsilon\right)}\cdot  \left[\frac{\partial}{\partial \mu}
      (1-\mu^2)\frac{\partial}{\partial \mu} +\frac{1}{1-\mu^2}
    \frac{\partial^2}{\partial \phi^2}\right]\psi+
    \mathcal{O}\left(\sigma_{as}\cdot \varepsilon^2\right),
  \end{split}
\end{equation}
which are the scaling properties that we expect to see and also
observe throughout the numerical experiments.

Before discussing the results of both the line-source and lattice
test case, it is worth mentioning where and how the theoretical
setting of the asymptotic analysis deviates from the actual setting
of the numerical experiments. Most importantly, the asymptotic
analysis works on the continuous level (in angle) while the as-S$_N$
implementation is already discretized (in angle). Next, when trying
to quantify the effect of artificial diffusion, we do so by proxying
the $L^2$ error with respect to a reference solution, rather than the
artificial diffusion itself. Lastly, while the effect of angular
diffusion is characterized by the product $\sigma_\text{as} \cdot
\varepsilon$, the $\mathcal{O}\left(\sigma_\text{as}\cdot
\varepsilon^2\right) $ term might, for certain parameter pairs, be
non-negligible to an extend where the difference between the as-S$_N$
and the S$_N$ method can not exclusively be attributed to the angular diffusion.

%Equipped with the insight of the asymptotic analysis and the remarks
% from the last paragraph, we can now discuss the numerical experiments.

\subsection{as-S$_N$ and the line-source test case}
Comparing the as-S$_N$ method with the rS$_N$ method, an arguable
drawback is the necessity two choose two parameters (\ie
$\sigma_\text{as}$ and $\beta$) instead of just one (\ie $\delta$).
However, the results from the asymptotic analysis show that (under
certain assumptions) the added artificial diffusion is controlled by
the product of $\sigma_\text{as}$ and $\beta$ rather than by each
parameter individually. This is mostly in agreement with the two
parameter studies that are summarized in Figures
\ref{fig:nse19-109figheatmapexplicitl2} and
\ref{fig:nse19-109figheatmapexplicitl2_fine}, computed for the coarse
setting $n_q \times n_x \times n_y = 12 \times 50 \times 50$ and the
fine setting $n_q \times n_x \times n_y = 92 \times 200 \times 200$,
respectively. In both settings, we (i) run as-S$_N$ simulations for
the line-source test case with different combinations of
$\sigma_\text{as}$ and $\beta$, (ii) compute the resulting $L^2$
error with respect to the semi-analytical reference solution, and
(iii) divide that error by the error obtained from a standard S$_N$
simulation. Each cell in the resulting heatmaps in Figures
\ref{fig:nse19-109figheatmapexplicitl2} and
\ref{fig:nse19-109figheatmapexplicitl2_fine} is then color-coded by
this \textit{baseline normalized error}. Values below (above) unity
indicate that the error has decreased (increased) when running the
as-S$_N$ method with that specific parameter pair. The optimal
parameter pair is highlighted in yellow and all pairs close to optimal in green.

In the coarse case, the optimal parameter choice of $\beta=3.5$ and
$\sigma_\text{as} = 18$ yields an error reduction down to
approximately $43$\%, measured in the $L^2$ norm over the full
spatial domain at the final time $t=1s$. Parameter pairs close to
optimal, \ie where the error does not exceed the optimal error by
more than $10\%$, mostly fall in the region $20 \leq \beta \cdot
\sigma_\text{as} \leq 80$, $\beta\geq 3$. For $120\leq \beta \cdot
\sigma_\text{as}$, the baseline normalized error will be above unity.
This corresponds to cases in which too much artificial scattering was added.

It is obviously infeasible to perform a full parameter study on the
same fine grid that the actual computation takes place on.
Consequently,  we may ask ourselves how well the coarse setting
performs as a surrogate for the fine setting. Comparing the heatmap
in Figure \ref{fig:nse19-109figheatmapexplicitl2} with the one in
Figure \ref{fig:nse19-109figheatmapexplicitl2_fine}, the success of
this strategy is evident.
While the optimal parameter pair in the fine setting reduces the
error down to $23$\%, the optimal parameter pair in the coarse
setting still reduces the error down to $42$\% when applied in the
fine setting. This is also visualized in Figure
\ref{fig:applysuboptimalparameters}.
The regions of optimal parameter configurations tend to overlap when
comparing both settings, although the fine setting has a wider valley
of optimal parameter pairs.

Computing an as-S$_N$ simulation with $n_q\times n_x \times n_y  = 92
\times 200 \times 200$ is approximately $490$ ($\approx 92/12 \times
4\times 4 \times 4 $, since also $n_t = \mathcal{O}(n_x,n_y)$) times
as costly as for $n_q\times n_x \times n_y  = 12 \times 50\times 50$.
Therefore, performing \textit{all} $306$ simulations for the coarse
heatmap, followed by one simulation in the fine setting with the
optimal parameter pair increases the runtime by $62$\% while the
error is decreased down to $42$\%. A similar return on investment
would not be achieved by simply increasing the number of ordinates in
the S$_N$ method.

Lastly, the discussion of an efficient implementation of the as-S$_N$
method has been omitted since all computations were done explicitly.
Because the physical and artificial scattering can be combined into a
new scattering kernel, the as-S$_N$ method yields no increase in runtime.

\begin{figure}[]
  \begin{subfigure}{0.27\textwidth}
    \includegraphics[width=1.0\linewidth]{Chapters/Chapter3/fig/lsassn/2020-07-29T12:50:04.603/200.0r0.0S4.0nx200.0cuts}
    \caption{S$_{4}$ with $92$ ordinates and no artificial
    scattering, computed with the icosahedron quadrature set.}
  \end{subfigure}
  \hfill
  \begin{subfigure}{0.27\textwidth}
    \includegraphics[width=1.0\linewidth]{Chapters/Chapter3/fig/lsassn/2020-07-29T12:54:25.179/200.0r0.0S4.0nx200.0cuts}
    \caption{Using artificial scattering with the optimal parameter
    set reduces the error down to about $23$\%.}
  \end{subfigure}
  \hfill
  \begin{subfigure}{0.27\textwidth}
    \includegraphics[width=1.0\linewidth]{Chapters/Chapter3/fig/lsassn/2020-07-29T12:56:08.095/200.0r0.0S4.0nx200.0cuts}
    \caption{Using the optimal parameter set computed in the coarse
    setting still reduces the error down to about $43$\%.}
  \end{subfigure}
  \caption{For the as-S$_N$ method, two parameters
    ($\sigma_\text{as}$ and $\beta$) need to be chosen. All three
    images show line-source cuts with $n_q \times n_x \times n_y = 92
    \times 200 \times 200$. Left, artificial scattering is not used.
    Center, $\sigma_\text{as} = 4.0$ and $\beta=8.0$, which corresponds
    to the optimal parameter found in the parameter study in Figure
    \ref{fig:nse19-109figheatmapexplicitl2_fine} on the same grid.
    Right,  $\sigma_\text{as} = 18.0$ and $\beta=3.5$, which are the
    optimal parameters from the coarse parameter study ($n_q \times n_x
  \times n_y = 12 \times 50 \times 50$).}
  \label{fig:applysuboptimalparameters}
\end{figure}

\begin{landscape}
  \begin{figure}
    \centering
    \includegraphics[width=1.0\linewidth]{Chapters/Chapter3/fig/assn/OLD/linesource/dummy}
    \caption{Parameter study for $\sigma_{\text{as}}$ and $\beta$ on
      the grid $n_q \times n_x \times n_y = 12 \times 50 \times 50$ in
      an explicit calculation. For every set of parameters, we compute
      the $L^2$ error of the scalar flux $\phi$ with respect to the
      semi-analytical reference solution on the same spatial grid. The
      number in each field of the heatmap is then the baseline
      normalized error, \ie the $L^2$ error obtained for that specific
      parameter configuration divided by the error obtained without
      artificial scattering. The optimal parameter configuration,
      highlighted in yellow, reduces the error down to 43\%.
    Close-to-optimal parameter pairs are highlighted in green.}
    \label{fig:nse19-109figheatmapexplicitl2}
  \end{figure}
\end{landscape}

\begin{landscape}
  \begin{figure}
    \centering
    \includegraphics[width=1.0\linewidth]{Chapters/Chapter3/fig/assn/NEW/tmp11}
    \caption{Parameter study for $\sigma_{\text{as}}$ and $\beta$ on
      the grid $n_q \times n_x \times n_y = 92 \times 200 \times200$ in
      an explicit calculation. For every set of parameters, we compute
      the $L^2$ error of the scalar flux $\phi$ with respect to the
      semi-analytical reference solution on the same spatial grid. The
      number in each field of the heatmap is then the baseline
      normalized error, \ie the $L^2$ error obtained for that specific
      parameter configuration divided by the error obtained without
      artificial scattering. The optimal parameter configuration,
      highlighted in yellow, reduces the error down to 23\%.
    Close-to-optimal parameter pairs are highlighted in green.}
    \label{fig:nse19-109figheatmapexplicitl2_fine}
  \end{figure}
\end{landscape}

\FloatBarrier

\subsection{as-S$_N$ and the lattice test case}
The quantitative benefits of artificial scattering are,
unsurprisingly, less pronounced in the lattice test case. This is
predominantly due to the fact that ray effects occur on neglectable
scales, orders of magnitudes smaller than the mean scalar flux.
Nevertheless, we will observe an error reduction when switching from
S$_N$ to as-S$_N$.
\begin{figure}[h!]
  \begin{subfigure}{0.24\textwidth}
    \includegraphics[width=1.0\linewidth]{Chapters/Chapter3/fig/assn_cb/ref_plasma}
    \caption{S$_{15}$ with $1962$ ordinates and no artificial
    scattering. Ray effects are still visible at the isolines.}
    \label{fig:assncbrefsolutionA}
  \end{subfigure}
  \hfill
  \begin{subfigure}{0.24\textwidth}
    \includegraphics[width=1.0\linewidth]{Chapters/Chapter3/fig/assn_cb/best_plasma2}
    \caption{For 92 ordinates, using $\sigma_\text{as} = 1.0$ and
    $\beta=12.0$ yields the best-case error (down to $69$\%).}
    \label{fig:assncbrefsolutionB}
  \end{subfigure}
  \hfill
  \begin{subfigure}{0.24\textwidth}
    \includegraphics[width=1.0\linewidth]{Chapters/Chapter3/fig/assn_cb/original_plasma}
    \caption{S$_{4}$ with $92$ ordinates and no artificial
    scattering. This computation serves as the baseline.}
    \label{fig:assncbrefsolutionC}
  \end{subfigure}
  \hfill
  \begin{subfigure}{0.24\textwidth}
    \includegraphics[width=1.0\linewidth]{Chapters/Chapter3/fig/assn_cb/worstcase}
    \caption{Worst-case scenario with too much artificial scattering
    yields a tenfold increase in  error.}
    \label{fig:assncbrefsolutionD}
  \end{subfigure}
  \caption{Results for the lattice test case with different amounts
    of artificial scattering. From left to right: A high-order
    reference solution; the best-case scenario; the baseline; and the
  worst-case scenario.}
  \label{fig:assncbrefsolution}
\end{figure}

To derive errors, we computed a reference solution that uses
$n_q=1962$ ordinates on a spatial grid of $n_x \times n_y = 280
\times 280$ cells, visualized in Figure \ref{fig:assncbrefsolutionA}.
The parameter study that was performed for the line-source test case
is replicated for the lattice test case ($n_q=92$), with the results
summarized in Figure \ref{fig:cbnse19-109figheatmapexplicitl2_fine}.
The optimal parameter configuration ($\beta= 12$, $\sigma_\text{as}
=1$) reduces the error down to $69$\% of the original error, shown
with the baseline S$_N$ result in Figures
\ref{fig:assncbrefsolutionB} and \ref{fig:assncbrefsolutionC}, respectively.

Interestingly, using a close-to-optimal parameter pair from the
coarse line-source parameter study for the lattice test case reduces
the error down to $72$\%, indicating the possibility to use different
discretizations \textit{and} test cases as surrogates.

While the worst-case error for the line-source test case was
reasonably small with a baseline normalized error of $2.23$ (coarse)
and $1.65$ (fine), this is not the case for the lattice test case
where we observe a baseline normalized error of up to $10.8$ in
Figure \ref{fig:assncbrefsolutionD}. The error is due to the enormous
amount of artificial diffusion that is added to the solution and
smears out the scalar flux unnecessarily.

\begin{landscape}
  \begin{figure}
    \centering
    \includegraphics[width=1.0\linewidth]{Chapters/Chapter3/fig/assn_cb/cb}
    \caption{Parameter study for $\sigma_{\text{as}}$ and $\beta$ on
      the grid $n_q \times n_x \times n_y = 42 \times 280 \times280$ in
      an explicit lattice calculation. For every set of parameters, we
      compute the $L^2$ error of the scalar flux $\phi$ with respect to
      a reference solution ($n_q= 1962$) on the same spatial grid. The
      number in each field of the heatmap is then the baseline
      normalized error, \ie the $L^2$ error obtained for that specific
      parameter configuration divided by the error obtained without
      artificial scattering. The optimal parameter configuration,
      highlighted in yellow, reduces the error down to 69\%.
      Close-to-optimal parameter pairs are highlighted in green. The
    colorbar ranges to a maximum value of $3.00$ to avoid color distortion. }
    \label{fig:cbnse19-109figheatmapexplicitl2_fine}
  \end{figure}
\end{landscape}
