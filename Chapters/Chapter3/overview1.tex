\begin{figure}[t]
  % S4
  \begin{subfigure}{0.24\textwidth}
    \includegraphics[width=1.0\linewidth]{Chapters/Chapter3/fig/compressing/PLOTS_1/r0S4imagesclarge.pdf}
  \end{subfigure}
  \hfill
  \begin{subfigure}{0.24\textwidth}
    \includegraphics[width=1.0\linewidth]{Chapters/Chapter3/fig/compressing/PLOTS_1/r4S4imagesclarge.pdf}
  \end{subfigure}
  \hfill
  \begin{subfigure}{0.24\textwidth}
    \includegraphics[width=1.0\linewidth]{Chapters/Chapter3/fig/compressing/PLOTS_1/r8S4imagesclarge.pdf}
  \end{subfigure}
  \hfill
  \begin{subfigure}{0.24\textwidth}
    \includegraphics[width=1.0\linewidth]{Chapters/Chapter3/fig/compressing/PLOTS_1/r10S4imagesclarge.pdf}
  \end{subfigure}

  % S6
  \begin{subfigure}{0.24\textwidth}
    \includegraphics[width=1.0\linewidth]{Chapters/Chapter3/fig/compressing/PLOTS_1/r0S6imagesclarge.pdf}
  \end{subfigure}
  \hfill
  \begin{subfigure}{0.24\textwidth}
    \includegraphics[width=1.0\linewidth]{Chapters/Chapter3/fig/compressing/PLOTS_1/r4S6imagesclarge.pdf}
  \end{subfigure}
  \hfill
  \begin{subfigure}{0.24\textwidth}
    \includegraphics[width=1.0\linewidth]{Chapters/Chapter3/fig/compressing/PLOTS_1/r8S6imagesclarge.pdf}
  \end{subfigure}
  \hfill
  \begin{subfigure}{0.24\textwidth}
    \includegraphics[width=1.0\linewidth]{Chapters/Chapter3/fig/compressing/PLOTS_1/r10S6imagesclarge.pdf}
  \end{subfigure}

  % S8
  \begin{subfigure}{0.24\textwidth}
    \includegraphics[width=1.0\linewidth]{Chapters/Chapter3/fig/compressing/PLOTS_1/r0S8imagesclarge.pdf}
  \end{subfigure}
  \hfill
  \begin{subfigure}{0.24\textwidth}
    \includegraphics[width=1.0\linewidth]{Chapters/Chapter3/fig/compressing/PLOTS_1/r4S8imagesclarge.pdf}
  \end{subfigure}
  \hfill
  \begin{subfigure}{0.24\textwidth}
    \includegraphics[width=1.0\linewidth]{Chapters/Chapter3/fig/compressing/PLOTS_1/r8S8imagesclarge.pdf}
  \end{subfigure}
  \hfill
  \begin{subfigure}{0.24\textwidth}
    \includegraphics[width=1.0\linewidth]{Chapters/Chapter3/fig/compressing/PLOTS_1/r10S8imagesclarge.pdf}
  \end{subfigure}

  % S10
  \begin{subfigure}{0.24\textwidth}
    \includegraphics[width=1.0\linewidth]{Chapters/Chapter3/fig/compressing/PLOTS_1/r0S10imagesclarge.pdf}
  \end{subfigure}
  \hfill
  \begin{subfigure}{0.24\textwidth}
    \includegraphics[width=1.0\linewidth]{Chapters/Chapter3/fig/compressing/PLOTS_1/r4S10imagesclarge.pdf}
  \end{subfigure}
  \hfill
  \begin{subfigure}{0.24\textwidth}
    \includegraphics[width=1.0\linewidth]{Chapters/Chapter3/fig/compressing/PLOTS_1/r8S10imagesclarge.pdf}
  \end{subfigure}
  \hfill
  \begin{subfigure}{0.24\textwidth}
    \includegraphics[width=1.0\linewidth]{Chapters/Chapter3/fig/compressing/PLOTS_1/r10S10imagesclarge.pdf}
  \end{subfigure}

  \caption{Scalar flux for the line-source test case  at $t=1 $
    second using the octahedron (lerp) quadrature. The colorbar is
    omitted but identical to that in Figure \ref{fig:linesourceref}.
    The spatial domain is $[-1.5 \text{ cm}, 1.5\text{ cm}]^2$ with
    $n_x=n_y = 100$. Rows vary in $n_q$ and columns vary in $\delta$,
  respectively.}
  \label{fig:rsn:ls:overview}
\end{figure}
