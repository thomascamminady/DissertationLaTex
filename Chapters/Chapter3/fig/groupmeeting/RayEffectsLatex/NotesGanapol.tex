\documentclass[]{article}
\usepackage[margin=0.5in]{geometry}
\usepackage{graphicx}
\usepackage{subcaption}
\usepackage{amsmath}
\usepackage{placeins}
\captionsetup[subfigure]{list=true, font=large, labelfont=bf,
labelformat=brace, position=top}
\title{Preliminary results: line source simulations}
\author{Thomas Camminady}

\begin{document}

\maketitle
\section*{Setup}
The setup of the line-source test case is as follows. We solve
time-dependent transport, i.e.,
\begin{align}
  \partial_t \psi(t,\bf{x},\bf{\Omega}) + \bf{\Omega} \cdot \nabla
  \psi + \sigma_s \psi
  = \sigma_s \int_{4\pi} \frac{1}{4\pi} \psi(t,\bf{x},\bf{\Omega'})
  \, d\bf{\Omega'},
\end{align}
with $\psi(0,{\bf{x}},{\bf{\Omega}}) = \psi_0(\bf{x})$ and $\sigma_s
=1$. For the exact line-source test case, $\psi_0$ is a Dirac-pulse
that is centered at $(0,0)$. In numerical S$_N$ computations, that
Dirac-pulse is resolved as a narrow Gaussian with $\sigma^2 = 0.03^2$
(similar to [1]). To distinguish these two initial conditions later
on, we will call the Gaussian-like initial condition $\psi_0^G$, as
opposed to the exact condition $\psi_0$. The exact solution for
$\phi(t,r) =\phi(t,||{\bf{x}}||_2) = \int_{4\pi}
\psi(t,\bf{x},\bf{\Omega'}) \, d\bf{\Omega'}$ at $t=1$ is due to [2]
and code by  Benjamin
Seibold\footnote{https://www.math.temple.edu/~seibold/research/starmap/
}, shown in Figure \ref{fig:ref}. Using a second order explicit S$_N$
code (similar to [3] and [4]) with $N=6$ (yielding $N^2$ ordinates)
and $N_x = N_y=200$ spatial cells, the scalar flux can be computed
and is shown in Figure \ref{fig:sn36}.

\section*{Almost exact angular flux}
To obtain the exact (up to errors in numerical integration) angular
flux, we can use the exact (up to errors in the numerical
integration) scalar flux as a known source for the transport equation
and solve for $\psi(t,\bf{x},\bf{\Omega})$ via
\begin{align}
  \psi(t,{\bf{x}},{\bf{\Omega}}) = e^{-\sigma_s \, t}\psi_0({\bf{x}}
  - t {\bf{\Omega}})
  + \int_0^t e^{-\sigma_s(t-\tau)} \phi(t-\tau, ||{\bf{x} }- t
  {\bf{\Omega}}||_2) \, d\tau.
\end{align}
Numerical, however, we need to solve
\begin{align}
  \psi^G(t,{\bf{x}},{\bf{\Omega}}) = \underbrace{e^{-\sigma_s \,
  t}\psi^G_0({\bf{x}} - t {\bf{\Omega}})}_{=:A(t,{\bf{x}},{\bf{\Omega}})}
  + \underbrace{\int_0^t e^{-\sigma_s(t-\tau)} \phi(t-\tau, ||{\bf{x}
  }- t {\bf{\Omega}}||_2) \, d\tau}_{=:B(t,{\bf{x}},{\bf{\Omega}})}.
\end{align}
The integral $B$ was computationally expensive (and deserves
double-checking and more care in general) and is therefore not solved
with high accuracy. The scalar fluxes that result from $A$ and $B$,
i.e., $\int_{4\pi} A \, d\bf{\Omega'}$ and $\int_{4\pi} B \,
d\bf{\Omega'}$, are computed for $N_x=N_y=100$ on $[0,1.5]^2$ and
then duplicated to the other four quadrants. The numerical quadrature
is a tensorized quadrature using $2\cdot 6$ angles equidistantly
spaced in the azimuthal angle and $6 /2$ ordinates for the polar
angle, chosen as Gauss-Legendre roots.

The exact (up to errors in the numerical integration) angular flux is
then integrated using the numerical quadrature, denoted by $\langle
\psi^G(1,{\bf{x}},{\bf{\Omega'}}) \rangle$, and shown in Figure
\ref{fig:ContributionCombined} and again compared to the S$_6$
computation in Figure \ref{fig:sn36again}. In addition, Figures
\ref{fig:IC} and \ref{fig:RHS} show the contributions of $\langle
A(1,{\bf{x}},{\bf{\Omega'}}) \rangle$ and $\langle
B(1,{\bf{x}},{\bf{\Omega'}})\rangle$, respectively.
\label{sec:ls}

\begin{figure}[h!]
  \begin{subfigure}[c]{0.43\textwidth}
    \includegraphics[width=1.0\linewidth]{../REF}
    \caption{Scalar flux reference solution due to [2].}
    \label{fig:ref}
  \end{subfigure}
  \hfill
  \begin{subfigure}[c]{0.43\textwidth}
    \includegraphics[width=1.0\linewidth]{../SN36}
    \caption{S$_N$ computation using $N=6$, i.e., $36$ ordinates.}
    \label{fig:sn36}
  \end{subfigure}

  \begin{subfigure}[c]{0.43\textwidth}
    \includegraphics[width=1.0\linewidth]{../ContributionCombined}
    \caption{$\langle \psi^G(1,{\bf{x}},{\bf{\Omega'}})\rangle$.}
    \label{fig:ContributionCombined}
  \end{subfigure}
  \hfill
  \begin{subfigure}[c]{0.43\textwidth}
    \includegraphics[width=1.0\linewidth]{../SN36}
    \caption{S$_N$ computation using $N=6$, i.e., $36$ ordinates.}
    \label{fig:sn36again}
  \end{subfigure}
  \begin{subfigure}[c]{0.43\textwidth}
    \includegraphics[width=1.0\linewidth]{../ContributionIC}
    \caption{$\langle A(1,{\bf{x}},{\bf{\Omega'}})\rangle $}
    \label{fig:IC}
  \end{subfigure}\hfill
  \begin{subfigure}[c]{0.43\textwidth}
    \includegraphics[width=1.0\linewidth]{../ContributionRHS}
    \caption{$\langle B(1,{\bf{x}},{\bf{\Omega'}}) \rangle$}
    \label{fig:RHS}
  \end{subfigure}
  \caption{Line-source simulation results.}
\end{figure}

\FloatBarrier
\section*{Discussion}
There are remarks, observations, and questions that follow.
\subsection*{Some remarks}
\begin{enumerate}
  \item $\int_{4\pi}\psi^G(t,{\bf{x}},{\bf{\Omega}}) \,
    d{\bf{\Omega'}} \not = \phi(t,\bf{x})$ since term $A$ does not
    use the exact initial conditions. However, evaluating $A$ with
    the exact initial condition does not seem possible numerically.
    Additionally, the S$_N$ computations also use the approximated
    initial conditions.
  \item The numerical integration of $B$ is performed with an
    absolute tolerance of $10^{-2}$ but (very) occasionally, the
    integral did not seem to convert. As a consequence the maximum
    number of function evaluations during the numerical integration
    was set to $1000$. Looking at Figure \ref{fig:RHS}, the results
    do however look reasonable, at least quality wise.
\end{enumerate}

\subsection*{Some observations}

\begin{enumerate}
  \item Since the angular flux is computed (almost) exactly, the ray
    effects in Figure \ref{fig:ContributionCombined} are exclusively
    due to the numerical integration of the scalar flux. This scalar
    flux looks very similar to the S$_6$ simulation results in Figure
    \ref{fig:sn36again}.
  \item Figure \ref{fig:IC} pinpoints the origin of the large ray
    effects in the S$_6$ simulations. The initial condition only
    contributes to very few spatial cells, based upon the choice of
    the quadrature.
\end{enumerate}

\subsection*{Some questions}
\begin{enumerate}
  \item Is the only factor that contributes to the presence of ray
    effects the incorrect integration of the scalar flux? Since we
    compute the angular flux exactly, this seems like the only error
    being made (except for numerical integration).
  \item Is it then fair to use the scalar flux as an error metric to
    compare S$_N$ computations to the reference scalar flux? Even if
    the angular flux is solved exactly, the error in the scalar flux
    can be non-zero. And can we ask the S$_N$ method for more than
    solving the angular flux exactly?
  \item We developed ray effect mitigation techniques. But since we
    saw that the exact angular flux still yields ray effects, aren't
    our mitigation techniques implying that we purposefully introduce
    an error in the angular flux just to get a better scalar flux?
  \item Should ray effects rather be tackled via a postprocessing step?
\end{enumerate}

\FloatBarrier

\section*{Bibliography}
\begin{itemize}
  \item [[1]] McClarren, Ryan G., and Cory D. Hauck. "Robust and
    accurate filtered spherical harmonics expansions for radiative
    transfer." Journal of Computational Physics 229.16 (2010): 5597-5614.

  \item [[2]] Ganapol, B. D., et al. Homogeneous infinite media
    time-dependent analytical benchmarks. No. LA-UR-01-1854. Los
    Alamos National Laboratory, 2001.

  \item [[3]] Camminady, Thomas, et al. "Ray effect mitigation for
    the discrete ordinates method through quadrature rotation."
    Journal of Computational Physics 382 (2019): 105-123.

  \item [[4]] Frank, Martin, et al. "Ray Effect Mitigation for the
    Discrete Ordinates Method Using Artificial Scattering." Nuclear
    Science and Engineering (2020): 1-18.
\end{itemize}
\end{document}
