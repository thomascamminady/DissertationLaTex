\part{Concluding remarks}

\chapter{Summary}
As already hinted in this document's name, we have covered two
topics---classical and non-classical transport---with three different
perspectives---theory, modeling, and numerics.

The initial microscopic description of transport in Chapter
\ref{sec:transportequations} was not only crucial in understanding
the origin of kinetic theory, it also highlighted how different
modeling assumptions lead to different governing equations. This has
manifested itself throughout Part \ref{part:2}: Changing the
scatterers' positions from completely random to a lattice structure
fundamentally changed the dynamics of particle transport, requiring a
completely new set of governing equations.

Moreover, the particle billiard game of Chapter
\ref{sec:transportequations} directly motivated the Monte Carlo
method as one member of the class of numerical algorithms to solve
the transport equation that were presented in Chapter \ref{chap:2}.

We discussed the crucial role of spherical quadratures for the
discrete ordinates method and presented quadratures that are based on
the triangulation of Platonic solids, resulting in a highly uniform
distribution of quadrature points on the unit sphere.

The underlying connectivity of these quadratures was imperative to
the successful realization of the rS$_N$ method. Together with
as-S$_N$, these two ray effect mitigation techniques drastically
reduced the spurious oscillations that are inherent to the discrete
ordinates method.

We saw in Chapter \ref{chap:3} that ray effects are a result of
incorrect computation of the spherical integral. The rS$_N$ and
as-S$_N$ method both circumvent this problem to an extend where it is
possible to (i) either gain a significant boost in accuracy while
keeping the runtime constant, or (ii) reduce the runtime while
keeping the solution quality virtually unaltered.

The introduction of non-classical particle transport in Chapter
\ref{chap:nonclassicaltransportequations}
allowed us to tackle questions around non-classical transport in
heterogeneous materials.  Throughout Chapter
\ref{chap:heterogeneousnonclassical}, we provided different ansätze
to model the behavior of particles that cross material interfaces;
each obeying different theoretical requirements. The existing notion
of correlated and uncorrelated particles was also extended to include
heterogeneities.

Let us now discuss the main results of the different chapters in more detail.
Acknowledging the introductory nature of Chapter
\ref{sec:transportequations}, we start with Chapter \ref{chap:2}.

{\textbf{{Numerical solution methods}}

  Three different solution methods for transport equations, as well
  as their main ingredients were presented and analyzed.

  For the Monte Carlo method, we discussed ways to efficiently sample
  random numbers, \eg via inverse transform sampling or rejection sampling.
  Moreover, we discussed the advantages and disadvantages of
  interpreting Monte Carlo as a numerical integrator that can be used
  to approximately solve the transport equation.

  % Interpreting Monte Carlo as a numerical integrator allowed us to
  % see how it can be used to approximate quantities of interests of
  % transport equations such as the energy doses. Its main
  % disadvantage is the slow convergence order of $N^{-1/2}$,
  % although it can be efficiently parallelized.

  The main focus was on the discrete ordinates method where we
  started with a discussion of quadrature sets. The octahedron- and
  icosahedron-based quadratures (Camminady, Frank, and Kusch
  \cite{camminady2019highly}) have three desirable properties: (i)
  Arbitrary many quadrature points with strictly positive quadrature
  weights can be generated at low costs; (ii) the ratio between the
  maximal and minimal quadrature weight can be kept at around $2.0$
  and below $1.5$ for the icosahedron \textit{lerp} and
  \textit{slerp} versions, respectively; and (iii) the resulting
  triangulation can be used to easily interpolate function values at
  points that are not included in the original quadrature set.

  For transport sweeps as one of the ways to solve the system of
  S$_N$ equations, we have proven that it is always possible to find
  an ordering that allows us to march through all spatial cells while
  considering the involved dependencies.

  Representing the class of moment methods, we have seen that the
  P$_N$ method can be used to obtain symmetrical solutions, albeit at
  the cost of oscillations that can render the solution negative.

  {\textbf{Ray effects and their mitigation}}

  With the help of a simple experiment, we have seen that ray effects
  are ultimately not the result of restricting transport to a fixed
  set of ordinates. Even in cases where we can solve for the angular
  flux $\psi(t,\bx,\bOmega)$ exactly (or at least up to machine
  precision), ray effects in the scalar flux $\phi(t,\bx)$ are
  omnipresent. Instead, the occurrence of these numerical artifacts
  can be attributed to the inexact approximation
  \begin{align*}
    \int_{\mathbb{S}^2}  \psi(t,\bx,\bOmega') \, d\bOmega' \approx
    \sum_{q'=1}^{n_q} w_{q'} \, \psi(t,\bx,\bOmega_{q'}).
  \end{align*}
  Both the presented rS$_N$ method (Camminady, Kusch, Frank, and
  Küpper \cite{camminady2019ray}) and the as-S$_N$ method (Frank,
  Kusch, Camminady, and Hauck \cite{frank2020ray}) mitigated this
  problem by making the angular flux easier to integrate,  once via
  additional angular diffusion and once with the help of artificial scattering.

  By adding a rotation-and-interpolation step to the standard S$_N$
  method, the rS$_N$ method significantly reduced ray effects in the
  line-source problem and the lattice test case. The solution quality
  of the scalar flux that is computed with the rS$_N$ method is on
  par with that of the S$_N$ method with up to three times the number
  of quadrature points. Adding the rotation-and-interpolation step
  only marginally increases the runtime of the S$_N$ method (by up to
  approximately 10$\%$ in our implementation).
  A drawback of the rS$_N$ method is the change of ordinates between
  two consecutive time steps. As a result, the order of transport
  sweeps needs to be recomputed which might create a bottleneck in
  highly optimized, many-core implementations on large-scale
  computers. However, forth-and-back rotations with two interpolation
  steps have the potential to overcome this problem by keeping the
  quadrature set fixed.

  The as-S$_N$ method altered the transport equation by including an
  additional scattering term with scattering kernel
  \begin{align*}
    s_\varepsilon(\mu) = \frac{2}{\sqrt{\pi} \,  \varepsilon \,
    \text{Erf}\left(\frac{2}{\varepsilon}\right)}
    \exp\left(-\frac{(1-\mu)^2}{\varepsilon^2}\right).
  \end{align*}
  Asymptotically, this forward-peaked type of scattering effectively
  behaves like a Fokker-Planck operator.
  Furthermore, the choice $\varepsilon = \beta / n_q$ ensured that
  the added effect vanishes in the limit of $n_q\rightarrow \infty$
  as we recover the original S$_N$ equations. Together with
  $\sigma_{as}$, the as-S$_N$ method then requires the user to select
  two parameters. For the line-source test case, we have seen an
  error reduction in the scalar flux of a factor four. Transferring
  the same pair of optimal parameters to the lattice test case on a
  significantly finer grid (in space, angle, and time) still reduced
  the error down to $72\%$ of the original error. This indicated that
  both different test cases as well as different discretizations can
  be used as a surrogate model with which optimal parameters can be
  derived at low costs before applying them to the actual test case of interest.

  Since the as-S$_N$ method does not change the initial ordinates, it
  is potentially better suited for optimized codes than the rS$_N$ method.

  {\textbf{Non-classical transport equations}

    This chapter introduced the non-classical transport equation in
    its forward and backward formulation.
    Several examples and numerical experiments revealed discrepancies
    between the
    the behavior predicted by classical transport and the
    observations that are made.
    A transformation between both formulations is possible (Larsen,
    Frank, and Camminady \cite{larsen2017equivalence}), and we can
    equivalently encode the non-classical dynamics in $s$-dependent
    cross sections or  non-exponential path length distributions.

    Furthermore, the existing concept of \textit{correlated} and
    \textit{uncorrelated} particles was presented, indicating the
    necessity to differentiate between particles that have already
    collided with scatterers from those that have not.

    {\textbf{Non-classical transport in heterogeneous materials}

      Different ways to model material heterogeneities were
      introduced and juxtaposed.
      Depending on the model, we were able to argue for a
      preservation or a reset of the particles' memory. For both
      cases, we extended the non-classical cross sections and path
      length distributions to include material heterogeneities
      (Camminady, Frank, and Larsen
      \cite{camminady2017nonclassical}). In the more complicated
      setting of \textit{true} heterogeneity in which the particles'
      memory should be preserved, the ODE system
      \begin{align*}
        \begin{cases}
          P(0,\bx,\bOmega) &= 0, \\
          \frac{d}{ds}P(s,\bx,\bOmega) &= p_{\bx(s)}(
          P_{\bx(s)}^{-1}\left ( P(s,\bx,\bOmega) \right))
        \end{cases}
        \label{eq:ode}
      \end{align*}
      had to be solved in order to generate samples of the path
      length distribution. Since this is costly, we provided
      efficient sampling strategies, applicable to (reasonably)
      simplified scenarios. Additionally, we discussed a series of
      sanity checks to investigate the validity of the proposed distribution.

      Lastly, the notion of correlated and uncorrelated particles was
      extended to include material heterogeneities as well. Particles
      that cross a material interface enter the new domain as
      uncorrelated particles---ultimately, this was the reason to not
      adopt the notion of \textit{collided} and \textit{uncollided} particles.
      As a result, two species of particles are described by two
      coupled non-classical transport equations.

      \chapter{Outlook}

      The presented ray effect mitigation techniques are implemented
      in a Julia research code that cannot be used to make
      predictions about the effectiveness of both methods on
      large-scale computing systems. An integration of both methods
      into production-ready software is therefore necessary to
      investigate both methods on a more quantitative level---rather
      than the more qualitative studies included in this thesis.

      Complexity that arises from adding energy-dependency,
      unstructured meshes, or high-order time-integration  is another
      hurdle that needs to be tackled as a logical next step. Several
      improvements, especially for the as-S$_N$ method are apparent.
      (i) A more granular tuning of the amount of artificial
      scattering is desirable. The artificial scattering strength
      $\sigma_{as}$ can become space-dependent. Moreover, work that
      researches whether it is sufficient to only add artificial
      scattering temporarily is underway. (ii) Whether it is possible
      to transfer optimal parameter sets from low-cost surrogate
      models to the actual high-cost simulation needs to be tested
      more broadly. To promote the as-S$_N$ method as a reliable ray
      effect mitigation technique, it is necessary to provide
      guidance on how to choose the involved parameters optimally and
      efficiently. (iii) For the rS$_N$ method, changing ordinates in
      every time-step might  impact the sweeping procedure too
      severely to prove useful. If this is the case, versions of the
      rS$_N$ method that do not suffer from this problem need to be
      investigated more rigorously. For example, instead of rotating
      once per time step, it is possible to perform two
      rotations---once forth, once back---in every time step with
      half the rotation magnitude. This has already been discussed,
      but not further analyzed.

      Current research on angular adaptivity for the S$_N$ method has
      shown promising results \cite{DARGAVILLE2020109124}. The
      angular quadrature set is refined or coarsened in different
      areas of the spatial domain and at different time steps. The
      filtered spherical harmonics method is used as an error
      indicator, telling the algorithm where to refine or coarsen the
      angular mesh. This error indicator can, in theory, be used for
      the as-S$_N$ or rS$_N$ method as well, indicating regions that
      require artificial scattering or rotation, respectively.

      Compared to classical transport, non-classical transport theory
      is still in its infancy.
      Here too, quantitative studies are necessary to understand the
      effect that $s$-dependent cross sections or non-exponential
      path length distributions have  on real-world problems. So far,
      mostly the Monte Carlo method has been explored as a way to
      solve the non-classical transport equation. However,
      non-classical time-independent transport is structurally
      similar to time-dependent classical transport. Whereas time
      only increases, the path length $s$ gets reset after every
      collision (in the backward formulation).
      Nevertheless, if we rewrite the backward formulation as
      \begin{align*}
        \begin{cases}
          \partial_s \psi(s,\bx,\bOmega) +\bOmega \cdot \nabla_\bx
          \psi(s,\bx,\bOmega) + \sigma_t(s) \psi(s,\bx,\bOmega)  = 0, \\
          \psi(0,\bx,\bOmega) =  \int_0^{l(\bx,\bOmega)}
          \int_{\mathbb{S}^2} c \, \sigma_t(s') s(\bOmega' \cdot
          \bOmega) \, \psi(s',\bx,\bOmega')\, d\bOmega' \, ds'+  Q(\bx),
        \end{cases}
      \end{align*}
      we can add indices $l$ and $l-1$ to the equation to obtain
      \begin{align*}
        \begin{cases}
          \partial_s \psi^{l}(s,\bx,\bOmega) +\bOmega \cdot
          \nabla_\bx \psi^l(s,\bx,\bOmega) + \sigma_t(s)
          \psi^l(s,\bx,\bOmega)  = 0, \\
          \psi^l(0,\bx,\bOmega) =  \int_0^{l(\bx,\bOmega)}
          \int_{\mathbb{S}^2} c \, \sigma_t(s') s(\bOmega' \cdot
          \bOmega) \, \psi^{l-1}(s',\bx,\bOmega')\, d\bOmega' \, ds'.
        \end{cases}
      \end{align*}
      With $\psi^{0}(s,\bx,\bOmega) \equiv Q(\bx)$, we can solve this
      system repeatedly with a standard S$_N$ solver. In this
      formulation, $\psi^{l+1}$ relates to those particles that have
      scattered at most $l$ times. Assuming convergence due to
      sufficient absorption, $\psi^{L}(s,\bx,\bOmega) \approx
      \psi(s,\bx,\bOmega)$ for a sufficiently large $L$.
      True when moving from classical to non-classical transport, the
      added complexity of the distinction between correlated and
      uncorrelated particles needs to be quantified, too.

      Next is the topic of heterogeneities. Ultimately, the proposed
      distributions need to be validated by means of real-world,
      physical experiments or computer simulations. Actually
      simulating trajectories of particles throughout a heterogeneous
      field of non-classically arranged scatterers and tallying those
      that do cross a domain interface is computationally expensive,
      if a sufficient statistics is to be achieved. This procedure
      gets more complicated when acknowledging the fact that we have
      limited knowledge about non-classical arrangements of
      scatterers. For example, algebraic decay of the path length
      distribution for the periodic Lorentz gas is observed only asymptotically.
      However, it does seem as if this is problem is conquerable by
      efficient implementations of the particle billiard games
      mentioned throughout this thesis.

      The computer graphics community has been using non-exponential
      transport as a tool that enhances the artistic degree of
      freedom when creating renders
      \cite{bitterli2018radiative,jarabo2018radiative}.
      The $s$-dependent cross sections are chosen freely, without
      considering whether they correspond to a certain arrangement of obstacle.
      Understanding which cross sections (and path length
      distributions) are the result of actual arrangements of
      scatterers might prove fruitful beyond mathematical curiosity:
      It enables us to better understand the physical world around us
      and make simulations and renders  more accurate.

      %Finally, all added complexity has to be justified by
      % measurable improvements of a model's predictions.
      %Finding more real-world applications that benefit from (and
      % hence justify) non-classical transport is therefore paramount
      % to the theory's success and acceptance in the scientific community.
