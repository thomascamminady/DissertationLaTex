\chapter{Spherical harmonics}
\todo{add references}The spherical harmonics are a complete set of orthogonal functions on the unit sphere. 
There exist different definitions of the spherical harmonics that vary in the respective normalization constants. 
One possible definition is 
\begin{align}
	Y_n^m(\theta,\phi) := \sqrt{\frac{2n+1}{4\pi} \, \frac{(n-m)!}{(n+m)!}}e^{i m \theta}P_n^m\left(\cos(\phi)\right),
\end{align}
with $|m|\leq n$, the azimuthal angle $\theta\in [0,2\pi]$, the polar angle $\phi\in [0,\pi]$, and $P_n^m$ as the associated Legendre functions.
This choices ensures
\begin{align}
	\int_0^{2\pi} \int_0^\pi  Y_n^m(\theta,\phi) \overline{Y_{n'}^{m'} (\theta,\phi)} \, \sin(\phi)\,d\phi \, d\theta
	%&\int_{\mathcal{S}^2}Y_n^m  \, \overline{Y_{n'}^{m'}} \, d\bOmega 
	= \delta_{n,n'}\delta_{m,m'},
\end{align}
\ie orthonormality with respect to integration over the unit sphere.
The real parts of the spherical harmonics up to $m=3$ are visualized in Fig.~\ref{fig:sphericalharmonics}.

The spherical harmonics relate to the Legendre polynomials via the \textit{addition theorem}. 
For $\bOmega,\bOmega'\in \mathbb{S}^2$ and $l\in \Ngeqz$ the equality
\begin{align}
	P_n(\bOmega\cdot \bOmega') = \frac{4\pi}{2n+1} \sum_{|m|\leq n}Y_n^m(\bOmega) \overline{Y_n^m(\bOmega')}
\end{align}
allows to express the Legendre polynomials in form of the spherical harmonics, given an argument that is the dot product of two vectors living on the unit sphere. 
Since the scattering kernel is frequently evaluated at the dot product between the in- and out-scattering directions, the \textit{addition theorem} allows to write
\begin{align}
	s(\bOmega \cdot \bOmega') = \sum_{n=0}^\infty s_n \sum_{|m|\leq n}Y_n^m(\bOmega) \overline{Y_n^m(\bOmega')},
\end{align}
where $s_n$ are expansion coefficients.
\label{appendix:sphericalharmonics}
\begin{landscape}
	\begin{figure}
		\centering
		\includegraphics[width=0.9\linewidth]{Chapters/Appendix1/figs/sphericalharmonics/sphericalharmonics}
		\caption{Real parts of the spherical harmonics for different orders.}
		\label{fig:sphericalharmonics}
	\end{figure}
\end{landscape}


