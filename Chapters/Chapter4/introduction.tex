
\section{Introduction}
Classical transport theory describes a memoryless process:
The distance that a particle has already traveled since its last
collision does not alter the probability of undergoing a collision in
the future.
While this theory correctly models and predicts transport in various
regimes, experimental and theoretical work has shown its limitations.
This initiated the quest for a more general formulation of transport
in the recent decade---non-classical transport theories.

On the experimental side, results in atmospheric physics demonstrate
the presence of a memory for light transport in clouds
\cite{veitel1998geometrical,kostinski2001scale,kostinski2001extinction,borovoi2002extinction,shaw2002super,shaw2002towards}.
Since the positions of water droplets in clouds are positively
correlated, the presence of water droplets increases the likelihood
of finding more droplets in the vicinal neighborhood. Conversely, a
photon that has already traveled a large distance since its previous
collision is likely to further traverse the cloud unhinderedly.

On the theoretical side, we start by mentioning the\textit{ periodic
Lorentz gas }
\cite{bourgain1998distribution,golse2008periodic,caglioti2008boltzmann,marklof2008boltzmann,marklof2014free,marklof2011periodic,marklof2014power,marklof2015generalized}.
The underlying motivation was to derive a transport equation for
particles that interact with a highly correlated medium.
Obstacles are placed on a two-dimensional lattice with certain radii
and certain distances apart from another and particles undergo
elastic collisions with the obstacles.
Shrinking the obstacles' radii and distances carefully yields a
transport equation that augments the phase-space of the classical
transport equation by an additional variable.
This variable is interpreted as \textit{the distance to the next collision}.
Independently, a generalized transport theory was hypothesized where
the additional variable was interpreted as the \textit{distance since
the last event} (be it birth or collision)
\cite{larsen2011generalized,vasques2014non,vasques2014non2}.
Originated in the nuclear engineering  community, this generalized
theory proofed to be a useful tool for e.g. transport in
mixtures\cite{vasques2014accuracy} or pebble bed reactors
\cite{vasques2014non2}.

More recently, non-classical transport has found its way into the
computer graphics community.
Initial work suggest that non-classical transport is a useful tool
for rendering images or animated movies
\cite{bitterli2018radiative,d2018reciprocal,d2019reciprocal,deon2019reciprocal3,jarabo2018radiative}.
Images appear more realistic and a larger artistic flexibility can be achieved.

Though already addressed in its infancy, the computer graphics
community has again highlighted certain limitations of non-classical
transport that have to be lifted for it to find broad acceptance and usability.
In this work, we are going to focus on one of these limitations: The
lack of a description for non-classical transport in heterogeneous
media where each medium dictates different transport behavior.
Before hypothesizing two sets of heterogeneous non-classical
transport equations in Section \ref{sec:transporteqs}, we elaborate
on the necessity of distinguishing two species of particles---those
that are correlated with the medium and those that are
uncorrelated---in Section \ref{sec:corr}.
Solution methods for the resulting equations are sketched in Section
\ref{sec:Solution methods}.
We end with a discussion in Section \ref{sec:discuss}.
