\documentclass[letterpaper]{template/mandc2019}
\input{commands.tex}
\input{packages.tex}

\title{Heterogeneous non-classical transport}
\author{
  Thomas Camminady\footnote{Corresponding author},\, Martin Frank\\
  \texttt{firstname.lastname@kit.edu}
}

\begin{document}
\maketitle
\justify
\begin{abstract}
\end{abstract}

\FloatBarrier
\section{Introduction}
\section{A dictionary for non-classical transport}
Non-classical transport is inherently more subtle than classical
transport; these subtleties  are amplified in the case of
heterogeneous non-classical transport. We require a clear definition
of words and concepts that are used interchangeably in classical
transport but have to be treated more carefully in the non-classical
counterpart. Consequently, this section can be considered a
dictionary that gathers notions and concepts that need further clarification.

\subsection{Heterogeneity}
Already the word \textit{heterogeneous} requires a precise definition
which is not trivial.
Consider Figures \ref{fig:hetnonclassical1colors},
\ref{fig:2hetnonclassical1colors2}, and \ref{fig:2hetnonclassical1colors}.
All three examples show a heterogeneous material.
However, the type of heterogeneity is different in each single one of them.
In Figure \ref{fig:hetnonclassical1colors}, we see two instances of
the periodic Lorentz gas \cite{marklof2010distribution}.
They differ in the value of $\sigma_{\text{left}} =
N_\text{left}\cdot r_\text{left}$ and $\sigma_{\text{right}} =
N_\text{right}\cdot r_\text{right}$, with $N_{\text{side}}$ the
number of obstacles and $r_{\text{side}}$ their respective radii on either side.
The underlying arrangement of obstacles is the same.
We will refer to the situation depicted in Figure
\ref{fig:hetnonclassical1colors} as a \textit{false} heterogeneous
material since our focus lies on the discussion of the situation
depicted in Figure \ref{fig:2hetnonclassical1colors2} and
\ref{fig:2hetnonclassical1colors}, respectively.

Both of them are \textit{true} heterogeneous in the sense that the
arrangement of obstacles differ not only in different sized obstacles
or their respective number, but are arranged in an inherently
different way---randomly on the left and aligned with a lattice on the right.

We do not further distinguish between the situations in Figure
\ref{fig:2hetnonclassical1colors2} and
\ref{fig:2hetnonclassical1colors} since in both cases, the left
arrangement of obstacles is not a scaled version of the right
arrangement of obstacles in any way.

The framework of \textit{true} heterogeneity is more general than the
one of \textit{false} heterogeneity.
Moreover, \textit{false} heterogeneity is included in \textit{true}
heterogeneity.

\begin{figure}
  \centering
  \begin{subfigure}[b]{0.32\textwidth}
    \centering
    \includegraphics[width=1.0\linewidth]{figures/false_hetero}
    \caption{The material is heterogeneous due to a variation in density.}
    \label{fig:hetnonclassical1colors}
  \end{subfigure}
  \hfill
  \begin{subfigure}[b]{0.32\textwidth}
    \centering
    \includegraphics[width=1.0\linewidth]{figures/true_hetero1}
    \caption{The material is heterogeneous due to the different materials.}
    \label{fig:2hetnonclassical1colors2}
  \end{subfigure}
  \hfill
  \begin{subfigure}[b]{0.32\textwidth}
    \centering
    \includegraphics[width=1.0\linewidth]{figures/true_hetero2}
    \caption{The material changes in material and varies in density.}
    \label{fig:2hetnonclassical1colors}
  \end{subfigure}
  \caption{Three types of heterogeneity.}
  \label{fig:three graphs}
\end{figure}

\subsection{Correlation}

Additionally to the definition of heterogeneity, the discussion of
correlated and uncorrelated materials is important.
Correlation in this subsection refers to the correlation between two
parts of a heterogeneous material---i.e. inter-correlation---not to
the correlation of obstacles within each of these materials---i.e.
intra-correlation. This distinction has recently been investigated in
the computer graphics community as well \cite{jarabo2018radiative}.

Let us start with the periodic Lorentz gas---where obstacles are
clearly inter-correlated---and perform the Boltzmann-Grad limit to
obtain a distribution function $q_L(s)$ (the $L$ stands for lattice)
as illustrated in Figure \ref{figure:simplelattice}.
Next, we color both halves of the material differently and pretend to
be in a heterogeneous situation, shown in Figure
\ref{figure:simplelattice2colors}.
In each of the two halves the respective Boltzmann-Grad limit for
that material is possible and yields the same distribution $q_L(s)$.
If we now combine the two distributions to find a combined
distribution,  we have to end up with the original distribution
$q_L(s)$ since we did not change anything but the color of the obstacles.
Thus in this case, two materials with the same underlying
path-lengths distribution can be combined into one material with the
same distribution.

Alternatively, we could also randomly rotate the lattice of one of
the two halves, seen in Figure \ref{figure:simplelattice2colors}.
Again, each individual lattice yields the same distribution in the
Boltzmann-Grad limit if performed separately, since the orientation
of the lattice vanishes in the limit.
Despite the lack of orientation of the lattice in the limit, there
seems to be no compelling evidence for why the combined distribution
should be equivalent to the one obtained in the previous case.
Instead the resulting distribution $p_L(s)$ might differ from $q_L(s)$.
Instead of rotating one side of the lattice, we might shift it by
half the obstacle distance in vertical direction. This places
obstacles in the corridors which allowed particles to move from the
left to the right and consequently has to have some effect on the way
particles move.

Moreover, another problem arises when acknowledging this situation:
On the mesoscopic level we can not distinguish between the two
situations since we exclusively know the material by its distribution function.
If the combined distribution function resulting from the thought
experiment of Figure \ref{figure:simplelattice2colors} is indeed
different from the one in Figure \ref{figure:simplelattice}, then
describing a material solely by its distribution function is inaccurate.

We only discussed the correlation between materials with the periodic
Lorentz gas as an example.
However, similar problems may arise in any arrangement of obstacles
that are not independent of another. One other example would be an
obstacle arrangement where a certain minimal distance is present
between obstacles. This minimal distance possibly falls short at the
interface when combining two separate obstacle arrangements.

\begin{figure}[h]
  \centering
  \includegraphics[width=0.99\linewidth]{figures/large_small_ext_single}
  \caption{Boltzmann-Grad limit in a homogeneous and non-classical
    material. This situation and the one in Figure
  \ref{figure:simplelattice2colors} have to be equivalent.}
  \label{figure:simplelattice}
\end{figure}

\begin{figure}[h]
  \centering
  \includegraphics[width=0.99\linewidth]{figures/ext_correlated}
  \caption{Boltzmann-Grad limit in a heterogeneous and non-classical
    material. Both sides of the material are correlated due to the
  alignment of lattices}
  \label{figure:simplelattice2colors}
\end{figure}

\begin{figure}[h]
  \centering
  \includegraphics[width=0.99\linewidth]{figures/ext_uncorrelated}
  \caption{Boltzmann-Grad limit in a heterogeneous and non-classical
    material. Both sides of the material are uncorrelated. This case
    does not have to be equivalent to the case considered in Figure
  \ref{figure:simplelattice}.}
\end{figure}

\subsection{Path-lengths distribution}
\cite{bitterli2018radiative, d2018reciprocal}
\section{Path-lengths distribution for the heterogeneous non-classical case}

\section{A negative result for reciprocity in heterogeneous
non-classical transport}

\section{Discussion}

\newpage
\bibliographystyle{siamplain}
\bibliography{bib.bib}

\end{document}
