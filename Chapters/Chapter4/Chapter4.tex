\part{Non-classical transport}\label{part:2}
\chapter{Non-classical transport equations}
\label{chap:nonclassicaltransportequations}
Not all particles are created equal, for some engage in billiard
games that violate Section \ref{sec:transportequations}'s fundamental rule:
Background obstacles ought to be uncorrelated.
If they are not---and we will go through various examples where this
rule is  violated---\textit{classical} transport equations, \ie those
that we have encountered so far, model the underlying transport
phenomena incorrectly.
Instead, so-called \textit{non-classical} transport equations emerge,
augmenting the phase space of classical transport equations by an
additional variable.
This extra variable can be interpreted both as the distance to
collision or the distance since collision, depending on the precise
formulation of the non-classical transport equation.

Different from classical transport in Part \ref{part:1},
non-classical transport, to which Part \ref{part:2} of this thesis is
devoted to, is much more novel in its application to radiative transfer.
This section motivates the necessity of non-classical transport by
three applications before delving into further modeling steps that
ultimately lead to various ways to generalize classical transport,
one of which is the  \textit{generalized linear Boltzmann equation} (GLBE).
Throughout Part \ref{part:2}, we will discuss and contribute to
so-far unsolved questions regarding the modeling of non-classical
transport in the presence of heterogeneities.


Non-classical transport dates back to the late 2000s and early 2010s where
it was independently formalized by Caglioti and Golse
\cite{caglioti2010boltzmann, golse2006periodic} and Larsen and
Vasques \cite{larsen2011generalized}.
Caglioti and Golse discuss the Boltzmann-Grad limit for the periodic
Lorentz gas where scatterers are arranged on a lattice rather than at
random positions.
Larsen and Vasques provide an initial study of non-classical
transport, together with the derivation of the respective asymptotic
diffusion limit. Numerical test cases show the superiority of
non-classical transport equations when modeling transport in a pebble
bed reactor.
Davis and Xu \cite{doi:10.1080/23324309.2014.978083} discussed linear
transport in spatially correlated stochastic media in 2014.


Noteworthy, too, are the recent advances in the computer graphics
community. Non-classical transport has been used to enhance the
quality and realism of rendered images
\cite{bitterli2018radiative,d2018reciprocal, jarabo2018radiative}.

We postpone an extensive discussion of the relevant equations of
non-classical transport for now and start off with the first of three
examples that demonstrate the violation of classical transport.
These examples are deliberately chosen to cover different
applications and fields of research to showcase the universality of
non-classical transport.




\begin{exmp}[Radiation in clouds]
  According to NASA\footnote{The quote is taken from
    \url{https://www.nasa.gov/image-feature/cloudy-earth}, which itself
  paraphrases the work published by King et al. \cite{6422379}.},
  \enquote{about 67 percent of Earth’s surface is typically covered
  by clouds,} illustrated in Figure \ref{fig:cloudfraction}.
  Consequently, clouds influence the climate in a variety of ways,
  one of which is through the interaction with solar radiation.
  In the mid to late 1990s, however, observations of both under- and
  overestimation of the solar radiation were reported in the
  literature
  \cite{cess1995absorption,ramanathan1995warm,pilewskie1995direct,wild1995validation,arking1996absorption},
  ranging from $-10\sfrac{\text{W}}{\text{m}^2}$ to
  $+30\sfrac{\text{W}}{\text{m}^2}$, respectively.
  With the help of experimental data, Pfeilsticker
  \cite{pfeilsticker1998first} discovered that the geometric path
  length distribution of photons is Lévy distributed ($\sim
  s^{-\gamma-1}$ for large $s$) with Lévy index $1\leq \gamma \leq
  2$. This is in stark contrast to Chapter
  \ref{sec:transportequations}, were the path length distribution for
  the billiard game was exponentially decaying, not algebraically.

  When model and data do not match, it is often a good idea to revise
  the model (rather than, say, the data). Indeed, the modeling
  assumption that was made does not hold true in reality: Water
  droplets, a photon's scatterer, are not distributed randomly.
  Within clouds, droplets cluster together in some regions while
  being completely absent in others. Consequently, there is an
  increased (decreased) probability that a particle will scatter if
  it has (has not) scattered recently, rendering the  dynamics
  non-Markovian. As such, the process can not be described with
  classical transport equations.

  \begin{figure}
    \centering
    \includegraphics[width=1.0\linewidth]{Chapters/Chapter4/figs/cloudfraction/v8cloudfraction.pdf}
    \caption{For the month of July 2020, the image shows the averaged
      cloud fraction with a resolution of $0.25$ degrees. The data was
      recorded with the Aqua/MODIS satellite by NASA Earth
      Observations. This visualization reconstructs the image by Reto
      Stockil, hosted on the NASA website \cite{nasaclouds} (together
    with the corresponding data).}
    \label{fig:cloudfraction}
  \end{figure}

\end{exmp}

\begin{exmp}[Pebble-bed reactors]
  \enquote{Nuclear power generated around $10\%$ of the world’s
  electricity in 2019,}
  reads a report by the International Atomic Energy Agency.
  Even though nuclear energy is debated controversially, entirely
  undisputed, however, is the demand that its mathematical modeling
  and related predictions be accurate.

  Pebble-bed reactors (PBR) are a particular kind of
  graphite-moderated, gas-cooled, very-high-temperature reactors.
  They get their name from the spherical  pebbles that fuel the
  reactor. Stacked randomly on top of another, thousands of these
  pebbles are placed inside a cylindrical reactor core. With an
  average packing fraction of $81.7\%$, radiation is modeled by the
  diffusion approximation which uses the atomic mix model to compute
  the diffusion coefficient as a weighted average of a pebble’s cross
  section (weighted with $0.817$) and the cross section of vacuum
  (weighted with $0.183$)  \cite{vasques2009anisotropic,
  larsen2011generalized, vasques2014non}.

  When compared with Monte Carlo simulations, the diffusion
  approximation’s error is small but non-negligible. While it might
  be argued that the discrepancy between both simulations is due to
  the inapplicability of the diffusion approximation in that
  particular regime of transport, the asymptotic diffusion limit of a
  \textit{non-classical} transport equation reduces the error by a
  factor two to five for the same setup \cite{vasques2014non}.
  Moreover, when allowing for angular-dependent and non-exponential
  path length distributions, even asymmetries due to gravity (which,
  in three dimensions, acts only in the $z$-direction) can be
  incorporated sufficiently well into non-classical equations.
  Ultimately, the reason why non-classical equations predict
  radiation in PBRs more accurately is that a particle’s path lengths
  are not exponentially distributed due to the way the pebbles are stacked.

\end{exmp}

\begin{figure}
  \centering
  \includegraphics[width=1.0\linewidth]{Chapters/Chapter4/figs/bitterli_et_al/WDAS-Cloud.png}
  \caption{A cloud rendered with a non-exponentially decaying path
    length distribution. Light penetrates deeper than in the case of
    exponentially decaying path lengths.
    This image was provided by Bitterli and a side-by-side comparison
    of classical and non-classical renders can be found in Bitterli et
    al. \cite{bitterli2018radiative}. Heterogeneity is introduced into
    the simulation by assuming a statistically heterogeneous material
    with identical correlation, similar to Camminady et al.
  \cite{camminady2017nonclassical}.}
  \label{fig:nec}
\end{figure}

\begin{exmp}[Non-classical transport in the field of computer graphics]

  Produced by Pixar Animation Studios and released by Walt Disney
  Pictures in 1995, \textit{Toy Story} was the first-ever entirely
  computer-animated feature film. Its success---Toy Story was praised
  both by the audience and critics and achieved four Academy Awards
  in 1996---arguably propelled the animation industry as a whole. The
  high degree of realism, albeit multifaceted, is largely contributed
  to the (for its time) unparalleled sophistication of Pixar’s
  rendering software RenderMan. For objects to look real, not only
  had their movements to be modeled correctly but also every frame’s
  illumination. Since light transport is described by kinetic
  theory’s transport equations, there exists an inherent link between
  the work discussed thus far and that in the field of computer graphics.

  While renders of hard, smooth surfaces---\textit{Cars}'s cars or
  \textit{Toy Story}'s toys---achieve photorealism, some notoriously
  difficult settings were only recently mastered. This includes,
  e.g., fur and hair, rough and translucent objects, skin,
  and---relevant for non-classical transport---clouds. In 2018,
  Bitterli et al. \cite{bitterli2018radiative} explored the
  possibility to model light transport through clouds with
  heavy-tailed transmittance curves which allowed for \enquote{both
    deeper light penetration as well as a softer appearance near the
  surface.} A resulting render is shown in Figure \ref{fig:nec}.



  %This matches the observation of our first example, i.e., transport
  % in clouds is non-classical.
  Moreover, non-classical transport increases artistic flexibility,
  useful to fine-tune renders to look more realistic or appealing by
  freely choosing the path length distribution. Since scenes in
  animation movies rarely consist of homogeneous, luminiferous
  aether, non-classical transport in presence of heterogeneities
  needs to be understood both from a theoretical and
  application-oriented viewpoint.




\end{exmp}












\section{Particle billiard redux} \label{sec:particlebilliardredux}
The particles' interactions with the background obstacles don't
differ in their dynamics---particles still undergo elastic,
instantaneous collisions---but in the way the scatterers are arranged.

With randomly distributed scatterers representing one end of the
spectrum of correlation, scatters that build a perfect lattice fall
on the other end; such a situation is depicted with a single
particle's trajectory in Figure \ref{fig:particlebilliardtableNC}.
To quantify the distribution of path lengths, a distinction needs to
be made that was not necessary for the random billiard table:
Tallying a single particle's trajectory for $n$ steps is different
from tallying $n$ particles' trajectories for a single step. For a
single particle, only one path length (the very first) can be smaller
than the minimal distance between two obstacles' boundaries.
Afterwards, a particle that reflects from an obstacle will not
encounter another obstacle before having traveled this minimum
distance. But the different statistics are not only limited to the
infinitesimally small region. For asymptotically large path lengths,
the two resulting distributions decay with different orders of
magnitude. These findings are summarized in Figure
\ref{fig:differentpathlengthsoverview}.
This difference in behavior requires a more precise language.

\begin{definition}[Distance to collision]
  A particle at $(\bx,\bOmega) \in \mathbb{R}^d \times
  \mathbb{S}^{d-1}$ is said to have distance to collision $s$, if it
  undergoes its next collision (where it scatters or gets absorbed)
  at $\bx + s\cdot\bOmega$.
\end{definition}
\begin{definition}[Distance since collision]
  A particle at $(\bx,\bOmega) \in \mathbb{R}^d \times
  \mathbb{S}^{d-1}$ is said to have distance since collision $s$, if
  it underwent its last collision (where it scattered) at $\bx - s\cdot\bOmega$.
\end{definition}

\begin{definition}[Distance since birth]
  A particle at $(\bx,\bOmega) \in \mathbb{R}^d \times
  \mathbb{S}^{d-1}$ is said to have distance since birth $s$, if, at
  $\bx - s\cdot \bOmega$, it entered the domain through the boundary
  or was emitted by a source.
\end{definition}

\begin{definition}[Distance between collisions]
  A particle is said to have distance between collisions $s$, if it
  travels a distance $s$ between a scattering event and the
  consecutive collision (where it scatters or gets absorbed).
\end{definition}

\begin{figure}
  \centering
  \includegraphics[width=1.0\linewidth]{Chapters/Chapter4/figs/billiardtable/particlebilliardtable}
  \caption{A non-classical billiard game where scatterers' centers
    are arranged according to a lattice structure. This setup is also
    known under the name \textit{periodic Lorentz gas}. The (solid)
  white ball's trajectory is shown.}
  \label{fig:particlebilliardtableNC}
\end{figure}

\begin{figure}
  \begin{subfigure}{\textwidth}
    \includegraphics[width=1.0\linewidth]{Chapters/Chapter4/figs/pathlengths/stepmany}
    \caption{Path lengths of $10^6$ particles from their birth (at a
      random position inside the lattice, not touching any obstacle) to
      their first collision. These path lengths are the
      \textit{distances to collisions}. Distances smaller than the
    lattice gap occur frequently.}
    \label{fig:many}
  \end{subfigure}
  \vfill
  \begin{subfigure}{\textwidth}
    \includegraphics[width=1.0\linewidth]{Chapters/Chapter4/figs/pathlengths/stepone}
    \caption{Path lengths of a single particle's trajectory  over the
      course of $10^6$ consecutive collisions. At most one (the
      first) path length is smaller than the lattice gap.
      Except for the first path length, these are the \textit{distances
      between collisions}. The result, especially the asymptotic
      region, looks very similar when the experiment is repeated with
    other random initial positions and velocities.}
    \label{fig:one}
  \end{subfigure}
  \vfill
  \begin{subfigure}{\textwidth}
    \includegraphics[width=1.0\linewidth]{Chapters/Chapter4/figs/pathlengths/both}
    \caption{Both images combined with zoom. The histogram is
    recomputed with matching binning.}
    \label{fig:combined}
  \end{subfigure}
  \caption{Path length distributions in the periodic Lorentz gas with
    $\sigma= 1.0$ and $10^6$ obstacles, which results in a radius of
    $2\cdot 10^{-6}$ for each obstacle. The lattice gap---the minimal
    distance between obstacles' boundaries---is $0.000996$. The first
  two plots use $100$ log-spaced bins, the last plot uses $30$.}
  \label{fig:differentpathlengthsoverview}
\end{figure}

\begin{figure}
  \begin{subfigure}{0.32\textwidth}
    \includegraphics[width=1.0\linewidth]{Chapters/Chapter4/figs/lattice/100}
    \caption{$n_\text{obs}=100$, $r=0.02$.}
    \label{fig:100}
  \end{subfigure}
  \hfill
  \begin{subfigure}{0.32\textwidth}
    \includegraphics[width=1.0\linewidth]{Chapters/Chapter4/figs/lattice/900}
    \caption{$n_\text{obs} = 900$, $r=0.00\bar{2}$.}
    \label{fig:900}
  \end{subfigure}
  \hfill
  \begin{subfigure}{0.32\textwidth}
    \includegraphics[width=1.0\linewidth]{Chapters/Chapter4/figs/lattice/8100}
    \caption{$n_\text{obs}=8100$, $r\approx0.0002469$.}
    \label{fig:8100}
  \end{subfigure}
  \caption{Trajectory of $2000$ particles from their birth (at a
    random position, not touching an obstacle) to their first
    collision. For all three images, $n_\text{obs} \cdot r$ is
    constant. The orientation of the lattice fades as we increase the
  number of obstacles, visible in the trajectories. The domains are periodic.}
  \label{fig:latticerefinement}
\end{figure}

Consequentially, Figures \ref{fig:many} and \ref{fig:one} show the
distances to collisions and the distances between collision (except
for the very first of the $10^6$ path lengths), respectively.

Similar to the random case, we can also perform a Boltzmann-Grad
limit for the periodic Lorentz gas. The details of this procedure and
the mathematical analysis, however, are beyond the scope of this work
and are explained thoroughly in the work by Golse et al.
\cite{golse2006periodic,golse2008periodic}. Moreover, non-classical
transport is not restricted to the periodic Lorentz gas but a more
general phenomena.
Nevertheless, two essential properties are exemplified. Both of these
can be understood without executing the Boltzmann-Grad limit in a
formal manner and instead following Terrence Tao's idea that
\enquote{[t]here's more to mathematics than rigor and proofs.}

First---though somewhat counterintuitive---it is worth noting that in
the limit of $r\rightarrow 0$ with $n_\text{obs}\cdot r =
\text{const.}$, transport in the periodic Lorentz gas is isotropic
since the effect of the lattice orientation vanishes. While the
lattice imposes a preference for particles to travel along horizontal
and vertical \textit{channels}, shrinking the obstacles' radii
effectively opens up more and more channels along all possible
directions. This is illustrated in Figure \ref{fig:latticerefinement},
where we see trajectories for particles in a periodic Lorentz gas
with radii $r=0.02$, $r=0.00\bar{2}$, and $r\approx 0.00025$,
respectively. The higher the number of obstacles, the smaller the
preference for horizontal and vertical directions.

Second---and while this property is not surprising, its consequences
are---the path length distributions (all four that were defined
earlier) are non-exponential. Consider (without loss of generality)
the distance since a particle's previous interaction. If the
respective probability were to be exponential,
\begin{align}
  \text{Pr}\left(S>s+\Delta s| S>s \right) = \text{Pr}\left(S>\Delta s\right).
\end{align}
However, since this is not the case,
\begin{align}
  \text{Pr}\left(S>s+\Delta s| S>s \right) \not =
  \text{Pr}\left(S>\Delta s\right),
\end{align}
and as a consequence particles must have a \textit{memory}. This
memory manifests itself in the fact that the distance traveled by a
particle influences its probability to undergo collisions in the
future. Thinking back to the example of the periodic Lorentz gas, a
particle that has already traveled a long distance might move through
one of the domain's channels and is therefore likely to continue
unhinderedly. Conversely, a particle that just underwent a collision
might be reflected towards a nearby obstacle, scattering again almost
immediately. Instead of expressing this memory in terms of the path
length distribution, an equivalent formulation makes the total cross
section dependent on the distance since collision $s$, since
\begin{align}
  \sigma_t(s) = \frac{p(s)}{1-\int_0^s p(s')\, ds'}.
\end{align}
It is easy to verify that this expression is constant if and only if
$p(s)$ is an exponential distribution. If, for example,
\begin{align}
  p(s) = s e^{-s},
\end{align}
then
\begin{align}
  \sigma_t(s) = \frac{s}{1+s}
\end{align}
and the likelihood to scatter increases as $s$ increases.
Alternatively, we can make an algebraic ansatz and assume that $p$ is
a Pareto distribution, \ie
\begin{align}
  p_{\alpha,x_m}(s) =
  \begin{cases}
    0 &\text{ if } s<x_m,\\
    \frac{\alpha x_m^\alpha}{x^{\alpha+1}} &\text{ if } s\geq x_m.
  \end{cases}
\end{align}
For $\alpha=2$ and $x_m=1$,
\begin{align}
  \sigma_t(s) =
  \begin{cases}
    0 &\text{ if } s<1, \\
    \frac{2}{s} &\text{ if } s\geq 1,
  \end{cases}
\end{align}
and particles are less likely to scatter the further they travel.
Whether the flexibility to choose a path length distribution is
reflected in a corresponding arrangement of obstacles is unclear and
it seems more likely that certain distributions are not the result of
a Boltzmann-Grad-like limit of an arrangement of obstacles.

Finally, to tease Chapter \ref{chap:heterogeneousnonclassical}, we
briefly mention non-classical transport in the presence of material
heterogeneities. Here, the following question arises: How do we
correctly encode the memory of particles as they transition between
the different materials? Given a lack of experimental evidence,
arguments for different models can be made---some of which will be
presented here.
In this chapter, however, we restrict ourselves to the homogeneous
case and are now prepared to analyze the governing equations of
homogeneous, non-classical transport.

\section{Generalized linear Boltzmann equations}
We start out simple and add more complexity as we go along. For now,
we omit the distinction between distance since collision and distance
since birth and write the generalized liner Boltzmann equation (GLBE)
in terms of the distance since the last interaction.  Similar to
Larsen and Vasques \cite{larsen2011generalized}, we make the
following definitions:

\begin{align*}
  {V} =
  & \text{ ``the spatial domain with boundary }\partial{V}."
  \\
  \bx =
  & \text{ ``the position }(x,y,z)\in {V}."
  \\
  {\bf{n}}(\bx) =
  & \, {\bf{n}} =\text{ ``the outward pointing normal vector at }\bx
  \in \partial{V}."
  \\
  \bOmega =
  & \text{ ``the direction of flight }(\Omega_x,\Omega_y,\Omega_z)
  \text { with unit speed.}"
  \\
  s =
  & \text{ ``the distance that a particle has traveled since the
  previous event,}                         \\ & \text{ \, be it birth
  or scattering.}"
  \\
  v=
  & \text{ ``the particle speed.}"
  \\
  N(s,\bx,\bOmega)\, ds\,d\bx\, d\bOmega =
  & \text{ ``the number of particles in }ds\, d{\bx}\, d\bOmega
  \text{ about } (s,\bx,\bOmega)."
  \\
  \psi(s,\bx,\bOmega) =                             & \,  v \,
  N(s,\bx,\bOmega)= \text{ ``the non-classical angular flux.}"
  \\
  \sigma_{t}(s)\, ds =                           & \text{ ``the
  probability that a particle which has  traveled a distance }s
  \\ & \text{ \, since a previous event experiences a collision while
  traveling }\\ &\text{ \, a further distance } ds."
  \\
  c =                                              & \text{ ``the
  probability that a particle that experiences a collision will scatter.}"
  \\
  s(\bOmega' \cdot \bOmega)\, d\bOmega =           & \text{ ``the
  probability that a particle  with pre-collision direction }
  \bOmega'                           \\ &\text{ \, scatters into a
  post-collision direction that lies in }d\bOmega \text{ about } \bOmega."
  \\
  \delta(s) =                                      & \text { ``the
  Dirac delta function evaluated at }s."
  \\
  Q(\bx)\,d{\bx} =                            & \text{ ``the rate at
  which particles are isotropically emitted by an internal source}
  \\
  & \text{ \, in } d{\bx} \text{ about } \bx."
  \\
  l(\bx,\bOmega) = &\text{ ``the distance to the domain's boundary
  from }\bx \text{ in direction } -\bOmega."
\end{align*}
Just like the time-dependent classical transport equation, the GLBE
follows from balancing the gain terms (in-scattering and source
contributions) with the loss terms (leakage and collisions). Since
particles with any value of $s$ can leak or scatter, but we only gain
particles with $s=0$, the gain terms need to be multiplied with the
Dirac delta function.
The result is then
\begin{equation}
  \begin{split}
    \underbrace{\partial_s \psi(s,\bx,\bOmega)}_\text{rate of change}
    =& -\underbrace{\bOmega \cdot \nabla_\bx \psi(s,\bx,\bOmega) -
    \sigma_t(s) \psi(s,\bx,\bOmega)}_\text{leakage and collision} \\
    &
    +\underbrace{\delta(s)  \int_0^{l(\bx,\bOmega)}
      \int_{\mathbb{S}^2} c \, \sigma_t(s') s(\bOmega' \cdot \bOmega)
      \, \psi(s',\bx,\bOmega')\, d\bOmega' \, ds'+ \delta(s)
    Q(\bx)}_\text{in-scattering and source}.
  \end{split}
\end{equation}
Written more recognizable, we define the generalized linear Boltzmann
equation, formulated in terms of the distance since an interaction, as
\begin{equation}
  \begin{split}
    \partial_s \psi(s,\bx,\bOmega) &+\bOmega \cdot \nabla_\bx
    \psi(s,\bx,\bOmega) + \sigma_t(s) \psi(s,\bx,\bOmega) \\
    &=\delta(s)  \int_0^{l(\bx,\bOmega)} \int_{\mathbb{S}^2} c \,
    \sigma_t(s') s(\bOmega' \cdot \bOmega) \, \psi(s',\bx,\bOmega')\,
    d\bOmega' \, ds'+ \delta(s) Q(\bx).
  \end{split}
  \label{eq:GLBE_backward}
\end{equation}
Some remarks are noteworthy.
%\begin{remark}
%  Together with an integral equation formulation and an asymptotic
% diffusion limit,
%  the above formulation of the GLBE dates back to work by Larsen and
% Vasques [XXX]. Both publications motivate the non-classical
% transport with the aforementioned observations and measurements in clouds.
%  Angular anisotroy has been subsequently incorporated in Vasques
% Larsen , ......
%\end{remark}
\begin{remark}
  This is the \textit{backward} formulation of the GLBE. Backward in
  the sense that, given any tuple $(s,\bx,\bOmega)$, we do know where
  a particle originated (at $\bx - s\cdot\bOmega$), but not where it
  will collide.
\end{remark}
\begin{remark}
  Equation \eqref{eq:GLBE_backward} looks similar to the
  time-dependent classical transport equation, but with a
  time-dependent cross section and additional delta functions on the
  right-hand side. Indeed, if it weren't for the delta functions, the
  backward GLBE could be solved with a standard time-dependent
  transport code that allows to incorporate a dependency on $s$ in
  the cross section. But because particles have their memory reset
  after a collision or when they emerge from a source, a numerical
  algorithm like S$_N$ that evolves transport forward in time does
  not immediately work here. The Monte Carlo method, however, can be
  adapted easily.
\end{remark}
\begin{remark}
  When generalizing an equation, it should always reduce to its
  simpler form if the assumptions are the same. This is the case in
  \eqref{eq:GLBE_backward} with $\sigma_t(s) = \sigma_t$ independent
  of $s$, followed by operating with $\int_{-\varepsilon}^\infty
  \cdot \, ds$ and letting $\varepsilon \rightarrow 0$, \ie
  \begin{equation}
    \begin{split}
      \lim_{\varepsilon \rightarrow 0} \int_{-\varepsilon}^\infty \,
      \partial_s \psi(s,\bx,\bOmega) &+\bOmega \cdot \nabla_\bx
      \psi(s,\bx,\bOmega) + \sigma_t \psi(s,\bx,\bOmega)  \, ds\\
      &=
      \lim_{\varepsilon \rightarrow 0}
      \int_{-\varepsilon}^\infty\delta(s)  \int_0^{l(\bx,\bOmega)}
      \int_{\mathbb{S}^2} c \, \sigma_t s(\bOmega' \cdot \bOmega) \,
      \psi(s',\bx,\bOmega')\, d\bOmega' \, ds'+ \delta(s) Q(\bx) \, ds.
    \end{split}
    \label{eq:GLBE_backward_reduces_to_classical}
  \end{equation}
  Since the classical angular (and scalar) flux is agnostic to $s$,
  we have $\psi(\bx,\bOmega) = \int_0^\infty \psi(s,\bx,\bOmega) \,
  ds$\footnote{The non-classical angular flux outside the spatial
    domain is zero and we therefore write $\int_0^\infty \cdot \,ds$
  instead of $\int_0^{l(\bx,\bOmega)} \cdot \, ds$. }. Equation
  \ref{eq:GLBE_backward_reduces_to_classical} is then equivalent to
  \begin{equation}
    \begin{split}
      \lim_{\varepsilon \rightarrow 0}
      \psi(s,\bx,\bOmega)
      \bigg\rvert_{s=-\varepsilon}^{s=\infty}&+\bOmega \cdot
      \nabla_\bx \psi(\bx,\bOmega) + \sigma_t \psi(\bx,\bOmega) \\
      &=
      c \, \sigma_t \,    \int_{\mathbb{S}^2}  s(\bOmega' \cdot
      \bOmega) \, \psi(\bx,\bOmega')\, d\bOmega' +  Q(\bx) ,
    \end{split}
    \label{eq:GLBE_backward_reduces_to_classical_reduced}
  \end{equation}
  which---because the first term vanishes---is exactly the classical,
  time-independent transport equation. %\todo{WHY do the boundary terms vanish?}
\end{remark}

Expressing transport in terms of the distance to the next collisions,
we can alternatively derive a \textit{forward} formulation of the
GLBE. Just like for the \textit{backward} formulation, the phase
space of the forward formulation is augmented by one additional
variable when compared to the classical transport equation. To
distinguish both formulations, we use $f(s,\bx,\bOmega)$ for the
forward formulation, with $s$ denoting the distance to the next
interaction. Consequently, all definitions that were made at the
beginning of this section still hold true for the forward formulation
when \textit{distance since an interaction} is replaced by
\textit{distance to an interaction}.

However, now that the memory is encoded differently, the transport
equation, too, changes.
Representing the distance to the next interaction, $s$ has to
decrease as particles move  through space.
Moreover, the distance to the next interaction has to be known
\textit{at the beginning} of a particle's trajectory, sampled from a
probability density function. As a result, we know that all particles
represented by the infinitesimal phase space element
$f(s,\bx,\bOmega) \, ds \, d\bx \, d\bOmega$ are (with absolute
certainty) undergoing an interaction inside $\left[\bx+s \,
\bOmega,(\bx+d\bx)+(s+ds)\,  (\bOmega + d\bOmega)\right]$. Despite
the precise knowledge of future interactions' locations, the forward
formulation is of the same random nature as the backward formulation.
Even though the location of an upcoming interaction is deterministic,
the distance to the next interaction is sampled from a probability
density function.

\begin{minipage}[t!]{0.99\textwidth}
  \centering
  \begin{tabular}{cccc}\hline
    & the \textit{backward} formulation & the \textit{forward} formulation \\
    \hline \hline
    additional variable & $s$, distance since interaction & $s$,
    distance to interaction \\
    variable behavior & increases, initialized as $s=0$ & decreases,
    sampled from $p(s)$\\
    deterministic component & value of $s$ after collision known &
    next collision's location known  \\
    random component & collisions possible for any value of $s$ &$s$
    is sampled  randomly\\
    $(s,\bx,\bOmega )$ implies & particles originate at $\bx-s\,
    \bOmega$ & particles interact at $\bx + s \, \bOmega$ \\
    boundary conditions &$\psi(s,\bx,\bOmega) = \delta(s) \cdot
    g(\bx,\bOmega) $ & $f(s,\bx,\bOmega) = p(s) \cdot g(\bx,\bOmega)$ \\
    \hline
  \end{tabular}
  \captionof{table}{Juxtaposition of the forward and backward formulation.}
  \label{tab:forwardvsbackwards}
\end{minipage}

The forward formulation of the generalized linear Boltzmann equation
(omitting boundary contributions) is then given by
\begin{align}
  -\partial_s f(s,\bx,\bOmega) + \bOmega \cdot \nabla_\bx f(s,\bx,\bOmega)
  = p(s) c \int_{\mathbb{S}^2}s(\bOmega' \cdot \bOmega)
  f(0,\bx,\bOmega') \, d\bOmega' + p(s) Q(\bx).
\end{align}

To summarize the differences and similarities, Table
\ref{tab:forwardvsbackwards} juxtaposes both formulations in a
condensed form. We also include the respective boundary conditions,
prescribed for all incoming angles at the domain's boundary, \ie for
all $\bx \in \partial V$ and $\bOmega \in \mathbb{S}^2$ where
$\bOmega \cdot \mathbf{n}(\bx) < 0$.

Equivalence of both formulations has been proven in Larsen et al.
\cite{larsen2017equivalence}, using the transformation
\begin{align}
  f(s,\bx,\bOmega) \equiv \int_0^{l(\bx,\bOmega)}\sigma_t(s+s') \psi
  \left(s+s',\bx+s \, \bOmega, \bOmega\right) \, ds'
  \label{eq:trafo}
\end{align}
and the relation between cross section and path length distribution
\begin{align}
  \sigma_t(s) = \frac{p(s)}{\int_s^\infty p(s')\, ds'} \quad \text{ or }\quad
  p(s) = \sigma_t(s) \exp\left(-\int_0^s \sigma_t(s') \, ds'\right).
  \label{eq:trafo2}
\end{align}
The forward formulation inserts randomness into the dynamics by
sampling a particle's distance to the next interaction from $p(s)$;
from that point on, the particle's trajectory is deterministic.
Conversely, the backward formulation always resets the value of $s$
after a collision and inserts randomness by allowing particles to
scatter for any value of $s$, based on $\sigma_t(s)$. It is therefore
naturally, to link $p(s)$ to the distance to the next interaction
(\ie the forward formulation), while $\sigma_t(s)$ refers to the
distance since the last interaction (\ie the backward formulation).
Unless explicitly stated otherwise, we will from now on implicitly
assume that $s$
is to be interpreted in the forward or backward formulation, based on
the choice of $p(s)$ or $\sigma_t(s)$, respectively.

We are not going to recapitulate the thorough analysis of Larsen et
al. \cite{larsen2017equivalence} that proofs equivalence here, but
rather try to provide a physical interpretation of the involved
terms, demonstrating that both sides of \eqref{eq:trafo} indeed
express the same physical quantity. The left-hand side is easy to
understand: $f(s,\bx,\bOmega)$ quantifies the amount of particles
that pass through $(\bx,\bOmega)$ to undergo a collision at $\bx +
s\, \bOmega$. Next, $\psi(s+s', \bx + s \, \bOmega, \bOmega)$
represents the particles at $(\bx + s \, \bOmega, \bOmega)$ that have
traveled a distance $s+s'$ from $\bx - s\, \bOmega$ to reach this
point, passing through $(\bx,\bOmega)$ along that journey.
Multiplied by $\sigma_t(s+s')$, we obtain the fraction of those
particles that do undergo a collision at $\bx+s\, \bOmega$.
Integrating over all distance $s'$ from $\bx$ back to the domain
boundary at $\bx - l(\bx,\bOmega) \, \bOmega$, we obtain \textit{all}
particles that travel through $(\bx,\bOmega)$ and collide at $\bx + s
\bOmega$, \ie the same quantity as the left-hand side.

%re is no reason to assume that the forward and backward formulations
% of non-classical transport

%\todo{finish this by indicating that many different formulations
% exist. mention also integral transport etc etc.}


\section{Correlated and uncorrelated particles}
\label{sec:corr}

A thought (and numerical) experiment made by d'Eon \cite{d2018reciprocal} that was independently considered in Jarabo et al. \cite{jarabo2018radiative} made the following observation: 
For non-classical particle transport it is necessary to distinguish particles that are \textit{correlated} with the medium (related quantities are denoted with a subscript $c$) from those particles that are \textit{uncorrelated} with the medium (denoted with a subscript $u$).
%\footnote{
%	This terminology is similar to that of \textit{collided }and \textit{uncollided }particles in the transport community.
%	However, we will learn about mechanisms that allow correlated particles to become uncorrelated. 
	%Since collided particles can not become uncollided, and given that this was the notation adapted by the original authors, we will also use this notation here.}.
This distinction is explained by considering particle transport in an obstacle field generated via a Poisson disc sampling procedure. Rather than placing scatterers randomly---as it is the assumption in classical transport theory---we place scatterers in a way such that they obey a certain minimal distance from any other scatterer, visualized in Figures \ref{fig:sub1} and \ref{fig:sub2}, respectively.
The distributions for the \textit{distance between  consecutive collisions} has to be zero for distances smaller than the minimal distance between scatterers. However, for a particle that is placed independently of the scatters---\eg by a source---the distribution for the \textit{distance to the next collision} does not demand this property.

Consequently, we distinguish correlated  from uncorrelated particles. These two species then behave as follows:
\begin{itemize}
	\item[1)] We gain correlated (or uncorrelated) particles by a correlated (or uncorrelated) source.
	\item[2)] If correlated (or uncorrelated) particles get absorbed we lose correlated (or uncorrelated) particles.
	\item[3)] If correlated particles scatter they stay correlated.
	\item[4)] However, if uncorrelated particles scatter, they are no longer independent of the medium and become correlated particles.
\end{itemize}



\begin{figure}
	\centering
	\begin{subfigure}{.45\textwidth}
		\centering
		\includegraphics[width=.9\linewidth]{Chapters/Chapter4/figures/random}
		\caption{Scatterers generated in a uniformly random manner. Scatterers are allowed to overlap.}
		\label{fig:sub1}
	\end{subfigure}%
	\hfill
	\begin{subfigure}{.45\textwidth}
		\centering
		\includegraphics[width=.9\linewidth]{Chapters/Chapter4/figures/poisson}
		\caption{Scatterers generated via Poisson disc sampling where the minimal distance between two boundaries is equal to the radius $r$.}
		\label{fig:sub2}
	\end{subfigure}
	\caption{Scatterers of radius $r=0.01$ generated in the domain $[0,1]\times[0,1]$. Both pictures show $696$ scatterers.}
	\label{fig:test}
\end{figure}
\begin{figure}
	\centering
	\includegraphics[width=1\linewidth]{Chapters/Chapter4/figures/flowchart1.pdf}
	\caption{Life cycle of correlated and uncorrelated particles.}
	\label{fig:photo2019-12-1617-13-40}
\end{figure}
Ignoring heterogeneity, this life cycle is depicted in Figure  \ref{fig:photo2019-12-1617-13-40}.
We are aware that the above distinction is similar to the one of \textit{collided} and \textit{uncollided} particles. 
In fact, for the homogeneous case, uncorrelated particles are uncollided ones and correlated particles are collided ones. For the heterogeneous case, however, correlated particles can become uncorrelated, whereas collided particles always stay collided. We make the following definitions:


\begin{align*}
	N_u(s,\bx,\bOmega)\, ds\, d{\bx}\, d\bOmega = 
	& \text{ ``the number of uncorrelated particles in } ds\, d{\bx}\, d\bOmega\text{ about } (s,\bx,\bOmega)." 
	\\
	\psi_u(s,\bx,\bOmega) =                             & \,  v \, N_u(s,\bx,\bOmega)= \text{ ``the non-classical, uncorrelated angular flux.}"                                        
	\\
	\sigma_{u,t}(s)\, ds =                           & \text{ ``the probability that an uncorrelated particle which has  traveled}                         \\ & \text{ \, a distance }s\text{ since a previous event experiences a collision }\\ &\text{ \, while traveling a further distance } ds."                          
	\\
	Q_u(\bx)\,d{\bx} =                            & \text{ ``the rate at which uncorrelated particles are isotropically emitted by }                       
	\\
	& \text{ \, an internal source in } d{\bx} \text{ about } \bx."          
\end{align*}
Similarly, we obtain the quantities for correlated particles.
\begin{align*}
	N_c(s,\bx,\bOmega)\, ds\,d{\bx}\, d\bOmega = 
	& \text{ ``the number of correlated particles in } ds\, d{\bx}\, d\bOmega \text{ about } (s,\bx,\bOmega)." 
	\\
	\psi_c(s,\bx,\bOmega) =                             & \,  v \, N_c(s,\bx,\bOmega)= \text{ ``the non-classical, correlated angular flux.}"                                        
	\\
	\sigma_{c,t}(s)\, ds =                           & \text{ ``the probability that a correlated particle which has  traveled}                         \\ & \text{ \, a distance }s\text{ since a previous event experiences a collision }\\ &\text{ \, while traveling a further distance } ds."                          
	\\
	Q_c(\bx)\,d{\bx} =                            & \text{ ``the rate at which correlated particles are isotropically emitted by }                       
	\\
	& \text{ \, an internal source in } d{\bx} \text{ about } \bx."          
\end{align*}

Supplementing the definitions above, we remark that (i) these equations describe homogeneous transport, (ii) we only choose isotropic sources to shorten notation, and (iii) we subsequently assume the absence of sources that are correlated with the medium on a microscopic level. 
Following Larsen and Vasques \cite{larsen2011generalized}, we obtain versions of the generalized linear Boltzmann equation, provided first in a similar form by d'Eon \cite{d2018reciprocal}. They are given by
\begin{align}
	\partial_s \psi_u(s,\bx,\bOmega) + \bOmega\cdot \nabla_\bx \psi_u(s,\bx,\bOmega) & + \sigma_{t,u}(s)\psi_u(s,\bx,\bOmega) = \delta(s) Q_u(\bx)
	\label{eq:uncorrelatedtranspor}
\end{align}
for the uncorrelated angular flux and
\begin{equation}
	\begin{split}
		\partial_s \psi_c(s,\bx,\bOmega) &+ \bOmega\cdot \nabla_\bx \psi_c(s,\bx,\bOmega) + \sigma_{t,c}(s)\psi_c(s,\bx,\bOmega)  \\=\, &c\,\delta(s)\, \int_0^{l(\bx,\bOmega)} \int_{\mathbb{S}^2}
		s(\bOmega' \cdot \bOmega)\left( \sigma_{t,c}(s')\psi_c(s',\bx,\bOmega')
			+ \sigma_{t,u}(s')\psi_u(s',\bx,\bOmega') \right) \, d\bOmega'\, ds'
	\end{split}
	\label{eq:correlatedtranspor}
\end{equation}
for the correlated angular flux, respectively. With $l(\bx,\bOmega)$ we denote the distance from $\bx$ to $\partial{V}$  moving backward with direction $-\bOmega$. Boundary conditions are given by
\begin{subequations}
	\begin{align}
		%{\bf{n}}(\bx) \cdot\nabla 
		\psi_c(0,\bx,\bOmega) & = 0                              & \text{ for }\bx\in  \partial{V}, \,{\bf{n}}(\bx) \cdot \bOmega<0, & \\
		%{\bf{n}}(\bx) \cdot \nabla 
		\psi_u(s,\bx,\bOmega) & = \delta(s)\psi_u^{bc}(\bx,\bOmega) & \text{ for }\bx\in  \partial{V}, \,{\bf{n}}(\bx) \cdot \bOmega<0, &
		%{\bf{n}}(\bx) \cdot \nabla 
		%f_{i}(\bx,\bOmega,s) &= 0 & \text{ for }\bx\in  \partial{V}, \,{\bf{n}}(\bx) \cdot \bOmega\geq 0, &\, i\in\{u,c\}. DO I NEED THIS
		\label{eq:bc}
	\end{align}
\end{subequations}
with a prescribed function $\psi_u^{bc}(\bx,\bOmega)$.
This means that particles entering the domain cannot be correlated since, by definition, they have not yet interacted with the obstacles. There is only a flux of uncorrelated particles entering the domain with $s=0$.

Since \eqref{eq:uncorrelatedtranspor} can be solved independently of \eqref{eq:correlatedtranspor}, it is possible to solve \eqref{eq:uncorrelatedtranspor} for $\psi_u(s,\bx,\bOmega)$ first and rewrite  \eqref{eq:correlatedtranspor} as
\begin{equation}
	\begin{split}
		\partial_s \psi_c(s,\bx,\bOmega) &+ \bOmega\cdot \nabla_\bx \psi_c(s,\bx,\bOmega) + \sigma_{t,c}(s)\psi_c(s,\bx,\bOmega)  \\=\, &c\,\delta(s)\, \int_0^{l(\bx,\bOmega)}  \int_{\mathbb{S}^2}
		s(\bOmega' \cdot \bOmega)\sigma_{t,c}(s')\psi_c(s',\bx,\bOmega') \, d\bOmega'\, ds'
		+ Q_c(s,\bx,\bOmega),
	\end{split}
	\label{eq:correlatedtransporreqritten}
\end{equation}
with
\begin{equation}
	\begin{split}
		Q_c(s,\bx,\bOmega)=\, &c\,\delta(s)\, \int_0^{l(\bx,\bOmega)}  \int_{\mathbb{S}^2}
		s(\bOmega' \cdot \bOmega)\sigma_{t,u}(s')\psi_u(s',\bx,\bOmega') \, d\bOmega'\, ds'.
	\end{split}
	\label{eq:definitionqc}
\end{equation}

In the case of classical transport, \ie $\sigma_{t,c}(s) = \sigma_{t,u}(s) = \sigma_t$ independent of $s$, the equations reduce to the classical linear transport equation.

Additionally, the necessity for splitting particles into correlated and uncorrelated ones
vanishes if $\sigma_{t,c}(s) = \sigma_{t,u}(s) = \sigma_t(s)$, resulting in the generalized linear Boltzmann equation derived by Larsen and Vasques
\cite{larsen2011generalized} with the non-classical angular flux satisfying $\psi(s,\bx,\bOmega) = \psi_u(s,\bx,\bOmega) + \psi_c(s,\bx,\bOmega) $.

Jarabo et al. \cite{jarabo2018radiative} and d'Eon \cite{d2018reciprocal} omitted the consideration of heterogeneous non-classical transport. 
To extend their analysis to the heterogeneous case, we first need to introduce the concept of heterogeneity with all its subtleties. Correlated and uncorrelated transport in heterogeneous materials is then reconsidered in Section \ref{sec:transporteqs}.



%

\section{Solution methods}
\label{sec:Solution methods}
We will now shortly comment on numerical solution methods. The forward formulation seems favorable for Monte Carlo method and the backward formulation suggests time-dependent deterministic solver. However, existing solvers in both scenarios will have to be adapted to treat heterogeneous non-classical transport.
\subsection{Monte Carlo method}
\label{sec:SolutionMC}
Modifying existing Monte Carlo solver to use the forward formulation should be straight forward. In addition to new sampling routines, we only require a ray-tracing routine that tracks whether particles cross interfaces. Given the probability density functions for the distance to the next collision $P_c^i(s)$ and $P_u^i(s)$, particles can be tracked inside a homogeneous domain. When these particles cross a domain interface, the distance to the next collision simply gets re-sampled from the domain's probability density functions that these particles enter.
This however assumes the presence of efficient sampling routines for non-exponential distributions. If these sampling routines are not efficiently implementable, a performance loss is to be expected when comparing classical with non-classical transport.


\subsection{Deterministic methods}
Due to the similarity between stationary non-classical transport and time-dependent classical transport, it seems natural to use classical time-dependent solver.
However, time-independent non-classical heterogeneous transport differs from time-dependent classical (heterogeneous) transport: Whereas the time $t$ is only increasing, the distance since the last event $s$ is reset at an interface. Consequently, time-dependent solver that march forward in time have to be modified.

One possibility to overcome this problem might be an iterative approach. For every iteration, we take the angular flux from the previous iteration for the evaluation of the interface conditions.
{\color{red} Does this converge? Probably?}

