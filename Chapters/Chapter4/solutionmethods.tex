

\section{Solution methods}
\label{sec:Solution methods}
We will now shortly comment on numerical solution methods. The forward formulation seems favorable for Monte Carlo method and the backward formulation suggests time-dependent deterministic solver. However, existing solvers in both scenarios will have to be adapted to treat heterogeneous non-classical transport.
\subsection{Monte Carlo method}
\label{sec:SolutionMC}
Modifying existing Monte Carlo solver to use the forward formulation should be straight forward. In addition to new sampling routines, we only require a ray-tracing routine that tracks whether particles cross interfaces. Given the probability density functions for the distance to the next collision $P_c^i(s)$ and $P_u^i(s)$, particles can be tracked inside a homogeneous domain. When these particles cross a domain interface, the distance to the next collision simply gets re-sampled from the domain's probability density functions that these particles enter.
This however assumes the presence of efficient sampling routines for non-exponential distributions. If these sampling routines are not efficiently implementable, a performance loss is to be expected when comparing classical with non-classical transport.


\subsection{Deterministic methods}
Due to the similarity between stationary non-classical transport and time-dependent classical transport, it seems natural to use classical time-dependent solver.
However, time-independent non-classical heterogeneous transport differs from time-dependent classical (heterogeneous) transport: Whereas the time $t$ is only increasing, the distance since the last event $s$ is reset at an interface. Consequently, time-dependent solver that march forward in time have to be modified.

One possibility to overcome this problem might be an iterative approach. For every iteration, we take the angular flux from the previous iteration for the evaluation of the interface conditions.
{\color{red} Does this converge? Probably?}
